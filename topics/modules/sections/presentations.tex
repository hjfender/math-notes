\subsection{Presentations}

\subsubsection{Torsion}\label{torsion}
An element $m \in M$ of an $R$-module is a \emph{torsion} element if $\{ m \}$ is linearly dependent.\newline

\noindent The subset of torsion elements of $M$ is denoted Tor$_R(M)$. \newline

\noindent A module $M$ is \emph{torsion-free} if Tor$_R(M) = \{ 0 \}$. \newline

\noindent A \emph{torsion} module is a module $M$ in which every element is a torsion element.

\begin{lemma}
Submodules and direct sums of torsion-free modules are torsion-free. Free modules over an integral domain are torsion-free.
\end{lemma}

\subsubsection{Cyclic}\label{cyclicmodule}
An $R$-module $M$ is \emph{cyclic} if it is generated by a singleton, that is, if $M \cong R/I$ for some ideal $I$ of $R$.

\begin{lemma}
Let $R$ be an integral domain. Assume that every cyclic $R$-module is torsion-free. Then $R$ is a field.
\end{lemma}

\subsubsection{Annihilator}\label{annihilator}
The \emph{annihilator} of an $R$-module $M$ is\dots
$$\textrm{Ann}_R(M) := \{ r \in R | \forall m \in M, rm = 0 \}.$$

\subsubsection{Definition of Presentation}\label{presentation}
An $R$-module $M$ is finitely presented if for some positive integers $m,n$ there is an exact sequence\dots
$$R^n \xrightarrow[]{\varphi} R^m \longrightarrow M \longrightarrow 0.$$
Such a sequence is called a \emph{presentation} of $M$.

\begin{lemma}
If $R$ is a Noetherian ring, then every finitely generated $R$-module is finitely presented.
\end{lemma}

\begin{lemma}
Let $A,B$ be matrices with entries in an integral domain $R$, and let $M$, $N$ denote the corresponding $R$-modules. Then
$M \oplus N$ corresponds to the block matrix\dots
\[
\begin{pmatrix}
A & \rvline & 0\\ \hline
0 & \rvline & B
\end{pmatrix}.
\]
\end{lemma}

\subsubsection{Resolution}\label{resolution}
A \emph{resolution} of an $R$-module $M$ by finitely generated free modules is an exact complex\dots
$$\dots \longrightarrow R^{m_3} \longrightarrow R^{m_2} \longrightarrow R^{m_1} \longrightarrow R^{m_0} \longrightarrow M \longrightarrow 0.$$

\begin{proposition}[Characterization of a Field]
Let $R$ be an integral domain. Then $R$ is a field if and only if every finitely generated $R$-module is free. That is for a finitely generated $R$-module $M$
there is an exact sequence\dots
$$0 \longrightarrow R^m \longrightarrow M \longrightarrow 0.$$
\end{proposition}

\begin{proposition}[Half Characteriztion of a PID]
Let $R$ be an integral domain. Suppose that for every finitely generated $R$-module $M$ and every begining
of a free resolution $R^{m_0} \xrightarrow[]{\pi} M \rightarrow 0$ there is an integer $m_1$ and an $R$-module homomorphism
$R^{m_1} \rightarrow R^{m_0}$ such that the sequence\dots
$$0 \longrightarrow R^{m_1} \longrightarrow R^{m_0} \xrightarrow[]{\pi} M \longrightarrow 0$$
is exact. Then $R$ is a PID.
\end{proposition}

\begin{proposition}
Let $A$ be a matrix with entries in an integral domain $R$, and let $B$ be obtained from $A$ by any sequence of the following operations:
\begin{itemize}
  \item switch two rows or two columns;
  \item add to one row (resp., column) a multiple of another row (resp., column);
  \item multiply all entries in one row (or column) by a unit $R$;
  \item if a unit is the only nonzero entry in a row (or column), remove the row and column containing that entry.
\end{itemize}
Then $B$ represents the same $R$-module as $A$, up to isomorphism.
\end{proposition}