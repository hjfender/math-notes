\subsection{Submodules}\label{submodules}
A \emph{submodule} $N$ of an $R$-module $M$ is an $R$-module such that the inclusion $N \subseteq M$ is an
$R$-module homomorphism.

\subsubsection{Generated Submodules}\label{generatedsubmodules}
Let $M$ be an $R$-module, and let $A \subseteq M$. By the universal property of free modules, there is a unique homomorphism of $R$-modules\dots
$$\varphi_A : R^{\oplus A} \rightarrow M.$$

\noindent The \emph{submodule generated by} $A$ in $M$, denoted $\langle A \rangle$, is the image of this homomorphism. \newline

\noindent Thus\dots
$$\langle A \rangle = \{\sum_{a \in A} r_a a | r_a \neq 0 \textrm{ for only finitely many elements } a \in A \}.$$

\subsubsubsection{Finitely Generated}\label{finitelygeneratedmodule}
The module $M$ is \emph{finitely generated} if $M = \langle A \rangle$ for a \emph{finite} set $A$.\newline

\noindent Alternatively, the module $M$ is \emph{finitely generated} if there is an onto homomorphism of $R$-modules\dots
$$R^{\oplus n} \twoheadrightarrow S.$$

\subsubsection{Noetherian Modules}\label{noetherianmodules}
An $R$-module $M$ is \emph{Noetherian} if every submodule of $M$ is finitely generated as an $R$-module.

\begin{proposition}
Let $M$ be an $R$-module, and let $N$ be a submodule of $M$. Then $M$ is Noetherian if and only if both $N$ and $M/N$
are Noetherian.
\end{proposition}

\begin{proof}
If $M$ is Noetherian, then so is $M/N$, and so if $N$ (because every submodule of $N$ is a submodule of $M$, so
it is finitely generated because $M$ is Noetherian).

For the converse, assume $N$ and $M/N$ are Noetherian, and let $P$ be a submodule of $M$; we have to prove that $P$ is finitely generated.
Since $P\cap N$ is a submodule of $N$ and $N$ is Noetherian, $P \cap N$ is finitely generated. Thus\dots
$$\frac{P}{P \cap N} \cong \frac{P + N}{N},$$
and hence $P / (P \cap N)$ is isomorphic to a submodule of $M/N$. Since $M/N$ is Noetherian, this shows that $P/(P \cap N)$ is finitely generated.

It follows that $P$ itself is finitely generated.
\end{proof}

\begin{corollary}
Let $R$ be a Noetherian ring, and let $M$ be a finitely generated $R$-module. Then $M$ is Noetherian (as an $R$-module).
\end{corollary}

\begin{proof}
Indeed, by hypothesis there is an onto homomorphism $R^{\oplus n} \twoheadrightarrow M$ of $R$-modules; hence $M$ is isomorphic
to a quotient of $R^{\oplus n}$. By the previous proposition, it suffices to prove that $R^{\oplus n}$ is Noetherian.

This may be done by induction. The statement is true for $n = 1$ by hypothesis. For $n > 1$, assume we know that $R^{\oplus(n-1)}$ is Noetherian;
since $R^{\oplus(n-1)}$ may be viewed as a submodule of $R^{\oplus n}$, in such a way that\dots
$$\frac{R^{\oplus n}}{R^{\oplus(n-1)}} \cong R,$$
and $R$ is Noetherian, it follows that $R^{\oplus n}$ is Noetherian, again by applying the previous proposition.
\end{proof}