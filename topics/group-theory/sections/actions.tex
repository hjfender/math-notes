\subsection{Group Actions}\label{groupactions}

An \hyperref[categoricalgroupaction]{action} of a group $G$ on a set $A$ is a set-function\dots
$$\rho : G \times A \rightarrow A$$
such that $\rho (e_G, a) = a$ for all $a \in A$ and\dots
$$(\forall g,h \in G), (\forall a \in A) : \rho(gh,a) = \rho(g, \rho(h,a)).$$

\subsubsection{Natural Action}\label{naturalgroupaction}
Every group $G$ acts in a natural way on the underlying set $G$. The action $\rho : G \times G \rightarrow G$ is simply
the operation in the group\dots
$$(\forall g,a \in G): \rho(g,a) = ga$$

\begin{theorem}[Cayley's theorem]
\label{cayleystheorem}
Every group acts \hyperref[faithful]{faithfully} on some set. That is, every group may be realized as a subgroup of a permutation group.
\end{theorem}

\begin{proof}
The natural action acts faithfully on $Aut_{Set}(G)$.
\end{proof}

\subsubsection{Transitive Actions}\label{transitiveactions}
An action of a group $G$ on a (nonempty) set $A$ is \emph{transitive} if $\forall a,b \in A, \exists g \in G$ such that $b = ga$.

\subsubsection{Orbit}\label{orbit}
The \emph{orbit} of $a \in A$ under an action of a group $G$ is the set\dots
$$O_G(a) := \{ ga | g \in G \}.$$

\subsubsection{Stabilizer Subgroup}\label{stabilizer}
Let $G$ act on a set $A$, and let $a \in A$. The \emph{stabilizer subgroup} of $a$ consists of the elements of $G$ which fix $a$:
$$\textrm{Stab}_G(a) := \{ g \in G | ga = a \}.$$

\subsubsection{Category G-Set}\label{gsetcategory}
The \hyperref[functorcategories]{functor category} of group actions. Thus morphisms are commutative diagrams such as\dots

\begin{figure}[H]
\centering
\begin{tikzcd}[column sep = huge]
G \times A \arrow[r, "\textrm{id}_G \times \varphi"] \arrow[d, "\rho"] & G \times A' \arrow[d, "\rho'"] \\
A \arrow[r, "\varphi"] & A'
\end{tikzcd}
\end{figure}

\noindent Intuitively, we think of these objects as sets endowed with a group action, i.e. \emph{G-sets}. Arrows are morphisms (functions) such as $\varphi$
above which preserve the group action. They are called \emph{G-equivariant}.

\begin{proposition}
\label{characterizationofactions}
Every transitive left-action of $G$ on a nonempty set $A$ is isomorphic to the left-multiplication of $G$ on $G/H$, for $H \, =$ the stabilizer
of any $a \in A$.
\end{proposition}

\begin{proof}
Let $G$ act transitively on a set $A$, let $a \in A$ be any element, and let $H = \textrm{Stab}_G(a)$. I claim that there is an equivariant bijection\dots
$$\varphi : G/H \rightarrow A$$
defined by\dots
$$gH \mapsto ga$$
for all $g \in G$.

First of all $\varphi$ is well-defined: if $g_1H =g_2H$, then $g_1^{-1}g_2 \in H$, hence $(g_1^{-1}g_2)a = a$, and it follows that $g_1a = g_2a$ as needed.
To verify that $\varphi$ is bijective, define a function $\psi : A \rightarrow G/H$ by sending an element $ga$ of $A$ to $gH$; $\psi$ is well-defined becasue if $g_1a = g_2a$,
then $g^{-1}(g_2a) = a$, so $g_1^{-1}g_2 \in H$ and $g_1H = g_2H$. It is clear that $\varphi$ and $\psi$ are inverses of each other; hence $\varphi$ is a bijection.

Equivariance is immediate: $\varphi(g'(gH)) = g'ga = g' \varphi(gH).$
\end{proof}

\begin{corollary}
\label{lagrangeforactions}
If $O$ is an orbit of the action of a finite group $G$ on a set $A$, then $O$ is a finite set and\dots
$$|O| \textrm{ divides } |G|.$$
\end{corollary}

\begin{proof}
Use Lagrange's theorem (\ref{lagrangesthm}) and the previous theorem.
\end{proof}

\begin{proposition}
Suppose a group $G$ acts on a set $A$, and let $a \in A$, $g \in G$, $b = ga$. Then\dots
$$\textrm{Stab}_G(b) = g\textrm{Stab}_G(a)g^{-1}.$$
\end{proposition}

\begin{proof}
Observe if $h \in \textrm{Stab}_G(a)$, then\dots
$$(ghg^{-1})(b) = gh(g^{-1}g)a = gha = ga = b,$$
proving $\supseteq$. For $\subseteq$ note $a = g^{-1}b$ apply the same argument.
\end{proof}

\subsubsection{Fixed Point Set}
The set of \emph{fixed points} of a group action is\dots
$$Z = \{ a \in S | (\forall g \in G) : ga = a \}$$

\begin{proposition}
Let $S$ be a finite set, and let $G$ be a group acting on $S$. Then\dots
$$|S| = |Z| + \sum_{a \in A} [G : \textrm{Stab}_G(a)]$$
where $A \subseteq S$ has exactly one element for each nontrivial orbit of the action.
\end{proposition}

\begin{proof}
The orbits form a partition of $S$, and $Z$ collects the trivial orbits; hence\dots
$$|S| = |Z| + \sum_{a \in A} |O_a|,$$
where $O_a$ denotes the orbit of $a$. By \ref{lagrangeforactions}, the order $|O_a|$
equals the index of the stabilizer of $a$, yielding the statement.
\end{proof}

\begin{corollary}
\label{fixedpointrestriction}
Let $G$ be a $p$-group acting on a finite set $S$, and let $Z$ be the fixed point set of the action. Then\dots
$$|Z| \equiv |S| \textrm{ mod } p.$$
\end{corollary}

\subsubsection{Center}\label{center}
For the action $\sigma : G \rightarrow S_G$, the \emph{center} of $G$, denoted $Z(G)$, is the subgroup ker $\sigma$ of $G$.\newline

\noindent Concretely\dots
$$Z(G) = \{ g \in G | (\forall a \in G) : ga = ag \}.$$

\begin{lemma}
Let $G$ be a finite group, and assume $G/Z(G)$ is cyclic. Then $G$ is commutative (and hence
$G/Z(G)$ is in fact trivial).
\end{lemma}

\begin{proof}
As $G/Z(G)$ is cyclic, there exists an element $g \in G$ such that the class $gZ(G)$ generates $G/Z(G)$. Then $\forall a \in G$\dots
$$aZ(G) = (gZ(G))^r$$
for some $r \in \mathbb{Z}$; that is, there is an element $z \in Z(G)$ of the center such that $a = g^rz.$

If now $a,b$ are in $G$, use this fact to write\dots
$$a = g^rz, \; b = g^sw$$
for some $s \in \mathbb{Z}$ and $w \in Z(G)$; but then\dots
$$ab = (g^rz)(g^sw) = g^{r+s}zw = (g^sw)(g^rz) = ba,$$
where I have used the fact that $z$ and $w$ commute with every element of $G$. As $a$ and $b$ were arbitrary, this proves that $G$ is commutative.
\end{proof}

\subsubsection{Conjugation Action}\label{conjugationgroupaction}
Every group $G$ acts by conjugation on the underlying set $G$. The action $\rho : G \times G \rightarrow G$ is
the operation in the group\dots
$$(\forall g,a \in G): \rho(g,h) = ghg^{-1}$$

\subsubsubsection{Centralizer and Normalizer}\label{centralizer}\label{normalizer}
The \emph{centralizer} (or \emph{normalizer}) $Z_G(a)$ for $a \in G$ is its stabilizer under conjugation. Concretely\dots
$$Z_G(a) = \{ g \in G | gag^{-1} = a \}.$$

\noindent The \emph{normalizer} $N_G(A)$ of $A$ is its stabilizer under conjugation.\newline

\noindent The \emph{centralizer} $Z_G(A) \subseteq N_G(A)$ fixing each element of $A$.

\subsubsubsection{Conjugacy Class}\label{conjugacyclass}
The \emph{conjugacy class} of $a \in G$ is the orbit $[a]$ of $a$ under the conjugation action.\newline

\noindent Two elements $a,b$ of $G$ are conjugate if they belong to the same conjugacy class.

\begin{proposition}[Class Formula]
\label{classformula}
Let $G$ be a finite group. Then\dots
$$|G| = |Z(G)| + \sum_{a \in A}[G : Z(a)],$$
where $A \subseteq G$ is a set containing one representative for each nontrivial conjugacy class in $G$.
\end{proposition}

\begin{lemma}
Let $H \subseteq G$ be a subgroup. Then (if finite) the number of subgroups conjugate to $H$ equals the index
$[G : N_G(H)]$ of the normalizer of $H$ in $G$.
\end{lemma}

\begin{corollary}
If $[G:H]$ is finite, then the number of subgroups conjugate to $H$ is finite and divides $[G:H]$.
\end{corollary}

\begin{proof}
$$[G : H] = [G : N_G(H)] \cdot [N_G(H) : H]$$
\end{proof}