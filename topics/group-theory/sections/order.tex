\subsection{Order}

\subsubsection{Order of an element}\label{elementorder}
The \emph{order of an element} $g \in G$, denoted $|g|$, is the smallers positive integer $n$ such that $g^n = e$.

\noindent $g$ has \emph{finite order} if any such integer exists.\newline

\noindent $g$ has \emph{infinite order} if no such integer exists, denoted $|g| = \infty$.\newline

\begin{lemma}
\label{orderdividesothernums}
If $g^n = e$ for some positive integer $n$, then $|g|$ is a divisor of $n$.
\end{lemma}

\begin{proof}
As observed, $n \geq |g|$ for $n \in \mathbb{Z}$, that is $n - |g| \geq 0$. Since $\mathbb{Z}$ is a Euclidean domain, there must exist
an integer $m > 0$ such that\dots
$$r = n -|g|\cdot m \geq 0 \; \textrm{ and } \; n - |g| \cdot (m + 1) < 0,$$
that is, $r < |g|$. Note that\dots
$$g^r = g^{n - |g| \cdot m} = g^n \cdot (g^{|g|})^{-m} = e \cdot e^{-m} = e.$$
By definition of order, $|g|$ is the smallest positive integer such that $g^{|g|}=e.$ Since $r$ is smaller than $|g|$ and $g^r = e,$ $r$ cannot
be positive; hence $r = 0$ necessarily. So $n = |g| \cdot m$.
\end{proof}

\begin{corollary}
Let $g$ be an element of finite order, and let $N \in \mathbb{Z}$. Then
$$g^N = e \Leftrightarrow N \textrm{ is a multiple of } |g|$$
\end{corollary}

\subsubsection{Order of a group}\label{grouporder}
If $G$ is finite as a set, its \emph{order} $|G|$ is the number of its elements; we write $|G| = \infty$ if $G$ is infinite.

\begin{proposition}
\label{orderofmultipleofelement}
Let $g \in G$ be an element of finite order. Then $g^m$ has finite order $\forall m \geq 0$, and in fact
$$|g^m| = \frac{\textrm{lcm}(m,|g|)}{m} = \frac{|g|}{\textrm{gcd}(m,|g|)}.$$
\end{proposition}

\begin{proof}
The order of $g^m$ is the least positive $d$ for which\dots
$$g^{md} = e.$$
In other words, $m|g^m|$ is the smallest multiple of $m$ which is also a multiple of $|g|$:
$$m|g^m| = \textrm{lcm}(m,|g|).$$
\end{proof}

\begin{proposition}
If $gh$ = $hg$, then $|gh|$ divides lcm$(|g|,|h|)$.
\end{proposition}

\begin{proof}
Observe\dots
$$(gh)^{\textrm{lcm}(m,n)} = (gh)(gh)\cdots(gh) = gg\cdots g \LargerCdot hh\cdots h = g^{\textrm{lcm}(m,n)}h^{\textrm{lcm}(m,n)} = e.$$
\end{proof}

\subsubsection{Index of a subgroup}\label{subgroupindex}
The \emph{index} of $H$ in $G$, denoted $[G:H]$, is the number of elements $|G/H|$ of $G/H$, when this is finite, and $\infty$ otherwise.

\begin{lemma}
\label{indexofcoset}
Let $H$ be a subgroup of a group $G$. Then $\forall g \in G$ the functions
$$H \rightarrow gH, \; h \mapsto gh,$$
$$H \rightarrow Hg, \; h \mapsto hg$$
are bijections.
\end{lemma}

\begin{proof}
Surjectiveness is clear and cancellation implies that they are injective.
\end{proof}

\subsubsubsection{Lagrange's Theorem}

\begin{corollary}
\label{lagrangesthm}
If $G$ is a finite group and $H \subseteq G$ is a subgroup, then $|G| = [G : H] \cdot |H|$. In particular, $|H|$ is a divisor of $|G|$.
\end{corollary}

\begin{proof}
Indeed, $G$ is the disjoint union of $|G/H|$ distinct cosets $gH$, and $|gH| = |H|$ by \ref{indexofcoset}.
\end{proof}

\begin{corollary}
\label{elementorderdividesgrouporder}
If $g \in G$, then $a \cdot |g| = |G|$ for some positive integer $a$.
\end{corollary}