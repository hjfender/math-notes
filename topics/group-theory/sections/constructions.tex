\subsection{Group Constructions}\label{groupconstructions}

\subsubsection{Product of Groups}\label{groupproduct}
Let $G$ and $H$ be two groups. Define $G \times H := \{ (g,h) | g \in G, h \in H \}$ with the operation
$(g_1,h_1) \LargerCdot_{G \times H} (g_2,h_2) = (g_1 \LargerCdot_G g_2,h_1 \LargerCdot_H h_2)$. Then $G \times H$
is the product group of the groups $G$ and $H$.

\subsubsection{Free Product of Groups}\label{freegroupproduct}

\subsubsection{Free Groups}\label{freegroup}
$F(A)$ is a free group on a set $A$ if there is a set-function $j : A \rightarrow F(A)$ such that, for all groups $G$
and set-functions $f : A \rightarrow G$, there exists a unique group homomorphism $\varphi : F(A) \rightarrow G$ such that
the following diagram commutes.

\begin{figure}[H]
\centering
\begin{tikzcd}
F(A) \arrow[r, "\varphi"] & G\\
A \arrow[u,"j"] \arrow[ur, "f" below]
\end{tikzcd}
\end{figure}

\subsubsubsection{Concrete construction}

Consider the set $A$ as an 'alphabet' and construct 'words' whose letters are elements of $A$ or 'inverses' of elements of $A$.
That is, a \emph{word} on $A$ is an ordered list
$$(a_1,a_2,\dots,a_n)$$,
which we denote by the juxtaposition
$$w = a_1a_2\dots a_n,$$
where each letter is either an element of $A$ or an inverse of an element in $A$. Denote the set of words on $A$ as $W(A)$.

Define an 'elementary' reduction $r : W(A) \rightarrow W(A)$: given $w \in W(A)$, search for the first occurrence (from left
to right) of a pair $aa^{-1}$ or $a^{-1}a$, and let $r(w)$ be the word obtained by removing such a pair.

Note that $r(w) = w$ precisely when 'no cancellation is possible'; We say that $w$ is a 'reduced word' in this case.

\begin{lemma}
If $w \in W(A)$ has length $n$, then $r^{\lfloor \frac{n}{2} \rfloor}(w)$ is a reduced word.
\end{lemma}

\begin{proof}
Indeed, either $r(w) = w$ or the length of $r(w)$ is less than the length of $w$; but one cannot decrease the length of $w$ more
than $n/2$ times, since each non-identity application of $r$ decreases the length by two.
\end{proof}

Now define the 'reduction' $R : W(A) \rightarrow W(A)$ by setting $R(w) = r^{\lfloor \frac{n}{2} \rfloor}(w)$, where $n$ is the length of $w$.
By the lemma, $R(w)$ is always a reduced word.

Let $F(A)$ be the set of reduced words on $A$, that is, the image of the reduction map $R$ we have just defined.

Define a binary operation on $F(A)$ by juxtaposition and reduction: $w \LargerCdot w' = R(ww')$. $F(A)$ is a group under this operation.

\begin{proposition}
The pair $(j, F(A))$ satisfies the universal property for free groups on $A$.
\end{proposition}

\subsubsection{Quotient Group}\label{groupquotients}

\subsubsubsection{Quotient Group by $\sim$}\label{quotientgroupbyrelation}
\begin{proposition}
The operation\dots
$$[a] \LargerCdot [b] := [ab]$$
defines a group structure on $G/\sim$ if and only if $\forall a, a', g \in G$
$$ a \sim a' \Rightarrow ga \sim ga' \textrm{ and } ag \sim a'g.$$
In this case the quotient function $\pi : G \rightarrow G/\sim$ is a homomorphism and is universal with respect to homomorphisms
$\varphi : G \rightarrow G'$ such that $a \sim a' \Rightarrow \varphi(a) = \varphi(a').$
\end{proposition}

\subsubsubsection{Cosets}\label{cosets}

\begin{proposition}
Let $\sim$ be an equivalence relation on a group $G$, satisfying $(\forall g \in G) : \; a \sim b \Rightarrow ga \sim gb$. Then\dots
\begin{itemize}
  \item the equivalence class of $e_G$ is a subgroup of $H$ of $G$; and
  \item $a \sim b \Leftrightarrow a^{-1}b \in H \Leftrightarrow aH = bH.$
\end{itemize}
\end{proposition}

\begin{proof}
Let $H \subseteq G$ be the equivalence class of the identity; $H \neq \emptyset$ as $e_G \in H$. For $a,b \in H$,
we have $e_G \sim b$ and hence $b^{-1} \sim e_G$; hence $ab^{-1} \sim a$; and hence\dots
$$ab^{-1} \sim a \sim e_G$$
by the transitivity of $\sim$ and since $a \in H$. This shows $ab^{-1} \in H$ for all $a,b \in H$, proving that $H$ is a subgroup.

Next, assume $a,b \in G$ and $a \sim b$. Multiplying on the left by $a^{-1}$, implies $e_G \sim a^{-1}b$, that is, $a^{-1}b \in H$. Since $H$ is closed
under the operation, this implies $a^{-1}bH \subseteq H$, hence $bH \subseteq aH$; as $\sim$ is symmetric, the same reasoning gives $aH \subseteq bH$; and
hence $aH = bH$. Thus, we have proved\dots
$$a \sim b \Rightarrow a^{-1}b \in H \Rightarrow aH = bH.$$
Finally, assume $aH = bH$. Then $a = ae_G \in bH$, and hence $a^{-1}b \in H$. By definition of $H$, this means $e_G \sim a^{-1}b$. Multiplying on the left by $a$
shows that $a \sim b$.
\end{proof}

\noindent The \emph{left-cosets} of a subgroup $H$ in a group $G$ are the sets $aH$, for $a \in G$. The \emph{right-cosets}
of $H$ are the sets $Ha$, $a \in G$.

\begin{proposition}
If $H$ is any subgroup of a group $G$, the relation $\sim_L$ defined by
$$(\forall a,b \in G) : \; a \sim_L b \Leftrightarrow a^{-1}b \in H$$
is an equivalence relation satisfying $(\forall g \in G) : \; a \sim b \Rightarrow ga \sim gb$.
\end{proposition}

\noindent Taking the previous two propositions together we get\dots

\begin{proposition}
There is a bijection between the set of subgroups of $G$ and equivalence relations on $G$ satisfying $(\forall g \in G) : \; a \sim b \Rightarrow ga \sim gb$;
for the relation $\sim_L$ corresponding to a subgroup $H$, $G/\sim_L$ may be described as the set of left-cosets $aH$ of $H$.
\end{proposition}

\noindent Similar statements exist for right cosets and the property $(\forall g \in G) : \; a \sim b \Rightarrow ag \sim bg$ leading to\dots
\begin{proposition}
There is a bijection between the set of subgroups of $G$ and equivalence relations on $G$ satisfying $(\forall g \in G) : \; a \sim b \Rightarrow ag \sim bg$;
for the relation $\sim_R$ corresponding to a subgroup $H$, $G/\sim_R$ may be described as the set of left-cosets $Ha$ of $H$.
\end{proposition}

\begin{proposition}
The relations $\sim_L$, $\sim_R$ corresponding to subgroups of $H$ coincide if and only if $H$ is normal.
\end{proposition}

\subsubsubsection{Definition}\label{definitionofquotientgroup}

Let $H$ be a normal subgroup of $G$. The \emph{quotient group of G modulo H}, denoted $G/H$, is the group $G/\sim$
obtained from the relation $\sim$ as defined in the previous propositions. In terms of cosets, the product in $G/H$
is defined by
$$(aH)(bH) := (ab)H.$$
The identity element is $H$.

\subsubsubsection{Universal Property}\label{universalpropertyofquotientgroups}
\begin{theorem}
Let $H$ be a normal subgroup of a group $G$. Then for every group homomorphism $\varphi : G \rightarrow G'$ such that
$H \subseteq \textrm{ker} \varphi$ there exists a unique group homomorphism $\tilde \varphi: G/H \rightarrow G'$ so that
the diagram

\begin{figure}[H]
\centering
\begin{tikzcd}
G \arrow[rr, "\varphi" above] \arrow[rd, "\pi" below] & & G'\\
 & G/H \arrow[ur, "\exists ! \tilde \varphi" below] &
\end{tikzcd}
\end{figure}

\noindent commutes.
\end{theorem}