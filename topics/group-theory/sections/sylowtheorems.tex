\subsection{Sylow Theorems}\label{sylowtheorems}

\subsubsection{$p$-Sylow subgroups}\label{sylowsubgroup}
A $p$-\emph{Sylow subgroup} of a finite group $G$ is a subgroup of order $p^r$, where $|G| = p^r m$
and gcd$(p,m) = 1$. That is, $P \subseteq G$ is a $p$-Sylow subgroup if it is a $p$-group and $p$ does
not divide $[G:P]$.

\subsubsection{Sylow I}\label{sylow1}

\begin{theorem}[First Sylow Theorem]
\label{sylowthm1}
Every finite group contains a $p$-Sylow subgroup, for all primes $p$.
\end{theorem}

\noindent Sylow I follows from the following\dots

\begin{proposition}
If $p^k$ divides the order of $G$, then $G$ has a subgroup of order $p^k$.
\end{proposition}

\begin{proof}
If $k = 0$, there is nothing to prove, so we may assume $k \geq 1$ and in particular that $|G|$ is a multiple of $p$.

Argue by induction on $|G|$: if $|G| = p$, again there is nothing to prove; if $|G| > p$ and $G$ contains a proper subgroup $H$
such that $[G : H]$ is relatively prime to $p$, then $p^k$ divides the order of $H$, and hence $H$ contains a subgroup of order $p^k$
by induction hypothesis, and thus so does $G$.

Therefore, we may assume that all proper subgroups of $G$ have index divisible by $p$. By the class formula, $p$ divides the order 
of the center $Z(G)$. By Cauchy's theorem, $\exists a \in Z(G)$ such that $a$ has order $p$. The cyclic subgroup $N = \langle a \rangle$
is contained in $Z(G)$, and hence it is normal in $G$. Now consider the quotient $G/N$.

Since $|G/N| = |G|/p$ and $p^k$ divides $|G|$ by hypothesis, we have that $p^{k-1}$ divides the order of $G/N$. By the induction hypothesis,
we may conclude that $G/N$ contains a subgroup of order $p^{k-1}$. By the structure of the subgroups of a quotient, this subgroup must
be of the form $P/N$, for $P$ a subgroup of $G$.

But then $|P| = |P/N| \cdot |N| = p^{k-1} \cdot p = p^k$, as needed.
\end{proof}

\subsubsection{Sylow II}\label{sylow2}

\begin{theorem}[Second Sylow Theorem]
\label{sylowthm2}
Let $G$ be a finite group, let $P$ be a $p$-Sylow subgroup, and let $H \subseteq G$ be a $p$-group. Then $H$ is contained
in a conjugate of $P$: there exists $g \in G$ such that $H \subseteq gPg^{-1}$.
\end{theorem}

\begin{proof}
Act with $H$ on the set of left-cosets of $P$, by left-multiplication. Since there are $[G : P]$ cosets and $p$ does not divide
$[G : P]$, we know this action must have fixed points: let $gP$ be one of them. This means that $\forall h \in H$:
$$hgP = gP;$$
that is, $g^{-1}hgP = P$ for all $h$ in $H$; that is, $g^{-1}Hg \subseteq P$; that is, $H \subseteq gPg^{-1}$, as needed.
\end{proof}

\begin{lemma}
\label{indexinnormalizer}
Let $H$ be a $p$-group contained in a finite group $G$. Then\dots
$$[N_g(H) : H] \equiv [G : H] \textrm{ mod } p.$$
\end{lemma}

\begin{proof}
I $H$ is trivial, then $N_G(H) = G$ and the two numbers are equal.

Assume then that $H$ is nontrivial, and act with $H$ on the set of left-cosets of $H$ in $G$, by left-multiplication. The fixed points
of this action are the cosets $gH$ such that $\forall h \in H$\dots
$$hgH = gH,$$
that is, such that $g^{-1}hg \in H$ for all $h \in H$; in other words, $H \subseteq gHg^{-1}$, and hence $gHg^{-1} = H$. This means precisely that
$g \in N_G(H)$. Therefore, the set of fixed points of the action consists of the set of cosets of $H$ in $N_G(H)$.

The statement then follows immediately from \ref{fixedpointrestriction}.
\end{proof}

\begin{proposition}
Let $H$ be a $p$-subgroup of a finite group $G$, and assume that $H$ is not a $p$-Sylow subgroup. Then there exists a $p$-subgroup
$H'$ of $G$ containing $H$, such that $[H' : H] = p$ and $H$ is normal in $H'$.
\end{proposition}

\begin{proof}
Since $H$ is not a $p$-Sylow subgroup of $G$, $p$ divides $[N_G(H) : H]$, by the previous lemma. Since $H$ is normal in $N_G(H)$, we may consider
the quotient group $N_G(H)/H$, and $p$ divides the order of this group. By \ref{cauchysthm}, $N_G(H)/H$ has an element of order $p$; this generates
a subgroup of order $p$ of $N_G(H)/H$, which must be of the form $H' / H$ for a subgroup $H'$ of $N_G(H)$.

It is straightforward to verify that $H'$ satisfies the stated requirements.
\end{proof}

\subsubsection{Sylow III}\label{sylow3}

\begin{theorem}[Third Sylow Theorem]
\label{sylowthm3}
Let $p$ be a prime integer, and let $G$ be a finite group of order $|G| = p^rm$. Assume that $p$ does not divide $m$. Then the number of $p$-Sylow subgroups $N_p$ satisfies\dots
\begin{itemize}
  \item $N_p | m$;
  \item $N_p \equiv 1 \textrm{ mod } p$.
\end{itemize}
\end{theorem}

\begin{proof}
Let $N_p$ denote the number of $p$-Sylow subgroups of $G$.

By \ref{sylowthm2}, the $p$-Sylow subgroups of $G$ are the conjugates of any given $p$-Sylow subgroup $P$. By \ref{numofconjugatesubgroups}, $N_p$ is the index of the normalizer $N_G(P)$
of $P$; thus by \ref{cornumofconjugatesubgroups} it divides the index $m$ of $P$. In fact,
$$m = [G:P] = [G:N_G(P)] \cdot [N_G(P) : P] = N_p \cdot [N_G(P) : P].$$

Now, by \ref{indexinnormalizer} we have\dots
$$m = [G : P] \equiv [N_G(P) : P] \textrm{ mod } p;$$
multiplying by $N_p$, we get\dots
$$mN_p \equiv m \textrm{ mod } p.$$
Since $m \not \equiv 0 \textrm{ mod } p$ and $p$ is prime, this implies\dots
$$N_p \equiv 1 \textrm{ mod } p,$$
as needed.
\end{proof}