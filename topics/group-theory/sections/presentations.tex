\subsection{Presentations}\label{presentations}

A \emph{presentation} of a group $G$ is an explicit isomorphism\dots
$$G \cong \frac{F(A)}{R}$$
where $A$ is a set and $R$ is a subgroup of 'relations.' In other wordsd, a presentation is an explicit surjection\dots
$$\varphi : F(A) \twoheadrightarrow G$$
of which $R$ is the kernel. \newline

\noindent To create a presentation it is enough to list 'enough' relations, i.e create a set $\mathcal{R}$ of words, and then let
$R$ be the smallest normal subgroup of $F(A)$ containing $\mathcal{R}$. We can then denote a presentation by $\langle A | \mathcal{R} \rangle$.

\subsubsection{Finitely Presented}\label{finitelypresented}
A group is \emph{finitely presented} if it admits a presentation $\langle A | \mathcal{R} \rangle$ in which both $A$ and $\mathcal{R}$ are finite.