\subsection{Constructions}\label{topologicalconstructions}

\subsubsection{Subspace Topology}\label{subspacetopology}
Let $X$ be a topological space with topology $\mathcal{T}$ and $A$, a subset of $X$. The \emph{subspace topology} on $A$ is given by $\mathcal{T}_A = \{ U \cap A | U \in \mathcal{T} \}$.

\begin{proposition}
The collection $\mathcal{T}_A$ is a topology on $A$ and with this topology the inclusion $\iota : A \rightarrow X$ is continuous.
\end{proposition}

\begin{proposition}
Suppose $X = A \cup B$ is a space, $A$, $B$ are open subsets of $X$, and $f: A \rightarrow Y$, $g : B \rightarrow Y$ are continuous functions (where $A$ and $B$ have the subspace topologies).
If $f(x) = g(x)$ for all $x \in A \cap B$,
then $F = f \cup g : X \rightarrow Y$ is a continuous function, where $F$ is defined by\dots
\[
F(x) = \begin{cases}
f(x), & \textrm{ if } x \in A,\\
g(x), & \textrm{ if } x \in B.
\end{cases}
\]
\end{proposition}

\subsubsection{Product Topology}\label{producttopology}
Given topological spaces $X$ and $Y$, the set $X \times Y$ is a topological space under the topology generated by the basis\dots
$$\mathcal{B} = \{ U \times V | U \textrm{ open in } X,V \textrm{ open in } Y \}.$$

\begin{proposition}
Given three topological spaces $X$, $Y$, and $Z$, and a function $f : Z \rightarrow X \times Y$, then $f$ is continuous if and only if
$\pi_1 \circ f : Z \rightarrow X$ and $\pi_2 \circ f : Z \rightarrow Y$ are continuous.
\end{proposition}

\begin{proposition}
If $X$ and $Y$ are separable spaces, so is $X \times Y$.
\end{proposition}

\begin{proposition}
If $X$ and $Y$ are connected spaces, then $X \times Y$ is connected.
\end{proposition}

\begin{proposition}
If $X$ and $Y$ are path-connected, then so is $X \times Y$.
\end{proposition}

\begin{proposition}
If $X$ and $Y$ are compact spaces, then $X \times Y$ is compact.
\end{proposition}

\begin{corollary}
If $K \subseteq \mathbb{R}^n$, then $K$ is compact if and only if $K$ is closed and bounded.
\end{corollary}

\subsubsubsection{Infinite Product Topology}\label{infiniteproducttopology}
For a family of sets $\{X_{\alpha}\}_{\alpha \in J}$, the set $\prod_{\alpha \in J} X_{\alpha}$ is a topological space under the following two topologies\dots
\begin{itemize}
  \item $\mathcal{T}_{box}$, the topology generated by the basis:
  $$\mathcal{B} = \{ \prod_{\alpha \in J} U_{\alpha} | U_{\alpha} \subset X_{\alpha} \textrm{ for all } \alpha, \textrm{ each } U_{\alpha} \textrm{ open in } X_{\alpha} \}.$$
  \item $\mathcal{T}_{prod}$, the topology generated by the basis:
  $$\mathcal{B} = \{ S_1 \cap S_2 \cap \cdots \cap S_n | n \geq 1, S_i \in \mathcal{S}\},$$
  where $\mathcal{S}$ is the subbasis of subseets $S \ \prod_{\alpha \in J} V_{\alpha}$, where for each $\beta \in J$, $V_{\beta}$ is open in $X_{\beta}$ and $V_{\gamma} = X_{\gamma}$
  for all but finitely many $\gamma \in J$.
\end{itemize}

\begin{proposition}
Let $X$ be a space and for all $\alpha \in J$, let $X_{\alpha} = X$. Define the function\dots
$$\Delta : X \rightarrow \prod_{\alpha \in J} X_{\alpha}$$
by $\Delta(x) : \alpha \mapsto x \in X_{\alpha} = X$. This function is continuous when $\prod_{\alpha \in J} X_{\alpha}$ has the product topology.
\end{proposition}

\noindent This proposition highlights the difference between $\mathcal{T}_{box}$ and $\mathcal{T}_{prod}$. Consider $\Delta : (\mathbb{R}, \textrm{usual}) \rightarrow (\mathbb{R}^{\omega}, \mathcal{T}_{box}).$
Let\dots
$$W = (-1,1) \times (-1/2,1/2) \times (-1/3, 1/3) \times \cdots$$
be an open set in $\mathcal{T}_{box}$. Then $\Delta^{-1}(W) = \{ 0 \}$, which is not open.

\subsubsection{Quotient Topology}\label{quotienttopology}
A subset $V \subset [X]$ is open in the \emph{quotient topology} on $[X]$ if $\pi^{-1}(V)$ is open in $X$. The space $[X]$ with this topology is called a \emph{quotient space} of $X$.

\subsubsubsection{Quotient Map}\label{quotientmap}
An surjective map $f : X \rightarrow Y$ is called a \emph{quotient map} when $V$ is open in $Y$ if and only if $f^{-1}(V)$ is open in $X$.

\begin{theorem}
Let $\sim$ be an equivalence relation in a space $X$ that is Hausdorff. Then $[X]$ is Hausdorff if and only if the graph of $\sim$,
$\{ (x,y) | x \sim y, x, y \in X\}$, is closed in $X \times X$.
\end{theorem}