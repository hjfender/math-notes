\subsection{Metric Spaces}\label{metricspace}

A \emph{metric space} $\langle X,d \rangle$ is a set $X$ together with a \emph{metric}\label{metric} $d : X \times X \rightarrow \mathbb{R}$ satisfying\dots
\begin{enumerate}
  \item $d(x,y) \geq 0$ for all $x,y \in X$ and $d(x,y) = 0$ if and only if $x = y$.
  \item $d(x,y) = d(y,x)$ for all $x,y \in X$.
  \item (\emph{The Triangle Inequality}\label{triangleinequality}): $d(x,y) + d(y,z) \geq d(x,z)$ for all $x,y,z \in X.$
\end{enumerate}

\subsubsection{Open Ball}\label{metricopenball}
The \emph{open ball} of radius $\varepsilon > 0$ centered at a point $x$ in a metric space $\langle X,d \rangle$ is given by\dots
$$B_{\varepsilon}(x) = \{ y \in X | d(x,y) < \varepsilon\}.$$

\subsubsection{Continuity}\label{metriccontinuity}
Suppose $\langle X,d_X \rangle$ and $\langle Y,d_Y \rangle$ are two metric spaces and $f: X \rightarrow Y$ is a function. Then $f$ is \emph{continuous at} $x\in X$ if for any $\epsilon > 0$ , there is a $\delta > 0$
so that $B_{\delta}(x) \subset f^{-1}(B_{\varepsilon}(f(x)))$.\newline

\noindent The function $f$ is \emph{continuous} if it is continous at $x$ for all $x \in X$.

\subsubsection{Open Set}\label{metricopenset}
The \emph{open set} $U$ of a metric space $(X,d)$ is \emph{open} if for any $u \in U$, there is $\varepsilon > 0$ so that $B_{\varepsilon}(x) \subseteq U$.

\begin{theorem}
A function $f:X \rightarrow Y$ between metric spaces $\langle X,d \rangle$ and $\langle Y,d \rangle$ is continuous if and only if for any open subset $V$ of $Y$, the subset $f^{-1}(V)$ is open in $X$.
\end{theorem}

\subsubsection{Lebesgue's Lemma}

\subsubsubsection{Diameter}\label{diameter}
The \emph{diameter} of a subset $A$ of a metric space $X$ is defined by diam $A = \sup\{d(x,y) | x,y \in A \}.$

\begin{lemma}[Lebesgue's Lemma]
\label{lebesguelemma}
Let $X$ be a compact metric space and $\{ U_i | i \in J \}$ an open cover. Then there is a real number
$\delta > 0$ (\emph{the Lebesgue number}\label{lebesguenumber}) such that any subset of $X$ of diameter less than $\delta$
is contained in some $U_i$.
\end{lemma}

\begin{proof}
Define the continuous function $d(-,A): X \rightarrow \mathbb{R}$ by $d(x, A) = \inf\{d(x,a) | a \in A \}$. In addition,
if $A$ is closed, then $d(x,A) > 0$ for $x \not \in A$. Fiven an open cover $\{ U_i | i \in J \}$ of the compact space $X$,
there is a finite subcover $\{U_{i_1},\dots, U_{i_n} \}$. Define $\varphi_j(x) = d(x,X \setminus U_{i_j})$ for $j = 1,2,\dots, n$
and let $\varphi(x) = \max\{\varphi_1(x), \dots, \varphi_n(x) \}$. Since each $x \in X$ lies in some $U_{i_j}$, $\varphi(x) \geq \varphi_j(x) > 0$.
Furthermore, $\varphi$ is continuous so $\varphi(X) \subseteq \mathbb{R}$ is compact, and $0 \not \in \varphi(X)$. Let $\delta = \min \{ \varphi(x) | x \in X \} > 0$.
FOr any $x \in X$, consider $B(x, \delta) \subseteq X$. We know $\varphi(x) = \varphi_j(x)$ for some $j$. For that $j$, $d(x,X \setminus U_{i_j}) \geq \delta$, which implies
$B(x, \delta) \subseteq U_{i_j}$.
\end{proof}

\subsubsection{Examples}\label{metricexamples}

\subsubsubsection{Euclidean Metric Space}\label{euclideanmetric}

If for $x,y \in \mathbb{R}^n$\dots
$$d(x,y) = ||x-y|| = \sqrt{(x_1 - y_1)^2 + \cdots + (x_n - y_n)^n},$$
then $\langle \mathbb{R}^n,d \rangle$ is a metric space.

\subsubsubsection{Box Metric Space}\label{euclideanmetric}

If for $x,y \in \mathbb{R}^n$\dots
$$d(x,y) = \max\{|x_1 - y_1|, \dots, |x_n - y_n|\},$$
then $\langle \mathbb{R}^n,d \rangle$ is a metric space.\newline

\noindent The set of \hyperref[metricopenball]{open balls} for the previous two metrics form \hyperref[basis]{bases} that generate the same topology.

\subsubsubsection{Bounded Real Functions Metric Space}\label{bddmetric}

Let $\textrm{Bdd}([0,1],\mathbb{R})$ denote the set of \emph{bounded functions} $f : [0,1] \rightarrow \mathbb{R}$. If for $f,g \in \textrm{Bdd}([0,1],\mathbb{R})$\dots
$$d(f,g) = \textrm{lub}_{t \in [0,1]}\{f(t) - g(t)\},$$
then $\langle \textrm{Bdd}([0,1],\mathbb{R}),d \rangle$ is a metric space.

\subsubsubsection{Discrete Metric space}\label{discretemetric}

Let $X$ be any set and define\dots
\[
	d(x,y) = \begin{cases}
				0, & \textrm{if } x = y,\\
				1, & \textrm{if } x \neq y.
			 \end{cases}
\]
Then $\langle X,d \rangle$ is a metric space.