\subsection{Basic Methods}\label{basiccombinatorics}

\subsubsection{Addition}\label{addition}

\begin{theorem}[Addition principle]
If $A$ and $B$ are two disjoint finite sets, then\dots
$$|A \cup B| = |A| + |B|.$$
\end{theorem}

\begin{theorem}[Generalized addition principle]
Let $A_1,A_2,\dots,A_n$ be finite sets that are pairwise disjoint. Then\dots
$$|\bigcup_{i=1}^nA_i| = \sum_{i=1}^n |A_i|$$
\end{theorem}

\subsubsection{Subtraction}\label{subtraction}

\begin{theorem}[Subtraction principle]
Let $A$ be a finite set, and let $B \subseteq A$. Then $|A \setminus B|=|A|-|B|$.
\end{theorem}

\begin{proof}
Observe $|A \setminus B|+|B| = |A|$ by the addition principle.
\end{proof}

\subsubsection{Multiplication}\label{multiplication}

\begin{theorem}[Product principle] 
Let $X$ and $Y$ be two finite sets. Then $|X \times Y| = |X| \times |Y|$.
\end{theorem}

\begin{theorem}[Generalized product principle]
Let $X_1,X_2,\dots,X_n$ be finite sets. Then $|\bigtimes_{i \in I}^{n} X_i| = \prod_{i \in I}^{n} |X_i|$.
\end{theorem}

\subsubsection{Division}\label{division}

\begin{theorem}
Let $S$ and $T$ be finite sets so that a $d$-to-one function $f : T \rightarrow S$ exists. Then
$$|S| = \frac{|T|}{d}.$$
\end{theorem}

\subsubsection{Binomial Coefficients}\label{binomials}

See \hyperref[permutation]{permutations}.

\begin{theorem}
Let $n$ be a positive integer, and let $k \leq n$ be a nonnegative integer. Then the number of all $k$-element subsets of $[n]$ is
$${n \choose k} = \frac{(n)_k}{k!} = \frac{n!}{k!(n-k)!}.$$
\end{theorem}

Note: ${n \choose k} = {n \choose n-k}$ exhibits \hyperref[duality]{duality}.

\begin{theorem}[Binomial theorem]
If $n$ is a positive integer, then\dots
$$(x+y)^n = \sum_{k=0}^n {n \choose k} x^k y^{n-k}.$$ 
\end{theorem}

\begin{proof}
The left-hand side of the equation contains the factor $(x + y)$ $n$ times. To compute the product we choose an $x$ or $y$ term from each factor and multiply those $n$ terms together, then do this
in all $2^n$ possible ways, adding all the resulting products. It suffices to show that there are exactly ${n \choose k}$ products of the form $x^k y^{n-k}$, which is immediately obvious from the
way we compute the product.
\end{proof}

\begin{theorem}
Let $n$ and $k$ be nonnegative integers so that $k < n.$ Then\dots
$${n \choose k} + {n \choose k + 1} = {n + 1 \choose k + 1}$$
\end{theorem}

\begin{theorem}
For all positive integers $n$,
$${2n \choose n} = \sum^n_{k=0}{n \choose k}^2.$$
\end{theorem}

\subsubsection{Pigeonhole Principle}\label{pigeonholeprinciplecombinatorics}

\begin{theorem}[Pigeonhole Principle]
Let $A_1, A_2, \dots, A_k$ be finite sets that are pairwise disjoint. Let us assume that
$$|A_1 \cup A_2 \cup \dots \cup A_k| > kr.$$
Then there exists at least one index i so that $|A_i| > r$. (See: \hyperref[pigeonholeprinciple]{Pigeonhole Priciple in Set Theory})
\end{theorem}

\begin{example}
Consider the sequence $1,3,7,15,31,\dots$, in other words, the sequence whose $i$th element is $a_i = 2^i-1.$ Let $q$ be any odd integer.
Then our sequence contains an element that is divisible by $q$.
\end{example}

\begin{proof}
Consider the first $q$ elements of our sequence. If one of them is divisible by $q,$ then we are done. If not, then consider their remainders modulo $q$. That is, let us write\dots
$$a_i = d_i q + r_i$$
where $0 < r_i < q$, and $d_i = \lfloor a_i / q \rfloor $. As the integers $r_1,r_2,\dots,r_q$ all come from the open interval $(0,q)$, there are $q-1$ possibilities for their values. 
On the other hand, their number is $q$, so, by the pigeonhole principle, there have to be two of them that are equal. Say these are $r_n$ and $r_m$, with $n > m$. Then $a_n = d_n q + r_n$
and $a_m = d_m q + r_n$, so\dots
$$a_n - a_m = (d_n - d_m)q$$
or, after rearranging,
\begin{align*}
  (d_n - d_m)q &= a_n - a_m \\
  			   &= (2^n - 1) - (2^m - 1) \\
 			   &= 2^m(2^{n-m} - 1)\\
 			   &= 2^m a_{n-m}
\end{align*}
As the first expression of our chain of equations is divisible by $q$, so too must be the last expression. Note that $2^{n-m}$ is relatively prime to any odd number $q$, that is, the largest
common divisor of $2^{n-m}$ and $q$ is $1$. Therefore, the equality $(d_n - d_m)q = 2^{n-m}a_{n-m}$ implies that $a_{n-m}$ is divisible by $q.$
\end{proof}