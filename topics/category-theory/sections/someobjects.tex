\subsection{Some Objects in Categories}\label{someobjects}

\subsubsection{Initial Objects}\label{initial}

We say that an object $i$ of a category $C$ is \emph{initial} in $C$ if for every object $a$ of $C$ there exists exactly one morphism $i \rightarrow a$ in $C$:

$$\forall a \in Obj(C): \; \; Hom_C(i,a) \textrm{ is a singleton}.$$

\subsubsection{Final Objects}\label{final}

We say that an object $f$ of a category $C$ is \emph{final} in $C$ if for every object $a$ of $C$ there exists exactly one morphism $a \rightarrow f$ in $C$:

$$\forall a \in Obj(C): \; \; Hom_C(a,f) \textrm{ is a singleton}.$$

\begin{proposition}
Let $C$ be a category.
\begin{itemize}
  \item If $i_1$, $i_2$ are both initial objects in $C$, then $i_1 \cong i_2$.
  \item If $f_1$, $f_2$ are both initial objects in $C$, then $f_1 \cong f_2$.
\end{itemize}
\end{proposition}

\subsubsection{Null Objects}\label{null}

An object that is both initial and terminal.

\subsubsection{Group Objects}\label{groupobjects}

A \emph{group object} in $C$ consists of an object $g$ of $C$ and of morphisms\dots
$$m : g \times g \rightarrow g, \; e : 1 \rightarrow g, \; \iota : g \rightarrow g$$
in $C$ such that the diagrams\dots

\begin{figure}[H]
\centering

\begin{tikzcd}[column sep=huge, row sep=huge]
	(g \times g) \times g \arrow[r, "m \times \textrm{id}_g"] \arrow[d] & g \times g \arrow[r, "m"] & g \arrow[d, "="]\\
	g \times (g \times g) \arrow[r, "\textrm{id}_g \times m"] & g \times g \arrow[r, "m"] & g
\end{tikzcd}

\end{figure}

\begin{figure}[H]
\centering

\[ \begin{array}{cc}
\begin{tikzcd}[column sep=huge, row sep=huge]
	1 \times g \arrow[r, "e \times \textrm{id}_g"] \arrow[rd, "\cong" below] & g \times g \arrow[d, "m"]\\
	 & g
\end{tikzcd}

&

\begin{tikzcd}[column sep=huge, row sep=huge]
	g \times 1 \arrow[r, "\textrm{id}_g \times e"] \arrow[rd, "\cong" below] & g \times g \arrow[d, "m"]\\
	 & g
\end{tikzcd}
\end{array} \]

\end{figure}

\begin{figure}[H]
\centering

\[ \begin{array}{cc}
\begin{tikzcd}[column sep=huge, row sep=huge]
	g \arrow[r, "\Delta"] \arrow[d] & g \times g \arrow[r, "\textrm{id}_g \times \iota"] & g \times g \arrow[d, "m"]\\
	1 \arrow[rr, "e"] & & g
\end{tikzcd}

&

\begin{tikzcd}[column sep=huge, row sep=huge]
	g \arrow[r, "\Delta"] \arrow[d] & g \times g \arrow[r, "\iota \times \textrm{id}_g"] & g \times g \arrow[d, "m"]\\
	1 \arrow[rr, "e"] & & g
\end{tikzcd}
\end{array} \]

\end{figure}

commute.