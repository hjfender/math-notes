\subsection{Higher Level Categories}\label{higherlevelcategories}

\subsubsection{Functor Categories}\label{functorcategories}

A \emph{functor category} is a category whose objects are \hyperref[functor]{functors} and whose arrows are \hyperref[naturaltransformations]{natural transformations}.
Since compositions of natural transformations are natural transformations, composition can be defined as in the following diagram\dots

\begin{figure}[H]
  \centering
  \begin{tikzcd}[column sep=huge, row sep=huge]
	Rc \arrow[r, "Rf"] \arrow[d, "\sigma c"] \arrow[dd, bend right, "(\tau \circ \sigma)c" left] & Rc' \arrow[d, "\sigma c'" left] \arrow[dd, bend left, "(\tau \circ \sigma)c'"] \\
	Sc \arrow[r, "Sf"] \arrow[d, "\tau c"]   													 & Sc' \arrow[d, "\tau c'" left] \\
	Tc \arrow[r, "Tf"] 						 													 & Tc'
\end{tikzcd}
\end{figure}

\subsubsection{2-Categories}

\subsubsubsection{Vertical Composition}\label{verticalcomposition}
For natural transformations $\tau$ and $\sigma$, we have "vertical" composition $\tau \dot \sigma$, as in the following diagram\dots

\begin{figure}[H]
  \centering
  \begin{tikzcd}[column sep=huge, row sep=huge]
	C \arrow[r] \arrow[d, "\sigma"] \arrow[dd, bend right, "\tau\sigma" left] & B \arrow[d, "\sigma"] \arrow[dd, bend left, "\tau\sigma" right] \\
	C \arrow[r] \arrow[d, "\tau"]  											  & B \arrow[d, "\tau"] 											\\
	C \arrow[r] 											 				  & B
\end{tikzcd}
\end{figure}

\subsubsubsection{Horizontal Composition}\label{horizontalcomposition}
We can also define "horizontal" composition for natural transformations $\tau$ and $\tau'$, $\tau' \circ \tau$, as in the following commutative diagrams\dots

\begin{figure}[H]
  \centering
  \begin{tikzcd}[column sep=huge, row sep=huge]
	C \arrow[r, "S"] \arrow[d, "\tau"] & B \arrow[r, "S'"] \arrow[d, bend right, "\tau" left] \arrow[d, bend left, "\tau'"] & A \arrow[d, "\tau'"]\\
	C \arrow[r, "T"] & B \arrow[r, "T'"] & A
\end{tikzcd}
\end{figure}

\begin{figure}[H]
  \centering
  \begin{tikzcd}[column sep=huge, row sep=huge]
	S'Sc \arrow[r, "\tau'Sc"] \arrow[d, "S'\tau c"] \arrow[rd, dotted, "(\tau' \circ \tau)c"] & T'Sc \arrow[d, "T'\tau c"] \\
	S'Tc \arrow[r, "\tau'Tc"]				   	  									  & T'Tc 
\end{tikzcd}
\end{figure}

\noindent The next diagram shows $\tau' \circ \tau : S'S \xrightarrow[]{\cdot} T'T$ is natural.

\begin{figure}[H]
  \centering
  \[ \begin{array}{cc}

\begin{tikzcd}[column sep=huge, row sep=huge]
	c \arrow[d, "f"]\\
	b
\end{tikzcd}

&

\begin{tikzcd}[column sep=huge, row sep=huge]
	S'Sc \arrow[r, "S'\tau c"] \arrow[d, "S'S f"] & S'Tc \arrow[r, "\tau' Tc"] \arrow[d, "S'T f"] & T'Tc \arrow[d, "T'T f"] \\
	S'Sb \arrow[r, "T'\tau b"] & S'Tb \arrow[r, "\tau' Tb"] & T'Tb
\end{tikzcd}

\end{array} \]
\end{figure}

\noindent So $\tau' \circ \tau = (T' \circ \tau) \cdot (\tau' \circ S) = (\tau' \circ T) \cdot (S' \circ \tau)$, which leads into our next concept.

\subsubsubsection{Interchange Law}\label{interchangelaw}
For natural transformations $\sigma,\sigma',\tau,\tau'$ satisfying\dots

\begin{figure}[H]
  \centering
  \begin{tikzcd}[column sep=huge, row sep=huge]
	C \arrow[r] \arrow[d, "\sigma"] & B \arrow[r] \arrow[d, bend right, "\sigma" left] \arrow[d, bend left, "\sigma'"] & A \arrow[d, "\sigma'"] \\
	C \arrow[r] \arrow[d, "\tau"] & B \arrow[r] \arrow[d, bend right, "\tau" left] \arrow[d, bend left, "\tau'"] & A \arrow[d, "\tau'"] \\
	C \arrow[r] & B \arrow[r] & A
\end{tikzcd}
\end{figure}

\noindent the \emph{interchange law} is $(\tau' \cdot \sigma') \circ (\tau \cdot \sigma)=(\tau' \circ \tau)\cdot(\sigma' \circ \sigma).$ \newline

\noindent The proof of the interchange law derives from the following diagram. Intuitively, the interchange law occurs along the dotted diagonal lines.

\begin{figure}[H]
  \centering
  \begin{tikzcd}[column sep=huge, row sep=huge]
	S'Sc \arrow[r, "\sigma'S"] \arrow[d, "S'\sigma"] \arrow[rd, dotted] \arrow[rdrd, dotted, bend left] & T'Sc \arrow[r, "\tau'S"] \arrow[d, "T'\sigma"] 				   & R'Sc \arrow[d, "R'\sigma'"] \\
	S'Tc \arrow[r, "\sigma'T"] \arrow[d, "S'\tau"]  				    & T'Tc \arrow[r, "\tau'T"] \arrow[d, "T'\tau"] \arrow[rd, dotted]  & R'Tc \arrow[d, "R'\tau'"]   \\
	S'Rc \arrow[r, "\sigma'R"] 					   	 				 	& T'Rc \arrow[r, "\tau'R"] 					  					   & R'Rc
\end{tikzcd}
\end{figure}

\begin{theorem}
The collection of natural transformations in the set of arrows of two different categories under two different operations of composition, $\cdot$ and $\circ$, which satisfy the interchange
law. Moreover, any arrow (transformation) which is an identity for the composition $\circ$ is also an identity for the composition $\cdot$.
\end{theorem}

\subsubsubsection{Double Category}\label{doublecategory}
The set of arrows for two different compositions with two different compositions which together satisfy the interchange law.

\subsubsubsection{2-Category}\label{twocategories}
A double category in which every identiy arrow for the first composition is also an identity for the second composition.