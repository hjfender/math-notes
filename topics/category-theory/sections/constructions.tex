\subsection{Category Constructions}\label{categoryconstructions}

\subsubsection{Products}\label{products}

Given categories $B$ and $C$ we construct the product category $B \times C$\dots
\begin{itemize}
  \item Objects: pairs of objects $\langle b,c \rangle$ ($b \in B$ and $c \in C$)
  \item Arrows: $\langle b,c \rangle \rightarrow \langle b',c' \rangle$ are a pair $\langle f,g \rangle$ of arrows ($f \in B$ and $g \in C$)
  \item Composition: $\langle f', g' \rangle \circ \langle f, g \rangle = \langle f' \circ f, g' \circ g \rangle$
\end{itemize}

\noindent The corresponding \hyperref[universality]{universal property} is: for any functors $R$ and $T$, there is a unique functor $F$ making the digram commute\dots

\begin{figure}[H]
\centering
\begin{tikzcd}
	& D \arrow[ld,"R" above] \arrow[d, dotted, "F"] \arrow[rd,"T"]& \\
	B  & B \times C \arrow[r, "Q" below] \arrow[l, "P"] & C
\end{tikzcd}
\end{figure}

\noindent Note: $P\langle f,g \rangle = f$ and $Q\langle f,g \rangle = g$ are called the \emph{projections}\label{projections} of the product.

\subsubsubsection{Products of Functors}\label{functorproducts}
Given functors $U$ and $V$, the functor product $U \times V$ satisfies\dots
\begin{itemize}
  \item $(U \times V)\langle b,c \rangle = \langle Ub,Uc \rangle$ for objects
  \item $(U \times V)\langle f,g \rangle = \langle Uf,Ug \rangle$ for arrows
\end{itemize}

\begin{figure}[H]
\centering
\begin{tikzcd}[column sep=huge, row sep=huge]
	B  \arrow[d, "U"] & B \times C \arrow[l, "P" above] \arrow[r, "Q"] \arrow[d, dotted, "U \times V"] & C \arrow[d, "V"]\\
	B 				  & B \times C \arrow[l, "P'" above] \arrow[r, "Q'"]							   & B
\end{tikzcd}
\end{figure}

\subsubsubsection{Bifunctors}\label{bifunctors}

A functor $S : B \times C \rightarrow D$. Intuitively, "a functor of two variables."\newline

\noindent Determined by the functors that result when any one object of exactly one of the categories is fixed. This is
recorded more explicitly in the following proposition\dots

\begin{proposition}
Let $B,C$, and $D$ be categories. For all objects $c \in C$ and $b \in B$, let
$$L_c : B \rightarrow D, \; M_b : C \rightarrow D$$
be functors such that $M_b(c) = L_c(b)$ for all $b$ and $c$. Then there exists a bifunctor $S: B \times C \rightarrow D$
with $S(-, c) = L_c$ for all $c$ and $S(b,-) = M_b$ for all $b$ if and only if for every pair of arrows $f:b \rightarrow b'$
and $g:c \rightarrow c'$ one has
$$M_{b'}g \circ L_cf = L_{c'}f \circ M_bg.$$
These equal arrows in $D$ are then the value $S(f,g)$ of the arrow function of $S$ at $f$ and $g$.
\end{proposition}

\begin{proof}
Observe\dots
$$\langle b',g \rangle \circ \langle f,c \rangle  = \langle b'f,gc \rangle = \langle f,g \rangle =  \langle fb,c'g \rangle = \langle f,c' \rangle \circ \langle b,g \rangle$$
(where $b,b',c,c'$ are identity arrows).\newline

\noindent This implies\dots
$$S(b',g)S(f,c)=S(f,c')S(b,g).$$
Which further implies\dots

\begin{figure}[H]
\centering
\begin{tikzcd}[column sep=huge, row sep=huge]
	S(b,c) \arrow[d, "S(f\textrm{,}c)"] \arrow[r, "S(b\textrm{,}g)"] & S(b,c') \arrow[d, "S(f\textrm{,}c')"] \\
	S(b',c) \arrow[r, "S(b'\textrm{,}g)"] & S(b',c') 
\end{tikzcd}
\end{figure}

\end{proof}

\subsubsubsection{Natural transformations between bifunctors}
Given $S,S':B \times C \rightarrow D$. Consider $\alpha(b,c) : S(b,c) \rightarrow S'(b,c)$. We say $\alpha$ is \emph{natural in}\label{naturalin} $b$ if $\forall c \in C$ the components $\alpha(b,c)$
for all $b$ define $\alpha(-,c): S(-,c) \xrightarrow[]{\cdot} S'(-,c)$, a natural transformation of functors $B \rightarrow D$.

\begin{proposition}
For bifunctors $S,S'$, the function $\alpha$ displayed above is a natural transformation $\alpha:S \xrightarrow[]{\cdot} S'$ (i.e., of bifunctors) if and only if $\alpha(b,c)$ is natural
in $b$ for each $c \in C$ and natural in $c$ for each $b \in B$.

\begin{figure}[H]
\centering
\begin{tikzcd}[column sep=huge, row sep=huge]
	S(b,c) \arrow[d, "S(f\textrm{,}g)"] \arrow[r, "\alpha(f\textrm{,}g)"] & S(b,c) \arrow[d, "S'(b\textrm{,}c)"] \\
	S(b',c') \arrow[r, "\alpha(b'\textrm{,}c')"] & S(b',c') 
\end{tikzcd}
\end{figure}

\end{proposition}

\subsubsubsection{The Universal Natural Transformation}\label{universalnaturaltransformation}

Given any natural transformation $\tau: S \xrightarrow[]{\cdot} T$ between $S,T: C \rightarrow B$ there is a unique functor $F: C \times 2 \rightarrow B$ with $F \mu c = \tau c$ for any object $c$.
\begin{itemize}
  \item $F\langle f,0 \rangle = Sf$
  \item $F\langle f,1 \rangle = Tf$
  \item $F\langle f, \downarrow \rangle = Tf \circ \tau c = \tau c' \circ Sf$ (where $\downarrow : 0 \rightarrow 1$)
\end{itemize}

\noindent Observe $C \times 2$ below\dots

\begin{figure}[H]
\centering
\begin{tikzcd}
	c \arrow[dr] \arrow[dd, "C \times 0 \textrm{ layer}" description ] \arrow[rr, "\mu c"] &    								   & c \arrow[dr, "C \times 1 \textrm{ layer}" description, near end ] \arrow[dd] &    \\
					  				     		 & c'' \arrow[rr, "\mu c''" near start] & 					     & c'' \\
	c' \arrow[ur] \arrow[rr, "\mu c'"]		     &	  								   & c' \arrow[ur] 			 &
\end{tikzcd}
\end{figure}

\noindent where $\mu c = \langle c, \downarrow \rangle$

\subsubsection{Coproducts}\label{coproducts}

Given categories $B$ and $C$ the dual of the product category is coproduct category $B \coprod C$.\newline

\noindent The corresponding \hyperref[universality]{universal property} is: for any functors $R$ and $T$, there is a unique functor $F$ making the digram commute\dots
\begin{figure}[H]
\centering
\begin{tikzcd}
	B \arrow[r, "I" above] & B \coprod C & C \arrow[l, "J" above]\\
	& D \arrow[lu,"R" below] \arrow[u, dotted, "F"] \arrow[ru,"T" below] &
\end{tikzcd}
\end{figure}

\subsubsection{Quotients}\label{quotients}

The \emph{quotient category} is specified in the following proposition.

\begin{proposition}
For a given category $C$, let $R$ be a function which assigns to each pair of objects $a,b$ of $C$ a binary relation $R_{a,b}$ on the hom-set $C(a,b)$.
Then there exist a category $C/R$ and a functor $Q = Q_R:C \rightarrow C/R$ such that\dots
\begin{enumerate}
  \item If $fR_{a,b}f'$ in $C$, then $Qf = Qf'$.
  \item If $H:C \rightarrow D$ is any functor from $C$ for which $f R_{a,b} f'$ implies $Hf = Hf'$ for all $f$ and $f'$, then there is a unique functor $H':C/R \rightarrow D$
  with $H' \circ Q_R = H$.
\end{enumerate}
Moreover, the functor $Q_R$ is a bijection on objects.
\end{proposition}

\noindent The corresponding \hyperref[universality]{universal property} is represented in the following diagram\dots

\begin{figure}[H]
\centering

\begin{tikzcd}
	C \arrow[r, "Q_R"] \arrow[dr, "H" below] & C/R \arrow[d, dotted, "H'"]\\
								 		   & D
\end{tikzcd}

\end{figure}

\subsubsubsection{Congruence}\label{congruence}
A \emph{congruence} is a relation $R$ on a category $C$ such that\dots
\begin{itemize}
  \item $\forall a,b \in \text{Obj}(C)$, $R_{a,b}$ is an \hyperref[equivalencerelation]{equivalence relation}
  \item if $f,f':a \rightarrow b$ have $f R_{a,b} f'$, then for all $g: a' \rightarrow a$ and all $h: b \rightarrow b'$ one has $(hfg)R_{a',b'}(hf'g)$.
\end{itemize}

\subsubsection{Free Categories}\label{freecategories}

\subsubsubsection{O-graph}\label{ograph}
The \emph{O-graph} is a directed graph on a fixed set $O$ of objects (not a simple graph). \newline

\noindent We define the \emph{product over O} as a \hyperref[composablepairsofarrows]{set of composable pairs of arrows}\dots
$$A \times_O B = \{ \langle g, f \rangle | \delta_0 g = \delta_1 f, g \in A, f \in B \}$$
where $\delta_0$, $\delta_1$, resp., are functions representing the $\textbf{dom}$, $\textbf{cod}$, resp., operations.\newline

\noindent A \hyperref[category]{category} with objects $O$ is an $O$-graph equipped with two morphisms $c:A \times_O A \rightarrow A$ and $i:O \rightarrow A$
of $O$-graphs making the following diagrams commutative.

\[ \begin{array}{cc}

\begin{tikzcd}[column sep=small]
	(A \times_O A) \times_O A \arrow[d,"c \times 1"] \arrow[r,"\cong"] & A \times_O (A \times_O A) \arrow[r,"1 \times c"] & A \times_O A \arrow[d,"c"]\\
	A \times_O A \arrow[rr,"c"] & & A
\end{tikzcd}

&

\begin{tikzcd}
	O \times_O A \arrow[r,"i \times 1"] \arrow[d,"\cong"] & A \times_O A \arrow[d,"c"] & A \times_O O \arrow[l,"1 \times i" above] \arrow[d,"\cong"]\\
	A \arrow[r,"="] & A & A \arrow[l,"="]
\end{tikzcd}

\end{array} \]

\subsubsubsection{Free Category}\label{freecategory}

Let $C(G)$ be the \emph{free category} generated by graph $G$, specified in the subsequent theorem\dots

\begin{theorem}
Let $G = \{A \rightrightarrows O\}$ be a small graph. There is a small category $C(G)$ with $O$ as its set of objects and a morphism $P: G \rightarrow UC$ of graphs from $G$ to
the underlying graph $UC$ of $C$ with the following property. Given any category $B$ and any morphism $D:G \rightarrow UB$ of graphs, there is a unique functor $D':C \rightarrow B$
with $(UD') \circ P = D$, as in the commutative diagram

\begin{figure}[H]
\centering
\begin{figure}[H]
\centering

\[ \begin{array}{cc}
\begin{tikzcd}
	C \arrow[d, dotted, "D'"]\\
	B
\end{tikzcd}

&

\begin{tikzcd}
	G \arrow[r, "P"] \arrow[dr, "D" below] & UC \arrow[d, dotted, "UD'"]\\
								 		   & UB
\end{tikzcd}
\end{array} \]

\end{figure}
\end{figure}

In particular, if $B$ had $O$ as set of objects and $D$ is a morphism of $O$-graphs, then $D'$ is the identity on objects.
\end{theorem}

\begin{corollary}
To any set $X$ there is a monoid $M$ and a function $p: X \rightarrow UM$, where $UM$ is the underlying set of $M$, with the following universal property: for any monoid $L$ and any
function $h: X \rightarrow UL$ there is a unique morphism $h': M \rightarrow L$ of monoids with $h: Uh' \circ p$.
$$\textrm{Hom}_{Cat}(C(G),B) \cong \textrm{Grph}(G,UB), \; D' \mapsto D = UD' \circ P$$
\end{corollary}

\subsubsection{Comma Categories}\label{commacategories}

\subsubsubsection{Category of objects unber b $(b \downarrow C)$}\label{objectsunder}
Objects $\langle f,c \rangle$:

\begin{figure}[H]
\centering
\begin{tikzcd}
	b \arrow[d, "f"]\\
	c
\end{tikzcd}
\end{figure}

\noindent Arrows $\langle f,c \rangle \xrightarrow[]{h} \langle f',c' \rangle$:

\begin{figure}[H]
\centering
\begin{tikzcd}
	 & b \arrow[dr, "f"] \arrow[dl, "f'" above] & \\
	c \arrow[rr, "h"] & & c'
\end{tikzcd}
\end{figure}

\subsubsubsection{Category of objects over a $(C \downarrow a)$}\label{objectsover}

Objects $\langle f,c \rangle$:

\begin{figure}[H]
\centering
\begin{tikzcd}
	c \arrow[d, "f"]\\
	a
\end{tikzcd}
\end{figure}

\noindent Arrows $\langle f,c \rangle \xrightarrow[]{h} \langle f',c' \rangle$:

\begin{figure}[H]
\centering
\begin{tikzcd}
	c \arrow[dr, "f" below] \arrow[rr, "h"] & & c' \arrow[dl, "f'"] \\
	  								  & a & 
\end{tikzcd}
\end{figure}

\subsubsubsection{Category of objects $S$-unber b $(b \downarrow S)$}\label{objectsunderfunctor}

Given a functor $S:D \rightarrow C$.\newline

\noindent Objects $\langle f,Sd \rangle$:

\begin{figure}[H]
\centering
\begin{tikzcd}
	b \arrow[d, "f"]\\
	Sd
\end{tikzcd}
\end{figure}

\noindent Arrows $\langle f,Sd \rangle \xrightarrow[]{Sh} \langle f',Sd' \rangle$:

\begin{figure}[H]
\centering
\begin{tikzcd}
	 & b \arrow[dr, "f"] \arrow[dl, "f'" above] & \\
	Sd \arrow[rr, "Sh"] & & Sd'
\end{tikzcd}
\end{figure}

\subsubsubsection{Category of objects $T$-over a $(T \downarrow a)$}\label{objectsoverfunctor}

Given a functor $T:E \rightarrow C$.\newline

\noindent Objects $\langle f,Te \rangle$:

\begin{figure}[H]
\centering
\begin{tikzcd}
	Te \arrow[d, "f"]\\
	a
\end{tikzcd}
\end{figure}

\noindent Arrows $\langle f,Te \rangle \xrightarrow[]{Th} \langle f',Te' \rangle$:

\begin{figure}[H]
\centering
\begin{tikzcd}
	Te \arrow[dr, "f" below] \arrow[rr, "Th"] & & Te' \arrow[dl, "f'"] \\
	  								  & a & 
\end{tikzcd}
\end{figure}

\subsubsubsection{Comma Category $(T \downarrow S)$}\label{commacategory}

Given functors $S:D \rightarrow C$ and $T:E \rightarrow C$.\newline

\noindent Objects $\langle e,d,f \rangle$:

\begin{figure}[H]
\centering
\begin{tikzcd}
	Te \arrow[d, "f"]\\
	Sd
\end{tikzcd}
\end{figure}

\noindent where $d \in$Obj$(D)$, $e \in$Obj$(E)$, $f:Te \rightarrow Sd$.\newline

\noindent Arrows $\langle e,d,f \rangle \xrightarrow[]{\langle k,h \rangle} \langle e',d',f' \rangle$:

\begin{figure}[H]
\centering
\begin{tikzcd}
	Te \arrow[d, "f"] \arrow[r, "Tk"] & Te' \arrow[d, "f'"] \\
	Sd				  \arrow[r, "Sh"] & Sd'
\end{tikzcd}
\end{figure}

\noindent where $k:e \rightarrow e'$, $h:d \rightarrow d'$ such that $f' \circ Tk = Sh \circ f$.\newline

\noindent Composition $\langle k',h' \rangle \circ \langle k,h \rangle = \langle k' \circ k, h' \circ h \rangle$ when defined.

\begin{figure}[H]
\centering
\begin{tikzcd}
& & T \downarrow S \arrow[lld, "P" above] \arrow[rrd, "R"] \arrow[d, "Q"] & & \\
E \arrow[r, "T" below] & C & C^2 \arrow[r, "C^{d_1}" below] \arrow[l, "C^{d_0}" below] & C & D \arrow[l, "S" below]
\end{tikzcd}
\end{figure}

\noindent $P$ and $Q$ are the \emph{projections} of the comma category. $C^{d_0},C^{d_1}$, resp., send arrows to domain, codomain, resp.

\begin{figure}[H]
\centering
\begin{tikzcd}
& & \langle e,d,f:Te \rightarrow Sd \rangle \arrow[lld, mapsto] \arrow[rrd, mapsto] \arrow[d, mapsto] & & \\
e \arrow[r, mapsto] & Te & (f:Te \rightarrow Sd) \arrow[r, mapsto] \arrow[l, mapsto] & Sd & d \arrow[l, mapsto]
\end{tikzcd}
\end{figure}