\subsection{Applications}\label{homotopyapplications}

\begin{theorem}[Brouwer's Theorem in dimension 2]
The two-disk $e^2 = \{ \textbf{x} \in \mathbb{R}^2 | ||\textbf{x}|| \leq 1 \} \subseteq \mathbb{R}^2$
has the fixed point property.
\end{theorem}

\begin{proof}
Suppose $f : e^2 \rightarrow e^2$ is a continuous function without a fixed point. Then for each $\textbf{x} \in e^2$, $f(\textbf{x}) \neq \textbf{x}$.
Define $g : e^2 \rightarrow S^1$ by\dots
$$g(\textbf{x}) = \textrm{ intersection of the ray from } f(\textbf{x}) \textrm{ to } \textbf{x} \textrm{ with } S^1.$$

To see that $g(\textbf{x})$ is continuous on $e^2$, we apply some vector geometry: write $Q = f(\textbf{x})$, $Z = g(\textbf{x})$. Let $O = (0,0)$ and define
$X = (\textbf{x} - Q)/||\textbf{x} - Q||.$ Then, $g(\textbf{x}) = Z = Q + tX$ for some $t \geq 0$ for which $Q + tX \in S^1$, that is, $(Q + tX) \cdot (Q + tX) = 1.$
This condition can be rewritten to solve for $t$, namely,
$$(Q + tX) \cdot (Q + tX) = t^2(X \cdot X) + 2t(Q \cdot X) + Q \cdot Q = 1.$$
The quadratic formula gives\dots
\begin{align*}
t_{\textbf{x}} &= - Q \cdot X + \sqrt{(Q \cdot X)^2 + 1 - Q \cdot Q}\\
			   &= -f(\textbf{x}) \cdot \frac{\textbf{x} - f(\textbf{x})}{||\textbf{x} - f(\textbf{x})||} + \sqrt{\left(f(\textbf{x}) \cdot \frac{\textbf{x} - f(\textbf{x})}{||\textbf{x} - f(\textbf{x})||}\right)^2 + 1 - f(\textbf{x}) \cdot f(\textbf{x})}.
\end{align*}
Note that this choice of signs gives $t_{\textbf{x}} \geq 0$, and $t_{\textbf{x}} = 0$ implies $f(\textbf{x}) = \textbf{x}$. Since we have
assumer that this doesn't happen, $t_{\textbf{x}} > 0$. Furthermore, $t_{\textbf{x}}$ is a continuous function of $\textbf{x}$. We can write
$g(\textbf{x})$ explicitly as\dots
$$g(\textbf{x}) = f(\textbf{x}) = t_{\textbf{x}}\frac{\textbf{x} - f(\textbf{x})}{||\textbf{x} - f(\textbf{x})||}$$
and so $g(\textbf{x})$ is continuous.

By the definition of the mapping $g$, if $\textbf{x} \in S^1 \subseteq e^2$, then $g(\textbf{x}) = \textbf{x}.$
We have constructed a continuous mapping $g : e^2 \rightarrow S^1$ for which $g \circ i = \textrm{id}_{S^1}$, that is,
the identity mapping on $S^1$ can be factored:
$$\textrm{id}_{S^1} : S^1 \xrightarrow[]{i} e^2 \xrightarrow[]{g} S^1.$$
The composite leads to a composite of group homomorphisms and fundamental groups:
$$\textrm{id} : \pi_1(S^1) \xrightarrow[]{i_*} \pi_1(e^2) \xrightarrow[]{g_*} \pi_1(S^1).$$
However, $\pi_1(e^2) = \{ [c_1] \}$ and so $g_* \circ i_*([\alpha]) = [c_1] \neq [\alpha]$ and $g_* \circ i_* \neq \textrm{id}$,
a contradiction. Therefore, a continuous function $f : e^2 \rightarrow e^2$ without fixed points is not possible.
\end{proof}

\begin{corollary}
$S^1$ is not a retract of $e^2$.
\end{corollary}

\begin{proposition}[The Borsuk-Ulam Theorem for $n=2$]
There does not exist a continuous function $f : S^2 \rightarrow S^1$ that satisfies
$f(-\textbf{x}) = -f(\textbf{x})$ for all $\textbf{x} \in S^2$.
\end{proposition}

\begin{proof}
Assume such a function exists. The condition satisfied by $f$ can be written $f(\pm \textbf{x}) = \pm f(\textbf{x})$.
It follows that $f$ induces $\hat{f} : \mathbb{R}P^2 \rightarrow \mathbb{R}P^1$ and $\hat{f}$ fits into a diagram:
\begin{figure}[H]
  \centering
  \begin{tikzcd}
S^2 \arrow[r, "f"] \arrow[d, "p"] & S^1 \arrow[d,"\overline{p}"]\\
\mathbb{R}P^2 \arrow[r, "\hat{f}"] & \mathbb{R}P^1
\end{tikzcd}
\end{figure}
for which $\overline{p} \circ f = \hat{f} \circ p$. Consider the induced homomorphism $\hat{f}_* : \pi_1(\mathbb{R}P^2) \rightarrow \pi_1(\mathbb{R}P^1)$.
Via considerations of fundamental groups, $\hat{f}_*$ is a homomorphism $\mathbb{Z}/2\mathbb{Z} \rightarrow \mathbb{Z}$. However, any such homomorphism must be
the trivial homomorphism. Let $\lambda: [0,1] \rightarrow S^2$ denote a path from the North Pole to the South Pole along a meridian of constant longitude.
It follows that $[p \circ \lambda] = [\alpha]$, a generator for $\mathbb{Z}/2\mathbb{Z} \cong \pi_1(\mathbb{R}P^2)$. Since the North and South Poles are antipodal,
these points are identified in $\mathbb{R}P^1$ after passage through $f$ and $\overline{p}$. Hence $[\overline{p} \circ f \circ \lambda]$ is nontrivial in $\pi_1(\mathbb{R}P^1)$.
But $[\overline{p} \circ f \circ \lambda] = [\hat{f} \circ p \circ \lambda] = \hat{f}_*([p \circ \lambda]) = 0$, a contradiction.
\end{proof}

\begin{corollary}
If $f: S^2 \rightarrow \mathbb{R}^2$ is a continuous function such that $f(-\textbf{x}) = -f(\textbf{x})$ for all $\textbf{x} \in S^2$,
then $f(\textbf{x}) = (0,0)$ for some $\textbf{x} \in S^2$.
\end{corollary}

\noindent The following theorem follows from the above development of the Borsuk-Ulam theorem.

\begin{theorem}
If $f: S^2 \rightarrow \mathbb{R}^2$ is a continuous function, then there exists a point $\textbf{x} \in S^2$ with $f(\textbf{x}) = f(-\textbf{x})$.
\end{theorem}

\begin{corollary}
No subset of $\mathbb{R}^2$ is homeomorphic to $S^2$.
\end{corollary}