\subsection{Covering Spaces}\label{coveringspaces}
Let $X$ be a space. A \emph{covering space} of $X$ is a path-connected space $\tilde X$ and a mapping $p: \tilde X \rightarrow X$
such that, for every $x \in X$, there is an open, path-connected subset $U$ with $x \in U$ for which each path component of $p^{-1}(U)$
is homeomorphic to $U$ by restriction of the mapping $p$. Such open sets are called \emph{elementary neighborhoods}\label{elementaryneighborhoods}.

\subsubsection{Path Lifting}\label{pathlifting}
\begin{lemma}
Let $p : \tilde X \rightarrow X$ be covering space and let $\tilde x_0 \in \tilde X$ with $p(\tilde x_0) = x_0 \in X$. If
$\lambda : [0,1] \rightarrow X$ is any path with $\lambda(0) = x_0$, then there exists a unique path $\tilde \lambda : [0,1] \rightarrow \tilde X$
with $\tilde \lambda (0) = \tilde x_0$ and $p \circ \tilde \lambda = \lambda$.
\end{lemma}

\begin{proof}
Cover $X$ by elementary neighborhoods. If $\lambda([0,1]) \subseteq U$ for some elementary neighborhood, then $x_0 \in U$ and $\tilde x_0 \in p^{-1}(U)$.
It follows that $\tilde x_0$ lies in some component $C_0$ of $p^{-1}(U)$ that is homeomorphis to $U$ via $(p \upharpoonright _{C_0})^{-1}(x_0) = \tilde x_0$,
since $\tilde x_0$ is the only point in $\tilde X$ corresponding to $x_0$ in this component. Finally, $p \circ \tilde \lambda = p \circ (p \upharpoonright _{C_0})^{-1} \circ \lambda = \lambda$.

If $\lambda([0,1]) \not \subseteq U$, consider the collection\dots
$$\{\lambda^{-1}(U') \subseteq [0,1] | U', \textrm{ an elementary neighborhood} \}.$$
This is a cover of $[0,1]$, which is a compact
metric space, and so by Lebesgue's Lemma we can choose $0 = t_0 < t_1 < \cdots < t_{n-1} < t_n = 1$ with each $\lambda([t_{i-1}, t_i])$ a subset of some elementary neighborhood (take $t_i - t_{i-1} < \delta$,
the Lebesgue number). Using the argument above, lift $\lambda$ on $[0,t_1]$. Then take $\lambda(t_1)$ as $x_0$ and $\tilde \lambda(t_1)$ as $\tilde x_0$ and lift $\lambda$ to $[t_1, t_2]$. Continuing in this
manner, we construct $\tilde \lambda$ on $[0,1]$ with $\tilde \lambda (0) = \tilde x_0$ and $p \circ \tilde \lambda = \lambda$.

Uniqueness is guaranteed by the following lemma.
\end{proof}

\begin{lemma}
Let $p : \tilde X \rightarrow X$ be a covering space and $Y$ a connected, locally connected space. Given two mappings $f_1 , f_2 : Y \rightarrow \tilde X$ with $p \circ f_1 = p \circ f_2$, then the set\dots
$$\{ y \in Y | f_1(y) = f_2(y) \}$$
is either empty or all of $Y$.
\end{lemma}

\begin{proof}
Consider the subset of $Y$ given by $B = \{ y \in Y | f_1(y) = f_2(y) \}.$ We show that $B$ is both open and closed. If $y \in \textrm{cls } B$, consider $x= p \circ f_1(y) = p \circ f_2(y)$ and
$U$ an elementary neighborhood containing $x$. The subset $(p \circ f_1)^{-1}(U) \cap (p \circ f_2)^{-1}(U)$ contains $y$. Because $Y$ is locally connected, there is an open set $W$ for which $y \in W \subseteq (p \circ f_1)^{-1} (U) \cap (p \circ f_2)^{-1}(U)$
with $W$ connected. Then $f_1(W)$ and $f_2(W)$ are connected subsets of $p^{-1}(U) \subseteq \tilde X$. Since $W$ is open and $y \in \textrm{cls } B$, there is a point $z \in W$ with $z \in B$. Thus
$f_1(z) = f_2(z)$ and $f_1(W) \cap f_2(W) \neq \emptyset$; therefore, $f_1(W)$ and $f_2(W)$ must lie in the same component of $p^{-1}(U)$. Since $p \circ f_1(y) = p \circ f_2(y)$ and the component in which
we find both $f_1(y)$ and $f_2(y)$ is homeomorphic to $U$ by the restriction of $p$, we have $f_1(y) = f_2(y)$. Thus $y \in B$ and $B$ is closed.

If we let $y \in B$, the argument above shows that the sets $f_1(W)$ and $f_2(W)$ lie in the same component $C_0$ of $p^{-1}(U)$. It follows that, for all $w \in W$,
$$f_1(w) = (p \upharpoonright _{C_0})^{-1} \circ p \circ f_1(w) = (p \upharpoonright _{C_0})^{-1} \circ p \circ f_2(w) = f_2(w)$$
and so $W$ is contained in $B$. Thus $B$ is open.

The only subsets of $Y$ that are both open and closed are $Y$ itself and $\emptyset$ and so we have proved the lemma.
\end{proof}

\subsubsection{Homotopy Lifting}\label{homotopylifting}

\begin{theorem}
Let $p : \tilde X \rightarrow X$ be a covering space and let $\eta_0, \eta_1 : [0,1] \rightarrow \tilde X$ be two paths in $\tilde X$ with $\eta_0(0) = \eta_1(0) = \tilde x_0$. If
$p \circ \eta_0(1) = x_1 = p \circ \eta_1(1)$ and $p \circ \eta_0 \simeq p \circ \eta_1$ via a homotopy that fixes the endpoints of the paths in $X$, then $\eta_1 \simeq \eta_2$ in
$\tilde X$ and, in particular, $\eta_0(1) = \eta_1(1)$.
\end{theorem}

\begin{proof}
Let $H : [0,1] \times [0,1] \times X$ be a homotopy between $p \circ \eta_0$ and $p \circ \eta_1$. In this case, we have, for all $s,t \in [0,1]$,
\begin{align*}
H(s,0) = p \circ \eta_0(s), && H(0,t) = p(\tilde x_0),\\
H(s,1) = p \circ \eta_1(s), && H(1,t) = p \circ \eta_0(1) = p \circ \eta_1(1).
\end{align*}
Since $[0,1] \times [0,1]$ is a compact metric space, when we cover it by the collection $\{ H^{-1}(U) | U, \textrm{ an elementary neighborhood of } X \}$, we can apply Lebesgue's Lemma to get
$\delta > 0$ for which any subset of $[0,1] \times [0,1]$ of diameter $< \delta$ lies in some $H^{-1}(U)$. If we subdivide the interval $[0,1]$,
$$0 = s_0 < s_1 < \cdots < s_{m-1} < s_m = 1$$
and
$$0 = t_0 < t_1 < \cdots < t_{n-1} < t_n = 1,$$
so that $s_i - s_{i-1} < \delta / 2$ and $t_j - t_{j-1} < \delta / 2$, then $H$ maps each subrectangle $[s_{i-1},s_i] \times [t_{j-1},t_j]$ into an elementary neighborhood for all $i$ and $j$.

To construct the lifting $\hat{H} : [0,1] \times [0,1] \rightarrow \tilde X$ and show it is a homotopy between $\eta_0$ and $\eta_1$, begin by lifting $H$ on $[0,s_1] \times [0,t_1]$ to $\tilde X$
by using $\hat{H} = (p \upharpoonright_{C_{11}})^{-1} \circ H$, where $C_{11}$ is the component of $p^{-1}(U_{11})$ containing $\eta_0(0)$ and $H([0,s_1] \times [0,t_1]) \subseteq U_{11}$, an elementary
neighborhood. Having done this, extend $\hat{H}$ next to $[s_1,s_2] \times [0,t_1]$. Notice that $\hat{H}$ has been defined on the line segment $\{s_1\} \times [0,t_1]$ which is connected and this determines the component of
$p^{-1}(U_{21})$ for the elementary neighborhood $U_{21}$ which contains $H([s_1,s_2]\times[0,t_1])$. Once the component, say $C_{21}$, is determined, extend $\hat{H}$ by $\hat{H} = (p \upharpoonright C_{21})^{-1} \circ H$.
Continue in this manner until $\hat{H}$ is defined on $[0,1] \times [0,t_1]$. Next, extend to $[0,1] \times [t_1, t_2]$ using the fact that the value of $\hat{H}$ has been determined on each successive subrectangle along the left
until $\hat{H}$ is defined on $[0,1] \times [0,1]$. By the preceding lemma, $\hat{H}$ is unique fulfilling the condition $\hat{H}(0,0) = \eta(0)$. Since $\eta_0(s)$ is also a lift of $H(s,0)$, we have that $\hat{H}(s,0) = \eta_0(s)$.
The condition $H(0,t) = p \circ \eta_0(0)$ implies that $\hat{H}(0,t) = \eta_0(0)$, that is, the homotopy $\hat{H}$ is constant on the subset $\{0\} \times [0,1]$. Thus, the lift $\hat{H}(s,1)$ of the path $p \circ \eta_1(s)$ in $X$
begins at $\eta_0(0) = \eta_1(0)$, and $\eta_1(s)$ is also such a lift. By uniqueness, $\hat{H}(s,1) = \eta_1(s)$. Finally, $H(1,t) = p \circ \eta_0(1) = p \circ \eta_1(1)$ for all $t \in [0,1]$,
$\hat{H}(1,t) = \eta_0(1)$, and we conclude that $\eta_0(1) = \eta_1(1)$ since $\hat{H}(1,t)$ is constant.
\end{proof}

\subsubsection{Fundamental Group Computations}\label{fundamentalgroupcomputations}

\begin{corollary}
Suppose $p : \tilde X \rightarrow X$ is a covering space:
\begin{enumerate}
  \item If $\eta : [0,1] \rightarrow \tilde X$ is a loop at $\tilde x_0$ and $p \circ \eta$ is homotopic to the constant loop $c_{x_0}$ for $x_0 = p(\tilde x_0)$, then $\eta \simeq c_{\tilde x_0}$.
  \item The induced homomorphism $p_{*} : \pi_1(\tilde X, \tilde x_0) \rightarrow \pi_1(X, x_0)$ is injective.
  \item For all $x \in X$, the subsets $p^{-1}(\{x\})$ of $\tilde X$ have the same cardinality.
\end{enumerate}
\end{corollary}

\begin{proof}
(1) One life of $c_{x_0}$ is simply the constant path $c_{\tilde x_0}$. By the homotopy lifting theorem,
$p \circ \eta \simeq p \circ c_{\tilde x_0} = c_{x_0}$ implies $\eta \simeq c_{\tilde x_0}.$

(2) If $p_*([\lambda]) = p_*([\mu])$, then, because $p_*$ is a homomorphism, $p_*([\lambda] * [\mu^{-1}]) = [c_{x_0}]$, that is,
$p \circ (\lambda * \mu^{-^1}) \simeq c_{x_0}$. By (1), $\lambda * \mu^{-1} \simeq c_{\tilde x_0}$ or $\lambda \simeq \mu$, that is,
$[\lambda] = [\mu]$.

(3) Suppose $x_0$ and $x_1$ are in $X$ and $\lambda : [0,1] \rightarrow X$ is a path joining $x_0$ to $x_1$. Suppose $y \in p^{-1}(\{x_0\}).$ We define
a mapping $A : p^{-1}(\{ x_0 \}) \rightarrow p^{-1}(\{ x_1 \})$ by lifting $\lambda$ to $\lambda_y : [0,1] \rightarrow \tilde X$ with $\lambda_y(0) = y$.
Define $A(y) = \lambda_y(1)$. SInce $\lambda_y$ is uniquely determined by being a lift of $p \circ \lambda_y = \lambda$ with $\lambda_y(0) = y$, the function $A$ is well defined.
By the path lifting 'uniqueness' lemma, lifts of $\lambda$ beginning at different elements in $p^{-1}(\{ x_0 \})$ must end at different points in $p^{-1}(\{ x_1 \})$ and so $A$ is injective.
Using lifts of $\lambda^{-1}$ we deduce that $A$ is surjective. (Notice that a different choice of $\lambda$ might give a different one-one correspondence $A$.)
\end{proof}

\begin{theorem}
$\pi_1(S^1) \cong \mathbb{Z}$.
\end{theorem}

\begin{proof}
If $\beta : [0,1] \rightarrow S^1$ is any loop at $1 \in S^1$, then the lift
of $\beta$ to $\hat{\beta} : [0,1] \rightarrow \mathbb{R}$ satisfies $\hat{\beta}(1) \in \mathbb{Z}$. The properties of liftings
determine a function $\Xi : \pi_1(S^1) \rightarrow \mathbb{Z}$ given by $[\beta] \mapsto \hat{\beta(1)}$.

Let $\alpha : [0,1] \rightarrow S^1$ be given by $\alpha(t) = (\cos(2 \pi t), \sin(2 \pi t)).$ Since
$\alpha = w \upharpoonright_{[0,1]}$, we see that one lift of $\alpha$ to $\mathbb{R}$ is just the identity and $\hat{\alpha}(1) = 1$.
It follows that $\alpha$ is not homotopic to the constant map at $1$, $c_1$. Next consider $\alpha^n$ for $n \in \mathbb{Z}$, given by $\alpha^n(t) = (\cos(2 \pi n t), \sin(2 \pi n t))$.
By the same argument for $\alpha$, $\hat{\alpha}^n(1) = n$ and so the mapping $\Xi : \pi_1(S^1) \rightarrow \mathbb{Z}$ is surjective.
Since each $\alpha^n \not \simeq c_1$ for $n \neq 0$, the subgroup generated by $[\alpha]$, isomorphic to $\mathbb{Z}$, is a subgroup of $\pi_1(S^1)$.

To finish the proof, we show that if $\beta$ is any loop based at $1$ in $S^1$, then $\beta \simeq \alpha^n$ for some $n \in \mathbb{Z}$.
Let $U_1 = \{ (x,y) \in S^1 | y > -1/10 \}$ and $U_2 = \{(x,y) \in S^1 | y < 1/10 \}$. The pair $\beta^{-1}(U_1)$, $\beta^{-1}(U_2)$ is an
open cover of $[0,1]$ and by Lebesgue's Lemma we can subdivide $[0,1]$ as $0 = t_0 < t_1 < \cdots < t_{m-1} < t_m = 1$ so that\dots
$$\textrm{i) } \beta([t_i,t_{i+1}]) \subseteq U_1 \textrm{ or } \beta([t_i, t_{i+1}]) \subseteq U_2 \textrm{ for } 0 \leq i < m.$$
Form the union of consecutive subintervals when both are mapped to the same $U_j$, $j=1$ or $2$. In detail, let $s_0 = 0$ and $s_1 = t_{i_1}$,
where $\beta([0,t_{i_1}]) \subseteq U_{j_1}$ for $j_1$ one of $1$ or $2$ and $\beta([t_{i_1},t_{i_1 + 1}]) \not \subseteq U_{j_1}$.
Let $U_{j_2} \neq U_{j_1}$ and $\beta([t_{i_1},t_{i_1 + 1}]) \subseteq U_{j_2}$. Then let $s_1 = t_{i_2}$, where
$\beta([t_{i_1}, t_{i_2}]) \subseteq U_{j_2}$ but $\beta([t_{i_2}, t_{i_2 + 1}]) \not \subseteq U_{j_2}$. Continue in this manner to get\dots
$$0 = s_0 < s_1 < \cdots < s_{k-1} < s_k = 1$$
so that
$$\textrm{ii) } \beta([s_{j-1}, s_j]) \textrm{ and } \beta([s_j,s_{j+1}]) \textrm{ are not both contained in the same } U_k, \textrm{ for } k = 1,2.$$

Let $\beta_j : [0,1] \rightarrow S^1$ denote the reparameterization of $\beta \upharpoonright_{[s_j, s_{j+1}]}$ so that
$\beta \simeq \beta_0 * \beta_1 * \cdots * \beta_{k-1}$, and each $\beta_j$ is a path in exactly one of $U_1$ or $U_2$.
Furthermore, $\beta(s_j) \in U_1 \cap U_2$, a subspace of two components, one of which contains $1 = e^{2 \pi i 0}$ and the other $-1 = e^{\pi i}$. For
$0 < j < m$ choose a path $\lambda_j : [0,1] \rightarrow U_1 \cap U_2$ with $\lambda_j(0) = \beta(s_j) = \beta_{j-1}(s_j)$ and $\lambda_j(1) = 1$ or $-1$,
depending on the component. Define\dots
$$\gamma_1 = \beta_0 * \lambda_1,$$
$$\gamma_j = \lambda_{j-1}^{-1} * \beta_{j-1} * \lambda_j \textrm{ for } 1 < j < k,$$
$$\gamma_k = \lambda^{-1}_{m-1} * \beta_{k-1}.$$

By cancelling $\lambda_j * \lambda_j^{-1}$, $\beta \simeq \gamma_1 * \gamma_2 * \cdots * \gamma_k$. Consider the paths $\gamma_l$.
If $\gamma_l$ is a closed path, it lies in $U_1$ or $U_2$, which are simply-connected and so $\gamma_l \simeq c_1$ or $\gamma_l \simeq c_{-1}$.
If $\gamma_l$ is not closed, then it crosses between the components of $U_1 \cap U_2$. It follows that $\gamma_l \simeq \eta_1^{\pm 1}$ or $\gamma_l \simeq \eta_2^{\pm 1}$,
where $\eta_1(t) = (\cos(\pi t), \sin(\pi t))$, the path joining $1$ to $-1$ in $U_1$, and $\eta_2(t) = (\cos(\pi t + \pi), \sin(\pi t + \pi))$, the path joining $-1$ to $1$ in $U_2$.
Making the cancellations of the type $\eta_1\eta_1^{-1} \simeq c_1$ or $\eta_2\eta_2^{-1} \simeq c_{-1}$, we
are left with three possibilities:
$$\beta \simeq c_1, \; \beta \simeq \eta_1 * \eta_2 * \eta_1 * \eta_2 * \cdots * \eta_1 * \eta_2, \textrm{ or}$$
$$\beta \simeq \eta_2^{-1} * \eta_1^{-1} * \eta_2^{-1} * \cdots * \eta_2^{-1}* \eta_1^{-1},$$
after cancelling out $c_{\pm 1}$. The ordering is determined by the fact that $\beta$ begins and ends at $1$, and each $\gamma_l$ either joins $1$ to $-1$, joins $-1$ to $1$,
or it simply stays put. After cancellation of the paths that stay put or products of paths that are homotopic to the constant path, we are left with such a product in that order.
Finally, $w \upharpoonright_{[0,1]} = \alpha \simeq \eta_1 * \eta_2$ and so $\beta \simeq \alpha^n$ for some $n \in \mathbb{Z}$.
\end{proof}

\begin{theorem}
$\pi_1(\mathbb{R}P^2) \cong \mathbb{Z}/2\mathbb{Z}$.
\end{theorem}

\begin{proof}
Consider the model of the projective plane given by the \emph{di-gon}, a disk with each point on the boundary identified with the point symmetric with respect to the origin. Let $s_0 \in \mathbb{R}P^2$
be the point $x_0 = [\pm(1,0,0)].$ Let $p : S^2 \rightarrow \mathbb{R}P^2$ denote the covering space $p(\textbf{x}) = [\pm \textbf{x}]$. Let the loop $a$ in $\mathbb{R}P^2$ denote half of the equator, and
lift $a$ to $S^2$. We get a path $\hat{a}$ from $(1,0,0)$ to $(-1,0,0)$ along the equator of $S^2$. By the corollary at the beginning of this subsection, $a \not \simeq c_{x_0}$. In the di-gon representation of $\mathbb{R}P^2$,
$a * a = a^2$ surrounds the disk, and so $a^2$ can be contracted to $c_{x_0}$ by shrinking to the center of the disk and translating over to $x_0$. If follows that $\pi_1(\mathbb{R}P^2)$ contains $\mathbb{Z}/2\mathbb{Z}$. To finish, we need
show that any loop at $x_0$ is homotopic to $a^n$ for some $n \in \mathbb{Z}$. Using the di-gon we see that away from the image of the path $a^2$ a path lies in the contractible interior of a disk. The disk can be used
to push any loop onto $a$ as often as it crosses between the copies of $x_0$. Thus we see that any loop based at $x_0$ is homotopic to $a^n$ for some $n \in \mathbb{Z}$ and so homotopic to $a$ or $c_{x_0}$. This implies that\dots
$$\pi_1(\mathbb{R}P^2) = \langle [a] \rangle/([a]^2 = [c_{x_0}]) \cong \mathbb{Z}/2\mathbb{Z}.$$
\end{proof}