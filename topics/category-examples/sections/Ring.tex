\subsection{The category Ring}
\begin{itemize}
  \item Objects: Rings
  \item Arrows: Ring homomorphisms
\end{itemize}

\subsubsection{Morphisms}

\begin{proposition}
For a ring homomorphism $\varphi : R \rightarrow S$, the following are equivalent:
\begin{enumerate}
  \item $\varphi$ is a monomorphism;
  \item ker$\varphi = \{0\}$;
  \item $\varphi$ is injective (as a set-function).
\end{enumerate}
\end{proposition}

\begin{proof}
Only $(1) \Rightarrow (2)$ warrants serious attention. Assume $\varphi : R \rightarrow S$ ois a monomorphism and $r \in \textrm{ker}\varphi$.
Applying the extension property given from the \hyperref[universalpropertyofpolynomialrings]{universal property of polynomial rings}, we obtain
unique ring homomorphisms $ev_r : \mathbb{Z}[x] \rightarrow R$ such that $ev_r(x) = r$ and $ev_0 : \mathbb{Z}[x] \rightarrow R$ such that $ev(x) = 0$.
Consider the parallel ring homomorphisms:
$$\mathbb{Z}[x] \underset{ev_0}{\overset{ev_r}{\rightrightarrows}} R \xrightarrow[]{\varphi} S,$$
since $\varphi(r) = 0 = \varphi(0),$ the two compositions $\varphi \circ ev_r$, $\varphi \circ ev_0$ agree (because they agree on $\mathbb{Z}$ and they agree on $x$);
hence $ev_r = ev_0$ since $\varphi$ is a monomorphism. Therefore\dots
$$r = ev_r(x) = ev_0(x) = 0,$$
showing $r \in \textrm{ker}\varphi.$
\end{proof}

\noindent In Ring, epimorphisms need not be surjective.

\begin{proposition}
The function $\iota: \mathbb{Z} \hookrightarrow \mathbb{Q}$ is an epimorphism.
\end{proposition}

\begin{proof}
Suppose $\alpha_1$ and $\alpha_2$ are parallel ring homomorphisms\dots
$$\mathbb{Z} \hookrightarrow \mathbb{Q} \underset{\alpha_2}{\overset{\alpha_1}{\rightrightarrows}} R$$
and $\alpha_1, \alpha_2$ agree on $\mathbb{Z}$. Then $\alpha_1, \alpha_2$ must agree on $\mathbb{Q}$: because
for $p,q \in \mathbb{Z}, q \neq 0,$
$$\alpha_i\left(\frac{p}{q}\right) = \alpha_i(p)\alpha_i(q^{-1})=\alpha(p)\alpha(q)^{-1}$$
is the same for both.
\end{proof}

\subsubsection{Universal Objects}

\begin{proposition}
Zero rings are \hyperref[final]{final objects} in Ring.
\end{proposition}

\begin{proposition}
The ring of integers $\mathbb{Z}$ is an \hyperref[final]{initial object} in Ring.
\end{proposition}

\begin{proof}
Observe $\varphi : \mathbb{Z} \rightarrow R$ defined by $(\forall n \in \mathbb{Z}): \; \; \varphi(n) = n \cdot 1_R$ is a
ring homomorphism by\dots
$$\varphi(mn) = \sum^{mn}_{i=1} 1_R = \sum^{m}_{i=1}(\sum^{n}_{j=1} 1_R) \overset{!}{=} (\sum^{m}_{i=1} 1_R) \cdot (\sum^{n}_{j=1} 1_R) = \varphi(m) \cdot \varphi(n),$$
(where $!$ occurs via the distributivity axiom) and is unique, since it is determined by the requirement that $\varphi(1) = 1_R$ and by the fact that $\varphi$ preserves addition. 
\end{proof}