\subsection{The category R-Mod}
\begin{itemize}
  \item Objects: $R$-modules (where $R$ is commutative)
  \item Arrows: $R$-module homomorphisms
\end{itemize}

\subsubsection{Morphisms}

\begin{proposition}
The following hold in $R$-Mod:
\begin{itemize}
  \item kernels and cokernels exists
  \item $\varphi$ is a monomorphism $\Leftrightarrow$ ker $\varphi$ is trivial $\Leftrightarrow$ $\varphi$ is injective as a set function
  \item $\varphi$ is an epimorphism $\Leftrightarrow$ coker $\varphi$ is trivial $\Leftrightarrow$ $\varphi$ is surjective as a set function
\end{itemize}
\noindent Further, every monomorphism identifies its source with the kernel of some morphism, and every epimorphism identifies its target
with the cokernel of some morphism.
\end{proposition}

\subsubsection{Isomorphism Theorems}\label{rmoduleisomorphismtheorems}

\begin{theorem}[Canonical Decomposition in R-Mod]
\label{rmodulecanonicaldecomposition}
Every $R$-module homomorphism $\varphi : M \rightarrow M'$ may be decomposed as follows:

\begin{figure}[H]
  \centering
  \begin{tikzcd}
	M \arrow[r, two heads] \arrow[rrr, bend left, "\varphi" above] & (M/\textrm{ker} \varphi) \arrow[r, "\sim" above, "\tilde \varphi" below] & \textrm{im}\varphi \arrow[r, hook] & M'
\end{tikzcd}
\end{figure}

\noindent where the isomorphism $\tilde \varphi$ in the middle is the homomorphism induced by $\varphi$ as in \ref{quotientrmodulethm}.
\end{theorem}

\begin{corollary}
Suppose $\varphi : M \rightarrow M'$ is a surjective $R$-module homomorphism. Then\dots
$$M' \cong \frac{M}{\textrm{ker}\varphi}.$$
\end{corollary}

\begin{theorem}
Let $N,P$ be submodules of an $R$-module $M$. Then\dots
\begin{itemize}
  \item $N + P$ is a submodule of $M$;
  \item $N \cap P$ is a submodule of $P$, and
  			$$\frac{N + P}{N} \cong \frac{P}{N \cap P}.$$
\end{itemize}
\end{theorem}

\begin{theorem}
Let $N$ be a submodule of an $R$-module $M$, and let $P$ be a submodule of $M$ containing $N$. Then $P/N$ is an ideal of $M/N$,
and\dots
$$\frac{M/N}{P/N} \cong \frac{M}{P}.$$
\end{theorem}

\subsubsection{Universal Objects}

\begin{proposition}
Trivial groups have a unique module structure over any ring $R$ and is a \hyperref[null]{null} object in R-Mod.
\end{proposition}

\noindent $R$-Mod is a similar category to that of Ab, note\dots

\begin{proposition}
Hom$_{R-Mod}(M,N)$ is an object in $R$-Mod.
\end{proposition}

\begin{proposition}
$R$-Mod has products and coproducts. See \hyperref[productmodules]{}.
\end{proposition}