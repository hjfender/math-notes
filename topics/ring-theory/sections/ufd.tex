\subsection{Unique Factorization Domains}\label{ufds}

\subsubsection{Definition}
An integral domain $R$ is a \emph{unique factorization domain} if every nonzero, nonunit element $r \in R$ has a unique factorization
into irreducibles.

\begin{lemma}
Let $R$ be a UFD, and let $a,b,c$ be nonzero elements of $R$. Then\dots
\begin{itemize}
  \item $(a) \subseteq (b) \Leftrightarrow$ the multiset of irreducible factors of $b$ is contained in the multiset of irreducible factors of $a$;
  \item $a$ and $b$ are associates (that is, $(a) = (b)$) $\Leftrightarrow$ the two multisets coincide;
  \item the irreducible factors of a product $bc$ are the collection of all irreducible factors of $b$ and of $c$.
\end{itemize}
\end{lemma}

\begin{lemma}
Let $R$ be a UFD, and let $a,b$ be nonzero elements of $R$. Then $a,b$ have a greatest common divisor.
\end{lemma}

\begin{lemma}
Let $R$ be a UFD, and let $a$ be an irreducible element of $R$. Then $a$ is prime.
\end{lemma}

\begin{theorem}
An integral domain $R$ is a UFD if and only if\dots
\begin{itemize}
  \item the a.c.c for principal ideals holds in $R$ and
  \item every irreducible element of $R$ is prime.
\end{itemize}
\end{theorem}