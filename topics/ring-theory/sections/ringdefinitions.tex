\subsection{Definitions}\label{ringdefinition}
A \emph{ring} $\langle R, +, \cdot \rangle$ is an \hyperref[abeliangroupdefinition]{abelian group} $\langle R,+ \rangle$ endowed with a \emph{second}
binary operation $\cdot$, satisfying on its onw the requirements of being associative and having a two-sided identity, i.e.
\begin{itemize}
  \item $(\forall r,s,t \in R): \; \; (r \cdot t) \cdot t = r \cdot (s \cdot t)$
  \item $(\exists 1_R \in R) (\forall r \in R): \; \; r \cdot 1_R = r = 1_R \cdot r$
\end{itemize}
which make $\langle R, \cdot \rangle$ a \emph{monoid}, and further interacting with $+$ via the following \emph{distributive properties}:
$$(\forall r,s,t \in R): \; \; (r+s)\cdot t = r \cdot t + s \cdot t \textrm{ and } t \cdot (r + s) = t \cdot r + t \cdot s.$$

\begin{lemma}
In a ring $R$,
$$0 \cdot r = r = r \cdot 0$$
and
$$r + (-1) \cdot r = 0$$
for all $r \in R.$
\end{lemma}

\begin{proof}
Observe\dots
$$r \cdot 0 = r \cdot (0 + 0) = r \cdot 0 + r \cdot 0 \Rightarrow 0 = r \cdot 0$$
and\dots
$$r + (-1) \cdot r = (1) \cdot r + (-1) \cdot r = (1 - 1) \cdot r = 0 \cdot r = 0$$
\end{proof}

\subsubsection{Commutative Rings}\label{commutativeringdefinition}
A ring $R$ is \emph{commutative} if $(\forall r,s \in R): \; r \cdot s = s \cdot r$.

\subsubsection{Subrings}\label{subrings}
A \emph{subring} $S$ of a ring $R$ is a ring whose underlying set is a subset of $R$ and such that
the inclusion function $S \hookrightarrow R$ is a ring homomorphism.

\subsubsection{Ideals}\label{ideal}
Let $R$ be a ring. A subgroup $I$ of $\langle R,+ \rangle$ is a \emph{left-ideal} of $R$ if $rI \subseteq I$
for all $r \in R$; that is,
$$(\forall r \in R)(\forall a \in I): \; ra \in I;$$
it is a \emph{right-ideal} if $Ir \subseteq I$ for all $r \in R$; that is,
$$(\forall r \in R)(\forall a \in I): \; ar \in I.$$
A \emph{two-sided ideal} is a subgroup $I$ which is both a left- and a right-ideal.

\noindent Some important features to keep in mind about ideals are\dots
\begin{itemize}
  \item If $\{ I_{\alpha}\}_{\alpha \in A}$ is a collection of ideals of a ring $R$. Then the intersection
  $\bigcap_{\alpha \in A} (I_{\alpha})$ is an ideal of $R$; the largest ideal contained in all of the ideals
  $I_{\alpha}$.
  \item If $I$, $J$ are ideals of $R$, then $IJ$ denotes the ideal \emph{generated} by all products $ij$ with
  $i \in I, j \in J$. More generally, if $I_1, \dots, I_n$ are ideals in $R$, then the 'product' $I_1 \cdots I_n$
  denotes the ideal generated by all products $i_1 \cdots i_n$ with $i_k \in I_k$.
\end{itemize}

\subsubsubsection{Principal Ideals}\label{principalideal}
Let $a \in R$ be any element of a ring. Then the subset $I = Ra$ of $R$ is a left-ideal of $R$ and $aR$ is a right-ideal.\newline

\noindent If $R$ is commutative, then we write $(a)$ for the ideal. It is called the \emph{principal ideal} generated by $a$.\newline

\noindent Some important features to keep in mind about principal ideals are\dots
\begin{itemize}
  \item $(a_{\alpha})_{\alpha \in A} := \sum_{\alpha \in A} (a_{\alpha})$ the ideal \emph{generated by the elements} $a_{\alpha}$
  \item $(R/(a))/(\overline{b}) \cong R/(a,b)$ where $(\overline{b})$ is the class of $b \in R/(a)$
\end{itemize}

\subsubsubsection{Finitely Generated}\label{finitelygenerated}
An ideal $I$ of a commutative ring $R$ is \emph{finitely generated} if $I = (a_1,\dots,a_n)$ for some $a_1, \dots, a_n \in R$.

\subsubsection{Characteristic}\label{characteristic}
Let $R$ be a ring and consider the unique ring homomorphism $\phi: \mathbb{Z} \rightarrow R$. Then ker$\phi = n\mathbb{Z}$
for some $n$. The \emph{characteristic} of $R$ is this nonnegative integer $n$.