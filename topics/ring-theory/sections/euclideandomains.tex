\subsection{Euclidean Domains}\label{euclideandomains}

\subsubsection{Euclidean Valuation}\label{euclideanvaluation}
A \emph{Euclidean valuation} of an integral domain $R$ is a function $v : R \setminus \{ 0 \} \rightarrow \mathbb{Z}^{\geq 0}$ satisfying the following property:
for all $a \in R$ and all nonzero $b \in R$ there exist $q,r \in R$ such that\dots
$$a = qb + r,$$
with either $r=0$ or $v(r) < v(b)$.

\subsubsection{Definition}\label{definition}
An integral domain $R$ is a \emph{Euclidean domain} if it admits a Euclidean valuation.

\begin{proposition}
Let $R$ be a Euclidean domain. Then $R$ is a PID.
\end{proposition}

\subsubsection{Euclidean Algorithm}\label{euclideanalgorithm}

\begin{lemma}
Let $a = bq + r$ in a ring $R$. Then $(a,b) = (b,r)$.
\end{lemma}

\begin{corollary}
Assume $a = bq +r$. Then $a,b$ have a gcd if and only if $b,r$ have a gcd, and in this case gcd$(a,b)$ $=$ gcd$(b,r)$.
\end{corollary}

\begin{proposition}
Given two elements $a,b \in R$, with $b \neq 0$, we can apply division with remainder repeatedly:
$$a = bq_1 + r_1,$$
$$b = r_1q_2 + r_2,$$
$$r_1 = r_2q_3 + r_3,$$
$$\dots$$
as long as the remainder $r_i$ is nonzero.

In a Euclidean domain this process terminates.
\end{proposition}

\begin{proposition}
The final remainder in the process above, $r_{N-1}$, is a gcd of $a,b$.
\end{proposition}