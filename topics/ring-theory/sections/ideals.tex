\subsection{Ideals}\label{ideal}
Let $R$ be a ring. A subgroup $I$ of $\langle R,+ \rangle$ is a \emph{left-ideal} of $R$ if $rI \subseteq I$
for all $r \in R$; that is,
$$(\forall r \in R)(\forall a \in I): \; ra \in I;$$
it is a \emph{right-ideal} if $Ir \subseteq I$ for all $r \in R$; that is,
$$(\forall r \in R)(\forall a \in I): \; ar \in I.$$
A \emph{two-sided ideal} is a subgroup $I$ which is both a left- and a right-ideal.\newline

\noindent Some important features to keep in mind about ideals are\dots
\begin{itemize}
  \item If $\{ I_{\alpha}\}_{\alpha \in A}$ is a collection of ideals of a ring $R$. Then the intersection
  $\bigcap_{\alpha \in A} (I_{\alpha})$ is an ideal of $R$; the largest ideal contained in all of the ideals
  $I_{\alpha}$.
  \item If $I$, $J$ are ideals of $R$, then $IJ$ denotes the ideal \emph{generated} by all products $ij$ with
  $i \in I, j \in J$. More generally, if $I_1, \dots, I_n$ are ideals in $R$, then the 'product' $I_1 \cdots I_n$
  denotes the ideal generated by all products $i_1 \cdots i_n$ with $i_k \in I_k$.
\end{itemize}

\subsubsection{Principal Ideals}\label{principalideal}
Let $a \in R$ be any element of a ring. Then the subset $I = Ra$ of $R$ is a left-ideal of $R$ and $aR$ is a right-ideal.\newline

\noindent If $R$ is commutative, then we write $(a)$ for the ideal. It is called the \emph{principal ideal} generated by $a$.\newline

\noindent Some important features to keep in mind about principal ideals are\dots
\begin{itemize}
  \item $(a_{\alpha})_{\alpha \in A} := \sum_{\alpha \in A} (a_{\alpha})$ the ideal \emph{generated by the elements} $a_{\alpha}$
  \item $(R/(a))/(\overline{b}) \cong R/(a,b)$ where $(\overline{b})$ is the class of $b \in R/(a)$
\end{itemize}

\subsubsection{Finitely Generated}\label{finitelygenerated}
An ideal $I$ of a commutative ring $R$ is \emph{finitely generated} if $I = (a_1,\dots,a_n)$ for some $a_1, \dots, a_n \in R$.

\subsubsection{Prime Ideals}\label{primeideal}
Let $I \neq (1)$ be an ideal of a commutative ring $R$. $I$ is a \emph{prime ideal} if $R/I$ is an integral domain.

\begin{proposition}
\label{primeidealcharacterization}
Let $I \neq (1)$ be an ideal of a commutative ring $R$. Then $I$ is prime if and only if for all $a,b \in R$\dots
$$ab \in I \Rightarrow (a \in I \textrm{ or } b \in I).$$
\end{proposition}

\begin{proof}
The ring $R/I$ is an integral domain if and only if $\forall \overline{a}, \overline{b} \in R/I$\dots
$$\overline{a} \cdot \overline{b} \Rightarrow (\overline{a} = 0 \textrm{ or } \overline{b} = 0).$$
This condition translates immediately to the given condition in $R$.
\end{proof}

\subsubsection{Maximal Ideals}\label{maximalideal}
Let $I \neq (1)$ be an ideal of a commutative ring $R$. $I$ is a \emph{maximal ideal} if $R/I$ is a field.

\begin{proposition}
\label{maximalidealcharacterization}
Let $I \neq (1)$ be an ideal of a commutative ring $R$. Then $I$ is maximal if and only if for all ideals $J$ or $R$\dots
$$I \subseteq J \Rightarrow (I=J \textrm{ or } J=R).$$
\end{proposition}

\begin{proof}
As for maximality, the given condition follows from the correspondence between ideals of $R/I$ and ideals of $R$ containing $I$
and the observation that a commutative ring is a field if and only if its ideals are $(0)$ and $(1)$.
\end{proof}

\noindent Existence of maximal ideals is equivalent to the axiom of choice, so we are justified in proving the following with \hyperref[zornslemma]{Zorn's Lemma}.

\begin{proposition}
Let $I \neq (1)$ be a proper ideal of a commutative ring $R$. Then there exists a maximal ideal $m$ of $R$ containing $I$.
\end{proposition}

\subsubsection{Chinese Remainder Theorem}\label{chineseremaindertheorem}

\begin{lemma}
Let $I_1,\dots,I_k$ be ideals of $R$ such that $I_i + I_k = (1)$ for all $i = 1,\dots,k-1$. Then $(I_1\dots I_k-1) + I_k = (1)$.
\end{lemma}

\begin{lemma}
Let $I_1,\dots,I_k$ be ideals of $R$ such that $I_i + I_j = (1)$ for all $i \neq j$. Then $I_1 \cdots I_k = I_1 \cap \cdots \cap I_k$.
\end{lemma}

\begin{theorem}
\label{crt}
Let $I_1, \dots, I_k$ be ideals of $R$ such that $I_i + I_j = (1)$ for all $i \neq j$. Then the natural homomorphism\dots
$$\varphi : R \rightarrow \frac{R}{I_1} \times \cdots \times \frac{R}{I_k}$$
is surjective and induces an isomorphism\dots
$$\tilde \varphi : \frac{R}{I_1 \cdots I_k} \xrightarrow[]{\sim} \frac{R}{I_1} \times \cdots \times \frac{R}{I_k}.$$
\end{theorem}

\begin{proof}
Argue by induction on $k$. By the former two lemmas we only need to show that the natural homomorphism\dots
$$R \rightarrow \frac{R}{I_1\cdots I_{k-1}}\times\frac{R}{I_k}$$
is surjective. By Lemma 6.2, $(I_1 \dots I_{k-1}) + I_k = (1)$; thus we are reduced to the case of two ideals.

Let then $I$, $J$ be ideals of a commutative ring $R$, such that $I + J = (1)$, and let $r_I, r_J \in R$; we have to verify that
$\exists r \in R$ such that $r \equiv r_I$ mod $I$ and $r \equiv r_J$ mod $J$. Since $I + J = (1)$, there are $a \in I$, $b \in J$
such that $a + b = 1$. Let $r = ar_J + br_I$: then\dots
$$r = ar_J + (1-a)r_I = r_I + a(r_J - r_I) \equiv r_I \textrm{ mod } I$$
as $a \in I$, and\dots
$$r = (1-b)r_J + br_I = r_J + b(r_I - r_J) \equiv r_J \textrm{ mod } J$$
as $b \in J$, as needed, and completing the proof.
\end{proof}