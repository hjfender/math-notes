\subsection{Integral Domains}

\subsubsection{Zero-divisors}\label{zerodivisors}
An element $a$ in a ring $R$ is a \emph{left-zero-divisor} if there exist elements $b \neq 0$
in $R$ for which $ab = 0$.

\begin{proposition}
\label{zerodivisormultiplication}
In a ring $R$, $a \in R$ is not a left- (resp., right-) zero-divisor if and only if left (resp., right)
multiplication by $a$ is an injective function $R \rightarrow R$.
\end{proposition}

\begin{proof}
($\Rightarrow$) Assume $a$ is not a left-zero-divisor and $ab = ac$ for $b,c \in R$. Then, by distributivity,
$$a(b-c) = ab - ac = 0,$$
and this implies $b-c = 0$ since $a$ is not a left-zero-divisor; that is, $b = c$.

($\Leftarrow$) If $a$ is a left-zero-divisor, then $\exists b \neq 0$ such that $ab = 0 = a \cdot 0$; this shows that
left-multiplication is not injective in this case.
\end{proof}

\subsubsection{Definition}\label{integraldomaindefinition}
An \emph{integral domain} is a nonzero \hyperref[commutativeringdefinition]{commutative ring} $R$ (with $1$) such that\dots
$$(\forall a,b \in R): \; \; ab =0 \Rightarrow a=0 \textrm{ or } b=0.$$

\begin{proposition}
Assume $R$ is a finite commutative ring; then $R$ is an integral domain if and only if it is a field.
\end{proposition}

\begin{proof}
($\Rightarrow$) If $a \in R$ is a non-zero-divisor, then multiplication by $a$ in $R$ is injective by \ref{zerodivisormultiplication};
hence it is surjective, as the ring is finite, by the \hyperref[pigeonholeprinciplecombinatorics]{pigeonhole principle}; hence $a$ is
a unit via \ref{unitproperties}.

($\Leftarrow$) This direction is obvious.
\end{proof}

\begin{corollary}
Let $I$ be an ideal of a commutative ring $R$. If $R/I$ is finite, then $I$ is prime if and only if it is maximal.
\end{corollary}

\subsubsection{Associates in Integral Domains}

\begin{lemma}
Let $a,b$ be nonzero elements of an integral domain $R$. Then $a$ and $b$ are associates if and only if $a = ub$, for
$u$ a unit in $R$.
\end{lemma}

\subsubsection{Prime Element}\label{primeelement}
An element $a \in R$ of an integral domain is \emph{prime} if $(a)$ is prime; that is, $a$ is not a unit and\dots
$$a | bc \Rightarrow (a|b \textrm{ or } a|c).$$

\subsubsection{Irreducible Element}\label{irreducibleelement}
An element $a \in R$ of an integral domain is \emph{irreducible} if $a$ is not a unit and\dots
$$a = bc \Rightarrow (b \textrm{ is a unit or } c \textrm{ is a unit}).$$

\begin{lemma}
Let $R$ be an integral domain, and let $a \in R$ be a nonzero prime element. Then $a$ is irreducible.
\end{lemma}

\subsubsection{Factorization}\label{factorization}
An element $r \in R$ of an integral domain has a \emph{factorization into irreducibles} if there exist irreducible elements $q_1, \dots, q_n$
such that $r = q_1 \cdots q_n$.

\subsubsection{Domain with factorization}\label{domainfactorization}
An integral domain $R$ is a \emph{domain with factorization} if every nonzero, nonunit element $r \in R$ has a factorization into irreducibles.

\begin{proposition}[Ascending Chain Condition]
\label{acc}
Let $R$ be an integral domain, and let $r$ be a nonzero, nonunit element of $R$. Assume that every ascending chain of principal ideals\dots
$$(r) \subseteq (r_1) \subseteq (r_2) \subseteq (r_1) \subseteq \cdots$$
stabilizes. Then $r$ has a factorization into irreducibles.
\end{proposition}

\subsubsection{Greatest Common Divisor}\label{gcd}
Let $R$ be an integral domain, and let $a,b \in R$. An element $d \in R$ is a \emph{greatest common divisor} (often abbreviated 'gcd') of
$a$ and $b$ if $(a,b) \subseteq (d)$ and $(d)$ is the smallest principal ideal in $R$ with this property.