\subsection{Integral Domains}

\subsubsection{Zero-divisors}\label{zerodivisors}
An element $a$ in a ring $R$ is a \emph{left-zero-divisor} if there exist elements $b \neq 0$
in $R$ for which $ab = 0$.

\begin{proposition}
\label{zerodivisormultiplication}
In a ring $R$, $a \in R$ is not a left- (resp., right-) zero-divisor if and only if left (resp., right)
multiplication by $a$ is an injective function $R \rightarrow R$.
\end{proposition}

\begin{proof}
($\Rightarrow$) Assume $a$ is not a left-zero-divisor and $ab = ac$ for $b,c \in R$. Then, by distributivity,
$$a(b-c) = ab - ac = 0,$$
and this implies $b-c = 0$ since $a$ is not a left-zero-divisor; that is, $b = c$.

($\Leftarrow$) If $a$ is a left-zero-divisor, then $\exists b \neq 0$ such that $ab = 0 = a \cdot 0$; this shows that
left-multiplication is not injective in this case.
\end{proof}

\subsubsection{Definition}\label{integraldomaindefinition}
An \emph{integral domain} is a nonzero \hyperref[commutativeringdefinition]{commutative ring} $R$ (with $1$) such that\dots
$$(\forall a,b \in R): \; \; ab =0 \Rightarrow a=0 \textrm{ or } b=0.$$

\begin{proposition}
Assume $R$ is a finite commutative ring; then $R$ is an integral domain if and only if it is a field.
\end{proposition}

\begin{proof}
($\Rightarrow$) If $a \in R$ is a non-zero-divisor, then multiplication by $a$ in $R$ is injective by \ref{zerodivisormultiplication};
hence it is surjective, as the ring is finite, by the \hyperref[pigeonholeprinciplecombinatorics]{pigeonhole principle}; hence $a$ is
a unit via \ref{unitproperties}.

($\Leftarrow$) This direction is obvious.
\end{proof}
