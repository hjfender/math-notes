\subsection{Division Rings}

\subsubsection{Units}\label{units}
An element $u$ of a ring $R$ is a \emph{left-unit} if $\exists v \in R$ such that $uv = 1$;
it is a \emph{right-unit} if $\exists v \in R$ such that $vu = 1$. \emph{Units} are two sided units.

\begin{proposition}
\label{unitproperties}
In a ring $R$:
\begin{itemize}
  \item $u$ is a left- (resp., right-) unit if and only if left- (resp., right-) multiplication by $u$
  is a surjective function $R \rightarrow R$
  \item if $u$ is a left- (resp., right-) unit, then right- (resp., left-) multiplication by $u$
  is injective; that is, $u$ is not a right- (resp., left-) zero-divisor;
  \item the inverse of a two-sided unit is unique;
  \item two-sided units form a group under multiplication.
\end{itemize}
\end{proposition}

\subsubsection{Definition}\label{divisionringdefinition}
A \emph{division ring} is a ring in which every nonzero element is a two-sided unit.