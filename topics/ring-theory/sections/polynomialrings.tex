\subsection{Polynomial Rings}\label{polynomialrings}

\subsubsection{Polynomials}\label{polynomials}
Let $R$ be a ring. A \emph{polynomial} $f(x)$ in the \emph{indeterminate} $x$ and with \emph{coefficients} in $R$
is a finite linear combination of nonnegative 'powers' of $x$ with coefficients in $R$:
$$f(x) = \sum_{i \geq 0} a_i x^i = a_0 + a_qx+a_2x^2 + \cdots,$$
where all $a_i$ are elements of $R$ and we require $a_i = 0$ for $i \gg 0.$\newline

\noindent Two polynomials are taken to be equal if\dots
$$\sum_{i \geq 0} a_i x^i = \sum_{i \geq 0} b_i x^i \Leftrightarrow (\forall i \geq 0): \; \; a_i = b_i.$$

\noindent NOTE: a polynomial \emph{actually is} an element of the infinite direct sum of the group $\langle R,+ \rangle$.\newline

Operations on polynomials are defined as follows: if\dots
$$f(x) = \sum_{i \geq 0} a_i x^i \textrm{ and } g(x) = \sum_{i \geq 0} b_i x^i$$
then\dots
$$f(x) + f(x) := \sum_{i \geq 0} (a_i + b_i) x^i$$
and\dots
$$f(x) \cdot f(x) := \sum_{k \geq 0}\sum_{i+j = k} a_ib_ix^{i+j}.$$

\subsubsection{Universal Property}\label{universalpropertyofpolynomialrings}
Let $\mathcal{R}_A$ be the category of \hyperref[objectsunder]{commutative rings under a set $A$} so that\dots
\begin{itemize}
  \item Objects: $(j, R)$ such that $j : A \rightarrow R$ 
  \item Arrows: $(j_1, R_1) \rightarrow (j_2, R_2)$ representing\dots
  \begin{figure}[H]
	  \centering
	  \begin{tikzcd}
	A \arrow[d, "j_1"] \arrow[dr, "j_2'" above] & \\
	R_1 \arrow[r, "\varphi"] & R_2
\end{tikzcd}
  \end{figure}
\end{itemize}

\begin{proposition}
$(i, \mathbb{Z}[x_1,\cdots,x_n])$ is initial in $\mathcal{R}_A$.
\end{proposition}

\begin{proof}
Let $(j,R)$ be an arbitrary object of $\mathbb{R}_A$; we have to show that there is a unique morphism $(i, \mathbb{Z}[x_1, \cdots, x_n]) \rightarrow (j, R)$.

The key point is that the requirements posed on $\varphi$ force its definition. The postulated commutativity of the diagram means that $\varphi(x_k) = j(a_k)$
for $k = 1, \cdots, n$. Then, since $\varphi$ must be a ring homomorphism, necessarily\dots
\begin{align*}
\varphi(\sum m_{i_1\dots i_n}x_1^{i_1}\cdots x_n^{i_n}) &= \sum \varphi(m_{i_1\dots i_n})\varphi(x_1)^{i_1}\cdots \varphi(x_n)^{i_n} \\
														&= \sum \iota(m_{i_1\dots i_n})j(x_1)^{i_1}\cdots j(x_n)^{i_n},
\end{align*}
where $\iota: \mathbb{Z} \rightarrow R$ is the unique ring homomorphism (as $\mathbb{Z}$ is initial in Ring).

Thus, if $\varphi$ exists, then it is unique. On the other hand, the formula we just obtained clearly preserves the operations and sends $1$ to $1$,
so it does define a ring homomorphism, concluding the proof.
\end{proof}

\subsubsubsection{Evaluation Map and Polynomial Functions}\label{polynomialevaluationmap}
Let $\alpha: R \rightarrow S$ be a fixed ring homomorphism, and $s \in S$ be an element commuting with $\alpha(r)$ for all $r \in R$.
Then there is a unique ring homomorphism $\overline{\alpha}: R[x] \rightarrow S$ extending $\alpha$ and sending $x$ to $s$.\newline

\noindent This we get an '\emph{evaluation map}' over commutative rings\dots
$$f(x) = \sum_{i \geq 0} a_i x^i \textrm{ and } r \in R \Rightarrow f(r) = \sum_{i \geq 0} a_i r^i \in R.$$
This may be viewed as $\overline{\alpha}(f(x))$, where $\overline{\alpha}$ is obtained with $id_R : R \rightarrow R$ and $s = r$.\newline

\noindent Thus, every polynomial $f(x)$ determines a \emph{polynomial function}\label{polynomialfunction} $f : R \rightarrow R$ defined by $r \mapsto f(r)$.