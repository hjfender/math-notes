\subsection{Polynomial Rings}\label{polynomialrings}

\subsubsection{Polynomials}\label{polynomials}
Let $R$ be a ring. A \emph{polynomial} $f(x)$ in the \emph{indeterminate} $x$ and with \emph{coefficients} in $R$
is a finite linear combination of nonnegative 'powers' of $x$ with coefficients in $R$:
$$f(x) = \sum_{i \geq 0} a_i x^i = a_0 + a_qx+a_2x^2 + \cdots,$$
where all $a_i$ are elements of $R$ and we require $a_i = 0$ for $i \gg 0.$\newline

\noindent Two polynomials are taken to be equal if\dots
$$\sum_{i \geq 0} a_i x^i = \sum_{i \geq 0} b_i x^i \Leftrightarrow (\forall i \geq 0): \; \; a_i = b_i.$$

\noindent NOTE: a polynomial \emph{actually is} an element of the infinite direct sum of the group $\langle R,+ \rangle$.\newline

Operations on polynomials are defined as follows: if\dots
$$f(x) = \sum_{i \geq 0} a_i x^i \textrm{ and } g(x) = \sum_{i \geq 0} b_i x^i$$
then\dots
$$f(x) + f(x) := \sum_{i \geq 0} (a_i + b_i) x^i$$
and\dots
$$f(x) \cdot f(x) := \sum_{k \geq 0}\sum_{i+j = k} a_ib_ix^{i+j}.$$

\subsubsubsection{Monic}
A \emph{monic} polynomial is a polynomial\dots
$$f(x) = x^d + a_{d-1]}x^{d-1} + \cdots + a_1x +a_0$$
where the leading coefficient is $1$.

\begin{lemma}
\label{uniquepolynomialdivision}
Let $f(x)$ be a monic polynomial, and assume\dots
$$f(x)q_1(x) + r_1(x) = f(x)q_2(x) + r_2(x)$$
with both $r_1(x)$ and $r_2(x)$ polynomials of degree $<$ deg$f(x)$. Then $q_1(x) = q_2(x)$
and $r_1(x) = r_2(x).$
\end{lemma}

\begin{proof}
Indeed, we have\dots
$$f(x)(q_1(x)-q_2(x))=r_2(x)-r_1(x);$$
if $r_2(x) \neq r_1(x)$, then $r_2(x) - r_1(x)$ has degree $<$ deg$f(x)$, while $f(x)(q_1(x)-q_2(x))$ has degree $\geq f(x)$,
giving a contradiction. Therefore $r_1(x) = r_2(x)$, and $q_1(x) = q_2(x)$ follows right away since monic polynomials are
non-zero-divisors.
\end{proof}

\subsubsection{Universal Property}\label{universalpropertyofpolynomialrings}
Let $\mathcal{R}_A$ be the category of \hyperref[objectsunder]{commutative rings under a set $A$} so that\dots
\begin{itemize}
  \item Objects: $(j, R)$ such that $j : A \rightarrow R$ 
  \item Arrows: $(j_1, R_1) \rightarrow (j_2, R_2)$ representing\dots
  \begin{figure}[H]
	  \centering
	  \begin{tikzcd}
	A \arrow[d, "j_1"] \arrow[dr, "j_2'" above] & \\
	R_1 \arrow[r, "\varphi"] & R_2
\end{tikzcd}
  \end{figure}
\end{itemize}

\begin{proposition}
$(i, \mathbb{Z}[x_1,\cdots,x_n])$ is initial in $\mathcal{R}_A$.
\end{proposition}

\begin{proof}
Let $(j,R)$ be an arbitrary object of $\mathbb{R}_A$; we have to show that there is a unique morphism $(i, \mathbb{Z}[x_1, \cdots, x_n]) \rightarrow (j, R)$.

The key point is that the requirements posed on $\varphi$ force its definition. The postulated commutativity of the diagram means that $\varphi(x_k) = j(a_k)$
for $k = 1, \cdots, n$. Then, since $\varphi$ must be a ring homomorphism, necessarily\dots
\begin{align*}
\varphi(\sum m_{i_1\dots i_n}x_1^{i_1}\cdots x_n^{i_n}) &= \sum \varphi(m_{i_1\dots i_n})\varphi(x_1)^{i_1}\cdots \varphi(x_n)^{i_n} \\
														&= \sum \iota(m_{i_1\dots i_n})j(x_1)^{i_1}\cdots j(x_n)^{i_n},
\end{align*}
where $\iota: \mathbb{Z} \rightarrow R$ is the unique ring homomorphism (as $\mathbb{Z}$ is initial in Ring).

Thus, if $\varphi$ exists, then it is unique. On the other hand, the formula we just obtained clearly preserves the operations and sends $1$ to $1$,
so it does define a ring homomorphism, concluding the proof.
\end{proof}

\subsubsubsection{Evaluation Map and Polynomial Functions}\label{polynomialevaluationmap}
Let $\alpha: R \rightarrow S$ be a fixed ring homomorphism, and $s \in S$ be an element commuting with $\alpha(r)$ for all $r \in R$.
Then there is a unique ring homomorphism $\overline{\alpha}: R[x] \rightarrow S$ extending $\alpha$ and sending $x$ to $s$.\newline

\noindent This we get an '\emph{evaluation map}' over commutative rings\dots
$$f(x) = \sum_{i \geq 0} a_i x^i \textrm{ and } r \in R \Rightarrow f(r) = \sum_{i \geq 0} a_i r^i \in R.$$
This may be viewed as $\overline{\alpha}(f(x))$, where $\overline{\alpha}$ is obtained with $id_R : R \rightarrow R$ and $s = r$.\newline

\noindent Thus, every polynomial $f(x)$ determines a \emph{polynomial function}\label{polynomialfunction} $f : R \rightarrow R$ defined by $r \mapsto f(r)$.

\subsubsection{Quotients of Polynomial Rings}\label{quotientsofpolynomialrings}
Assume that $R$ is a commutative ring. Via \ref{uniquepolynomialdivision}, if $f(x)$ is monic, then for every $g(x) \in R[x]$ there exists a unique polynomial $r(x)$
of degree $<$ deg$f(x)$ and such that\dots
$$g(x) + (f(x)) = r(x) + (f(x))$$
as cosets of the principal ideal $(f(x))$ in $R[x]$.

\begin{proposition}
Let $R$ be a commutative ring, and let $f(x) \in R[x]$ be a monic polynomial of degree $d$. Then the function\dots
$$\varphi : R[x] \rightarrow R^{\oplus d}$$
defined by sending $g(x) \in R[x]$ to the remainder of the division of $g(x)$ by $f(x)$ induces an isomorphism of abelian
groups\dots
$$\frac{F[x]}{(f(x))} \cong R^{\oplus d}$$
\end{proposition}

\begin{proof}
The given function $\varphi$ is well-defined by \ref{uniquepolynomialdivision}, and it is surjective since it has a right inverse\dots
$$\psi((r_0,r_1,\dots,r_{d-1}))=r_0 + r_1x + \cdots + r_{d-1}x^{d-1}.$$

The function $\varphi$ is a homomorphism of abelian groups. Indeed, if\dots
$$g_1(x) = f(x)q_1(x) + r_1(x) \textrm{ and } g_2 = f(x)q_2(x) + r_2(x)$$
with deg $r_1(x) < d$, deg $r_2(x) < d$, then\dots
$$g_1(x)+g_2(x) = f(x)(q_1(x) + q_2(x)) + (r_1(x) + r_2(x))$$
and deg $(r_1(x) + r_2(x)) < d$: this implies via \ref{uniquepolynomialdivision}\dots
$$\varphi(g_1(x) + g_2(x)) = r_1(x) + r_2(x) = \varphi(g_1(x)) + \varphi(g_2(x)).$$

By the first isomorphism theorem for abelian groups, then, $\varphi$ induces an isomorphism\dots
$$\frac{R[x]}{\textrm{ker}\varphi} \cong R^{\oplus d}.$$
On the other hand, $\varphi(g(x)) = 0$ if and only if $g(x) = f(x)q(x)$ for some $q(x) \in R[x]$,
that is, if and only if $g(x)$ is in the principal ideal generated by $f(x)$. This shows ker$\varphi = (f(x))$,
concluding the proof.
\end{proof}