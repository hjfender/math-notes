\subsection{Polynomial Rings}\label{polynomialrings}

\subsubsection{Polynomials}\label{polynomials}
Let $R$ be a ring. A \emph{polynomial} $f(x)$ in the \emph{indeterminate} $x$ and with \emph{coefficients} in $R$
is a finite linear combination of nonnegative 'powers' of $x$ with coefficients in $R$:
$$f(x) = \sum_{i \geq 0} a_i x^i = a_0 + a_qx+a_2x^2 + \cdots,$$
where all $a_i$ are elements of $R$ and we require $a_i = 0$ for $i \gg 0.$\newline

\noindent Two polynomials are taken to be equal if\dots
$$\sum_{i \geq 0} a_i x^i = \sum_{i \geq 0} b_i x^i \Leftrightarrow (\forall i \geq 0): \; \; a_i = b_i.$$

\noindent NOTE: a polynomial \emph{actually is} an element of the infinite direct sum of the group $\langle R,+ \rangle$.\newline

Operations on polynomials are defined as follows: if\dots
$$f(x) = \sum_{i \geq 0} a_i x^i \textrm{ and } g(x) = \sum_{i \geq 0} b_i x^i$$
then\dots
$$f(x) + f(x) := \sum_{i \geq 0} (a_i + b_i) x^i$$
and\dots
$$f(x) \cdot f(x) := \sum_{k \geq 0}\sum_{i+j = k} a_ib_ix^{i+j}.$$

\subsubsubsection{Monic}
A \emph{monic} polynomial is a polynomial\dots
$$f(x) = x^d + a_{d-1]}x^{d-1} + \cdots + a_1x +a_0$$
where the leading coefficient is $1$.

\begin{lemma}
\label{uniquepolynomialdivision}
Let $f(x)$ be a monic polynomial, and assume\dots
$$f(x)q_1(x) + r_1(x) = f(x)q_2(x) + r_2(x)$$
with both $r_1(x)$ and $r_2(x)$ polynomials of degree $<$ deg$f(x)$. Then $q_1(x) = q_2(x)$
and $r_1(x) = r_2(x).$
\end{lemma}

\begin{proof}
Indeed, we have\dots
$$f(x)(q_1(x)-q_2(x))=r_2(x)-r_1(x);$$
if $r_2(x) \neq r_1(x)$, then $r_2(x) - r_1(x)$ has degree $<$ deg$f(x)$, while $f(x)(q_1(x)-q_2(x))$ has degree $\geq f(x)$,
giving a contradiction. Therefore $r_1(x) = r_2(x)$, and $q_1(x) = q_2(x)$ follows right away since monic polynomials are
non-zero-divisors.
\end{proof}

\subsubsection{Universal Property}\label{universalpropertyofpolynomialrings}
Let $\mathcal{R}_A$ be the category of \hyperref[objectsunder]{commutative rings under a set $A$} so that\dots
\begin{itemize}
  \item Objects: $(j, R)$ such that $j : A \rightarrow R$ 
  \item Arrows: $(j_1, R_1) \rightarrow (j_2, R_2)$ representing\dots
  \begin{figure}[H]
	  \centering
	  \begin{tikzcd}
	A \arrow[d, "j_1"] \arrow[dr, "j_2'" above] & \\
	R_1 \arrow[r, "\varphi"] & R_2
\end{tikzcd}
  \end{figure}
\end{itemize}

\begin{proposition}
$(i, \mathbb{Z}[x_1,\cdots,x_n])$ is initial in $\mathcal{R}_A$.
\end{proposition}

\begin{proof}
Let $(j,R)$ be an arbitrary object of $\mathbb{R}_A$; we have to show that there is a unique morphism $(i, \mathbb{Z}[x_1, \cdots, x_n]) \rightarrow (j, R)$.

The key point is that the requirements posed on $\varphi$ force its definition. The postulated commutativity of the diagram means that $\varphi(x_k) = j(a_k)$
for $k = 1, \cdots, n$. Then, since $\varphi$ must be a ring homomorphism, necessarily\dots
\begin{align*}
\varphi(\sum m_{i_1\dots i_n}x_1^{i_1}\cdots x_n^{i_n}) &= \sum \varphi(m_{i_1\dots i_n})\varphi(x_1)^{i_1}\cdots \varphi(x_n)^{i_n} \\
														&= \sum \iota(m_{i_1\dots i_n})j(x_1)^{i_1}\cdots j(x_n)^{i_n},
\end{align*}
where $\iota: \mathbb{Z} \rightarrow R$ is the unique ring homomorphism (as $\mathbb{Z}$ is initial in Ring).

Thus, if $\varphi$ exists, then it is unique. On the other hand, the formula we just obtained clearly preserves the operations and sends $1$ to $1$,
so it does define a ring homomorphism, concluding the proof.
\end{proof}

\subsubsubsection{Evaluation Map and Polynomial Functions}\label{polynomialevaluationmap}
Let $\alpha: R \rightarrow S$ be a fixed ring homomorphism, and $s \in S$ be an element commuting with $\alpha(r)$ for all $r \in R$.
Then there is a unique ring homomorphism $\overline{\alpha}: R[x] \rightarrow S$ extending $\alpha$ and sending $x$ to $s$.\newline

\noindent This we get an '\emph{evaluation map}' over commutative rings\dots
$$f(x) = \sum_{i \geq 0} a_i x^i \textrm{ and } r \in R \Rightarrow f(r) = \sum_{i \geq 0} a_i r^i \in R.$$
This may be viewed as $\overline{\alpha}(f(x))$, where $\overline{\alpha}$ is obtained with $id_R : R \rightarrow R$ and $s = r$.\newline

\noindent Thus, every polynomial $f(x)$ determines a \emph{polynomial function}\label{polynomialfunction} $f : R \rightarrow R$ defined by $r \mapsto f(r)$.

\subsubsection{Quotients of Polynomial Rings}\label{quotientsofpolynomialrings}
Assume that $R$ is a commutative ring. Via \ref{uniquepolynomialdivision}, if $f(x)$ is monic, then for every $g(x) \in R[x]$ there exists a unique polynomial $r(x)$
of degree $<$ deg$f(x)$ and such that\dots
$$g(x) + (f(x)) = r(x) + (f(x))$$
as cosets of the principal ideal $(f(x))$ in $R[x]$.

\begin{proposition}
Let $R$ be a commutative ring, and let $f(x) \in R[x]$ be a monic polynomial of degree $d$. Then the function\dots
$$\varphi : R[x] \rightarrow R^{\oplus d}$$
defined by sending $g(x) \in R[x]$ to the remainder of the division of $g(x)$ by $f(x)$ induces an isomorphism of abelian
groups\dots
$$\frac{R[x]}{(f(x))} \cong R^{\oplus d}$$
\end{proposition}

\begin{proof}
The given function $\varphi$ is well-defined by \ref{uniquepolynomialdivision}, and it is surjective since it has a right inverse\dots
$$\psi((r_0,r_1,\dots,r_{d-1}))=r_0 + r_1x + \cdots + r_{d-1}x^{d-1}.$$

The function $\varphi$ is a homomorphism of abelian groups. Indeed, if\dots
$$g_1(x) = f(x)q_1(x) + r_1(x) \textrm{ and } g_2 = f(x)q_2(x) + r_2(x)$$
with deg $r_1(x) < d$, deg $r_2(x) < d$, then\dots
$$g_1(x)+g_2(x) = f(x)(q_1(x) + q_2(x)) + (r_1(x) + r_2(x))$$
and deg $(r_1(x) + r_2(x)) < d$: this implies via \ref{uniquepolynomialdivision}\dots
$$\varphi(g_1(x) + g_2(x)) = r_1(x) + r_2(x) = \varphi(g_1(x)) + \varphi(g_2(x)).$$

By the first isomorphism theorem for abelian groups, then, $\varphi$ induces an isomorphism\dots
$$\frac{R[x]}{\textrm{ker}\varphi} \cong R^{\oplus d}.$$
On the other hand, $\varphi(g(x)) = 0$ if and only if $g(x) = f(x)q(x)$ for some $q(x) \in R[x]$,
that is, if and only if $g(x)$ is in the principal ideal generated by $f(x)$. This shows ker$\varphi = (f(x))$,
concluding the proof.
\end{proof}

\subsubsection{Ideals in Polynomial Rings}

\begin{lemma}
Let $R$ be a ring, and let $I$ be an ideal of $R$. Then\dots
$$\frac{R[x]}{IR[x]} \cong \frac{R}{I}[x].$$
\end{lemma}

\begin{corollary}
If $I$ is a prime ideal of $R$, then $IR[x]$ is prime in $R[x]$.
\end{corollary}

\subsubsection{Primitivity}\label{primitivepolynomial}
Let $f \in R[x]$ be a polynomial. Then\dots
\begin{itemize}
  \item $f$ is \emph{very primitive} if for all prime ideals $p$ of $R$, $f \not \in pR[x]$.
  \item $f$ is \emph{primitive} if for all principal prime ideals $p$ of $R$, $f \not \in pR[x]$.
\end{itemize}

\begin{lemma}
Let $R$ be a commutative ring. Then for $f,g \in R[x]$\dots
$$fg \textrm{ is primitive} \Leftrightarrow \textrm{ both } f \textrm{ and } g \textrm{ are primitive}.$$
\end{lemma}

\begin{lemma}
Let $R$ be a commutative ring and $f = a_0 + a_1x + \cdots + a_dx^d \in R[x]$.
\begin{itemize}
  \item $f$ is very primitive if and only if $(a_0,\dots,a_d) = (1)$.
  \item If $R$ is a UFD, then $f$ is primitive if and only if gcd$(a_0,\dots,a_d) = 1$.
\end{itemize}
\end{lemma}

\subsubsection{Content}\label{content}
Let $R$ be a UFD. The \emph{content} of a nonzero polynomial $f \in R[x]$, denoted cont$_f$, is the gcd of its coefficients.

\begin{lemma}
Let $R$ be a UFD, and let $f \in R[x]$. Then\dots
\begin{itemize}
  \item $(f) = (\textrm{cont}_f)(\underline{f})$, where $\underline{f}$ is primitive;
  \item if $(f) = (c)(g)$, with $c \in R$ and $g$ is primitive, then $(c) = (\textrm{cont}_f)$.
\end{itemize}
\end{lemma}

\begin{proposition}[Gauss's lemma]
\label{gausslemma}
Let $R$ be a UFD, and let $f,g \in R[x]$. Then\dots
$$(cont_{fg}) = (cont_f)(cont_g).$$
\end{proposition}

\begin{corollary}
Let $R$ be a UFD, and let $f,g \in R[x]$. Assume $(f) \subseteq (g)$. Then $($cont$_f) \subseteq ($cont$_g)$.
\end{corollary}

\subsubsection{Field of rational functions}\label{fieldofrationalfunctions}
The field of \emph{rational functions} with coefficients in $R$ is the field of fractions of the ring $R[x]$. This field is denoted $R(x)$.

\subsubsection{Factorization in polynomial rings}

\begin{lemma}
Let $R$ be a UFD, and let $K = K(R)$ be its field of fractions. For nonzero $f,g \in R[x]$, denote by $(f), (g)$ the principal ideals $fR[x], gR[x]$ in $R[x]$,
and denote by $(f)_K$, $(g)_K$ the principal ideals $fK[x], gK[x]$ in $K[x]$. Assume\dots
\begin{itemize}
  \item $($cont$_g) \subseteq ($cont$_f) and$
  \item $(g)_K \subseteq (f)_K$.
\end{itemize}
Then $(g) \subseteq (f)$.
\end{lemma}

\begin{proposition}
Let $R$ be a UFD, and let $K$ be its field of fractions. Let $f \in R[x]$ be a nonconstant, irreducible polynomial. Then $f$ is irreducible as
an element of $K[x]$.
\end{proposition}

\begin{corollary}
Let $R$ be a UFD and $K$ the field of fractions of $R$. Let $f \in R[x]$ be a nonconstant polynomial. Then $f$ is irreducible in $R[x]$ if and only if
it is irreducible in $K[x]$ and primitive.
\end{corollary}

\noindent The preceding results all lead to the following result\dots

\begin{theorem}
Let $R$ be a UFD; then $R[x]$ is a UFD.
\end{theorem}

\subsubsection{Irreducibility in polynomial rings}

\begin{lemma}
Let $R$ be an integral domain, and let $f \in R[x]$ be a polynomial of degree $n$. Then the number of roots of $f$, counted with multiplicity, is at most $n$.
\end{lemma}

\begin{proof}
Replace $R$ by its field of fractions $K$. And observe $K$ is a UFD.
\end{proof}

\begin{corollary}
Let $R$ be an infinite integral domain, and let $f,g \in R[x]$ be polynomials. Then $f = g$ if and only if the evaluation functions $r \mapsto f(r)$,
$r \mapsto g(r)$ agree.
\end{corollary}

\begin{proposition}
Let $k$ be a field. A polynomial $f \in k[x]$ of degree $2$ or $3$ is irreducible if and only if it has no roots.
\end{proposition}

\begin{proposition}
Let $R$ be a UFD, and let $K$ be its field of fractions. Let\dots
$$f(x) = a_0 + a_1 x + \cdots + a_n x^n \in R[x],$$
and let $c = \frac{p}{q} \in K$ be a root of $f$, with $p,q \in R$, gcd$(p,q) = 1$. Then $p | a_0$ and $q | a_n$ in $R$.
\end{proposition}

\subsubsection{Eisenstein's Criterion}

\begin{proposition}
\label{eisenstein}
Let $R$ be a commutative ring, and let $\textbf{p}$ be a prime ideal of $R$. Let\dots
$$f = a_0 + a_1 x + \cdots + a_n x^n \in R[x]$$
be a polynomial, and assume that\dots
\begin{itemize}
  \item $a_n \not \in \textbf{p}$;
  \item $a_i \in \textbf{p}$ for $i = 0,\dots,n-1$;
  \item $a_0 \not \in \textbf{p}^2$
\end{itemize}
Then $f$ is not the product of polynomials of degree $<n$ in $R[x]$.
\end{proposition}

\subsubsection{Cyclotomic Polynomials}\label{cyclotomic}
\emph{Cyclotomic polynomials} are polynomials\dots
$$f(x) = 1 + x + x^2 + \cdots + x^{p-1} \in \mathbb{Z}[x].$$
for a prime integer $p$.

\begin{proposition}
The cyclotomic polynomial $f$ above is irreducible.
\end{proposition}

\begin{proof}
Observe $f(x)$ is irreducible $\Leftrightarrow$ $f(x+1)$ is irreducible. Also observe $f(x) = \frac{(x^p-1)}{(x-1)}$. Then\dots
$$f(x+1) = \frac{(x+1)^p - 1}{(x+1)-1} = x^{p-1} + {p \choose p-1}x^{p-2} + \cdots + {p \choose 3}x^{2} + {p \choose 2}x + {p \choose 1};$$
Eisenstein's criterion proves that this is irreducible, since ${p \choose 1} = p$ and $p | {p \choose k}$ for $k = 1. \dots, p-1$.
\end{proof}