\subsection{Definition}

A \emph{metric space} $\langle X,d \rangle$ is a set $X$ together with a \emph{metric}\label{metric} $d : X \times X \rightarrow \mathbb{R}$ satisfying\dots
\begin{enumerate}
  \item $d(x,y) \geq 0$ for all $x,y \in X$ and $d(x,y) = 0$ if and only if $x = y$.
  \item $d(x,y) = d(y,x)$ for all $x,y \in X$.
  \item (\emph{The Triangle Inequality}\label{triangleinequality}): $d(x,y) + d(y,z) \geq d(x,z)$ for all $x,y,z \in X.$
\end{enumerate}

\subsubsection{Open Ball}\label{metricopenball}
The \emph{open ball} of radius $\varepsilon > 0$ centered at a point $x$ in a metric space $\langle X,d \rangle$ is given by\dots
$$B_{\varepsilon}(x) = \{ y \in X | d(x,y) < \varepsilon\}.$$\newline

\noindent Open balls form the basis for a topology on $X$ called the \emph{metric topology}.\newline

\noindent A set $U$ of a metric space $(X,d)$ is \emph{open} if for any $u \in U$, there is $\varepsilon > 0$ so that $B_{\varepsilon}(x) \subseteq U$.

\subsubsection{Bounded}\label{bounded}
A subset $A$ of a metric space $(X,d)$ is \emph{bounded} if there is some number $M$ such that\dots
$$d(a_1,a_2) \leq M$$
for every pair $a_1,a_2$ of points of $A$.\newline

\noindent If $A$ is bounded and nonempty, the \emph{diameter} of $A$ is defined to be the numbed\dots
$$\textrm{diam } A = \sup \{ d(a_1,a_2) | a_1,a_2 \in A \}.$$

\subsubsection{Continuity}\label{metriccontinuity}
Suppose $\langle X,d_X \rangle$ and $\langle Y,d_Y \rangle$ are two metric spaces and $f: X \rightarrow Y$ is a function. Then $f$ is \emph{continuous at} $x\in X$ if for any $\epsilon > 0$ , there is a $\delta > 0$
so that $B_{\delta}(x) \subset f^{-1}(B_{\varepsilon}(f(x)))$.\newline

\noindent The function $f$ is \emph{continuous} if it is continous at $x$ for all $x \in X$.

\begin{theorem}
A function $f:X \rightarrow Y$ between metric spaces $\langle X,d \rangle$ and $\langle Y,d \rangle$ is continuous if and only if for any open subset $V$ of $Y$, the subset $f^{-1}(V)$ is open in $X$.
\end{theorem}