\subsection{Convergence}\label{metricconvergence}
\noindent We can use the sequence lemma to prove the following\dots
\begin{theorem}
Let $f : X \rightarrow Y$. If the function $f$ is continuous, then for every convergent sequence $x_n \rightarrow x$ in $X$, the sequence
$f(x_n)$ converges to $f(x)$. The converse holds if $X$ is metrizable.
\end{theorem}
\noindent As in the sequence lemma, we only need to assume the first countability axiom to prove the converse.

\subsubsection{Uniform Convergence}\label{uniformconvergence}
Let $f_n : X \rightarrow Y$ be a sequence of functions form the set $X$ to the metric space $Y$. Let
$d$ be the metric for $Y$. We say that the sequence $(f_n)$ \emph{converges uniformly} to the function $f: X \rightarrow Y$
if given $\varepsilon > 0$, there exists an integer $N$ such that\dots
$$d(f_n(x), f(x)) < \varepsilon$$
for all $n > N$ and all $x$ in $X$.

\begin{theorem}[Uniform Limit Theorem]
Let $f_n : X \rightarrow Y$ be a sequence of continuous functions from the topological space $X$ to the metric space $Y$.
If $(f_n)$ converges uniformly to $f$, then $f$ is continuous.
\end{theorem}

\begin{proof}
Let $V$ be open in $Y$; let $x_0$ be a point of $f^{-1}(V).$ We wish to find a neighborhood $U$ of $x_0$ such that $f(U) \subseteq V$.

Let $y_0 = f(x_0)$. First choose $\varepsilon$ so that $B_{\varepsilon}(y_0)$ is contained in $V$. Then using uniform convergence,
choose $N$ so that for all $n \geq N$ and all $x \in X$,
$$d(f_n(x),f(x)) < \varepsilon / 3.$$
Finally, using continuity fo $f_N$, choose a neighborhood $U$ of $x_0$ such that $f_N$ carries $U$ into the $\varepsilon / 3$ ball
in $Y$ centered at $f_N(x_0)$.

We claim that $f$ carries $U$ into $B(y_0,\varepsilon)$ and hence into $V$, as desired. For this purpose, note that if $x \in U$, then\dots
$$d(f(x), f_N(x)) < \varepsilon/3 \; \textrm{(by choice of } N \textrm{)},$$
$$d(f_N(x), f_N(x_0)) < \varepsilon/3 \; \textrm{(by choice of } U \textrm{)},$$
$$d(f_N(x_0), f(x_0)) < \varepsilon/3 \; \textrm{(by choice of } N \textrm{)}.$$

Adding and using the triangle inequality, we see that $d(f(x),f(x_0)) < \varepsilon$, as desired.
\end{proof}

\subsubsection{Lebesgue's Lemma}

\subsubsubsection{Diameter}\label{diameter}
The \emph{diameter} of a subset $A$ of a metric space $X$ is defined by diam $A = \sup\{d(x,y) | x,y \in A \}.$

\begin{lemma}[Lebesgue's Lemma]
\label{lebesguelemma}
Let $X$ be a compact metric space and $\{ U_i | i \in J \}$ an open cover. Then there is a real number
$\delta > 0$ (\emph{the Lebesgue number}\label{lebesguenumber}) such that any subset of $X$ of diameter less than $\delta$
is contained in some $U_i$.
\end{lemma}

\begin{proof}
Define the continuous function $d(-,A): X \rightarrow \mathbb{R}$ by $d(x, A) = \inf\{d(x,a) | a \in A \}$. In addition,
if $A$ is closed, then $d(x,A) > 0$ for $x \not \in A$. Fiven an open cover $\{ U_i | i \in J \}$ of the compact space $X$,
there is a finite subcover $\{U_{i_1},\dots, U_{i_n} \}$. Define $\varphi_j(x) = d(x,X \setminus U_{i_j})$ for $j = 1,2,\dots, n$
and let $\varphi(x) = \max\{\varphi_1(x), \dots, \varphi_n(x) \}$. Since each $x \in X$ lies in some $U_{i_j}$, $\varphi(x) \geq \varphi_j(x) > 0$.
Furthermore, $\varphi$ is continuous so $\varphi(X) \subseteq \mathbb{R}$ is compact, and $0 \not \in \varphi(X)$. Let $\delta = \min \{ \varphi(x) | x \in X \} > 0$.
FOr any $x \in X$, consider $B(x, \delta) \subseteq X$. We know $\varphi(x) = \varphi_j(x)$ for some $j$. For that $j$, $d(x,X \setminus U_{i_j}) \geq \delta$, which implies
$B(x, \delta) \subseteq U_{i_j}$.
\end{proof}