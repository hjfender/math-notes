\subsection{Euler Characteristic}\label{eulercharacteristic}

\begin{proposition}
Let\dots
$$0 \longrightarrow U \longrightarrow V \longrightarrow W \longrightarrow 0$$
be a short exact sequence of finite-dimensional vector spaces. Then\dots
$$\textrm{dim}(V) = \textrm{dim}(U) + \textrm{dim}(W).$$
Or equivalently\dots
$$\textrm{dim}(V / U) = \textrm{dim}(V) - \textrm{dim}(U).$$
\end{proposition}

\noindent Consider a complex of finite-dimensional vector spaces and linear maps\dots
$$V_{\bullet}: 0 \longrightarrow V_N \xrightarrow[]{\alpha_N} V_{N-1} \xrightarrow[]{\alpha_{N-1}} \cdots \xrightarrow[]{\alpha_2} V_{2} \xrightarrow[]{\alpha_1} V_{0} \longrightarrow 0.$$
The \emph{Euler characteristic} of $V_{\bullet}$ is the integer\dots
$$\chi(V_{\bullet}) := \sum_{i}(-1)^{i}\textrm{dim}(V_i).$$

\begin{proposition}
With notation as above,
$$\chi(V_{\bullet}) = \sum_{i=0}^{N}(-1)^{i}\textrm{dim}(H_i(V_{\bullet})).$$
In particular, if $V_{\bullet}$ is exact, then $\chi(V_{\bullet}) = 0$.
\end{proposition}

\begin{proof}
Use induction on the length of the complex $V_{\bullet}$.
\end{proof}

\subsubsection{Grothendieck Group}\label{grothendieckgroup}
Consider the category $k$-Vect$^{f}$ of finite-dimensional $k$-vector spaces. Each object $V$ of $k$-Vect$^f$ determines an isomorphism class $[V]$.
Let $F(k\textrm{-Vect}^f)$ be the free abelian group on the set of these isomorphism classes; further, let $E$ be the subgroup generated by the elements\dots
$$[V] -[U] - [W]$$
for all short exact sequences\dots
$$0 \longrightarrow U \longrightarrow V \longrightarrow W \longrightarrow 0$$
in $k$-Vect$^f$. The quotient group\dots
$$K(k\textrm{-Vect}^f) := \frac{F(k\textrm{-Vect}^f)}{E}$$
is called the \emph{Grothendieck group} of the category $k$-Vect$^f$. The element determined by $V$ in
the Grothendieck group is still denoted $[V]$.\newline

\noindent A \emph{Grothendieck group} may be defined for any category admitting a notion of exact sequence. \newline

\noindent Every complex $V_{\bullet}$ determines an element in $K(k\textrm{-Vect}^f)$, namely\dots
$$\chi_K(V_{\bullet}) := \sum_i(-1)^i[V_i] \in K(k\textrm{-Vect}^f).$$

\begin{proposition}
With notation as above, we have the following:
\begin{itemize}
  \item $\chi_K$ 'is an Euler characteristic', in the sense that it satisfies the formula given
  in the previous proposition\dots
  $$\chi(V_{\bullet}) = \sum_{i=0}(-1)^{i}[H_i(V_{\bullet})].$$
  \item $\chi_K$ is a 'universal Euler characteristic', in the following sense. Let $G$ be an abelian group, and let $\delta$ be a function associating an element of $G$ to
  each finite-dimensional vector space, such that $\delta(V) = \delta(V')$ if $V \cong V'$ and $\delta(V/U) = \delta(V) - \delta(U).$ For $V_{\bullet}$
  a complex, define\dots
  $$\chi(V_{\bullet}) = \sum_{i=0}(-1)^{i}\delta(V_i).$$
  Then $\delta$ induces a (unique) group homomorphism\dots
  $$K(k\textrm{-Vect}^f) \rightarrow G$$
  mapping $\chi_K(V_{\bullet})$ to $\chi_G(V_{\bullet})$.
  \item In particular, $\delta =$ \emph{dim} induces a group homomorphism\dots
  $$K(k\textrm{-Vect}^f) \rightarrow \mathbb{Z}$$
  such that $\chi_K(V_{\bullet}) \mapsto \chi(V_{\bullet}).$
  \item This is in fact an isomorphism.
\end{itemize}
\end{proposition}