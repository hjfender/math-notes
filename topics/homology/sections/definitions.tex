\subsection{Definitions}\label{homologydefinitions}

The \emph{$i$-th homology} of a complex\dots
$$M_{\LargerCdot}: \cdots \xrightarrow[]{d_{i+2}} M_{i+1} \xrightarrow[]{d_{i+1}}  M_i \xrightarrow[]{d_i} M_{i-1} \xrightarrow[]{d_{i-1}} \cdots$$
of $R$-modules is the $R$-module\dots
$$H_i(M_{\LargerCdot}) := \frac{\textrm{ker }d_i}{\textrm{im }d_{i+1}}.$$

\begin{lemma}[Snake Lemma]
\label{snakelemma}
Given two short exact sequences linked together by homomorphisms as in the following commutative diagram\dots
\begin{figure}[H]
\centering
\begin{tikzcd}
0 \arrow[r] & L_1 \arrow[r, "\alpha_1"] \arrow[d, "\lambda"] & M_1 \arrow[r, "\beta_1"] \arrow[d, "\mu"] & N_1 \arrow[r] \arrow[d, "\nu"] & 0\\
0 \arrow[r] & L_0 \arrow[r, "\alpha_0"] & M_0 \arrow[r, "\beta_0"] & N_0 \arrow[r] & 0
\end{tikzcd}
\end{figure}
\noindent We are guaranteed an exact sequence\dots
$$0 \rightarrow \textrm{ker } \lambda \rightarrow \textrm{ker } \mu \rightarrow \textrm{ker } \nu \xrightarrow[]{\delta} \textrm{coker } \lambda \rightarrow \textrm{coker } \mu \rightarrow \textrm{coker } \nu \rightarrow 0.$$
\end{lemma}

\begin{corollary}
In the same as the snake lemma, assume $\mu$ is surjective and $\nu$ is injective. Then $\lambda$ is surjective and $\nu$ is an isomorphism.
\end{corollary}