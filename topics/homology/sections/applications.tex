\subsection{Applications to Simplicial Complexes}\label{simplicialhomology}

\subsubsection{Space of p-chains}\label{pchains}
Suppose $K$ is a simplicial complex. Then\dots
$$K_p = \{ S \in K | \textrm{dim} S = p \textrm{ and } S \textrm{ is nondegenerate} \}$$
forms a basis for the space of $p$-chains of $K$. The field we use for the rest of these notes is $\mathbb{F}_2$.
Thus the space is\dots
$$C_p(K; \mathbb{F}_2) = \mathbb{F}_2[K_p].$$

\noindent A simplicial mapping $\phi : K \rightarrow L$ induces a linear mapping $\phi_* : C_p(K; \mathbb{F}_2) \rightarrow C_p(L; \mathbb{F}_2)$
defined on a $p$-simplex as\dots
\[
\phi_*(\{v_0,\dots,v_p\}) = \begin{cases}
                                \phi(v_0),\dots,\phi(v_p) & \textrm{if } \{ \phi(v_0), \dots, \phi(v_p) \} \textrm{ is nondegenerate in } L,\\
                                \textbf{0} & \textrm{if } \{ \phi(v_0), \dots, \phi(v_p) \} \textrm{ is degenerate in } L.
                            \end{cases}.
\]

\subsubsection{Boundary Homomorphisms}\label{boundaryhomomorphism}
Given the simplex $S = \{ v_0, \dots, v_p \}$ the face opposing vertex $v_i$ is given by\dots
$$\partial_i(S) = \{ v_0, \dots, \hat{v_i}, \dots, v_p \} \subset S.$$
We can define a mapping $\partial : K_p \rightarrow C_{p-1}(K; \mathbb{F}_2)$ by summing all the $(p-1)$
faces of a $p$-simplex. The extension of this map to\dots
$$\partial : C_{p}(K; \mathbb{F}_2) \rightarrow C_{p-1}(K; \mathbb{F}_2)$$
given by\dots
$$\partial(S) = \sum_{i=0}^p \partial_i(S), \textrm{ for } S \in K_p.$$
is called the \emph{boundary homomorphism}.

\begin{proposition}
If $\phi : K \rightarrow L$ is a simplicial mapping, then\dots
$$\partial \circ \phi_* = \phi_* \circ \partial : C_{p}(K; \mathbb{F}_2) \rightarrow C_{p-1}(K; \mathbb{F}_2).$$
Furthermore, the composite $\partial \circ \partial : C_p(K; \mathbb{F}_2) \rightarrow C_{p-1}(K; \mathbb{F}_2)$ is
the zero mapping.
\end{proposition}

\subsubsection{Space of p-cycles}\label{pcycles}
The \emph{space of $p$-cycles} is the kernel of the boundary homomorphism\dots
$$Z_p(K) = \textrm{ker }\partial : C_p \rightarrow C_{p-1} = \{ c \in C_p(K; \mathbb{F}_2) | \partial(c) = \textbf{0} \}.$$

\subsubsection{Space of p-boundaries}\label{pboundaries}
The \emph{space of $p$-boundaries} is the image of the boundary homomorphism\dots
$$B_p(K) = \partial(C_{p+1}(K; \mathbb{F}_2)) = \{ b \in C_p(K; \mathbb{F}_2) | b = \partial(c), \textrm{ for some } c \in C_{p+1}(K; \mathbb{F}_2) \}.$$\newline

\noindent Since $\partial \circ \partial = 0$ we have $B_p(K) \subseteq Z_p(K)$ and can thus define the homology on a \hyperref[complexes]{chain
complex} of boundary homomorphisms.

\subsubsection{Chain Homotopies}\label{chainhomotopy}
Homology is a functor on chain complexes, that is, if $\phi : K \rightarrow L$ is a simplicial mapping, then\dots
$$H(\phi) : H_p(K; \mathbb{F}_2) \rightarrow H_p(L; \mathbb{F}_2) \textrm{ given by } H(\phi)(c + B_p(K)) = \phi_*(c) + B_p(L)$$
is a linear map and $H$ satisfies the functor properties.\newline

\noindent Given two simplicial mappings $\phi$ and $\psi : K \rightarrow L$, there is a \emph{chain homotopy} between them if
there is a linear mapping $h : C_p(K; \mathbb{F}_2) \rightarrow C_{p+1}(L; \mathbb{F}_2)$ for each $p$ which satisfies\dots
$$\partial \circ h + h \circ \partial = \phi_* + \psi_*.$$

\begin{theorem}
If there is a chain homomtopy between $\phi$ and $\psi$, then $H(\phi) = H(\psi)$.
\end{theorem}

\begin{corollary}
If $\phi$ and $\psi : K \rightarrow L$ are simplicial mappings, and $\phi$ is contiguous to $\psi$, then
$H(\phi) = H(\psi) : H_p(K; \mathbb{F}_2) \rightarrow H_p(L;\mathbb{F}_2)$ for all $p$.
\end{corollary}

\begin{corollary}
If $\phi$ and $\psi : K \rightarrow L$ are simplicial approximations of a continuous mapping $f : |K| \rightarrow |L|$,
then $H(\phi) = H(\psi) : H_p(K; \mathbb{F}_2) \rightarrow H_p(L; \mathbb{F}_2)$, for all $p$.
\end{corollary}

\subsubsection{Topological Invariance}
\begin{theorem}
There is an isomorphism of vector spaces for all $p \geq 0$\dots
$$H_p(\textrm{sd }K; \mathbb{F}_2) \cong H_p(K; \mathbb{F}_2).$$
\end{theorem}

\begin{proof}
This proof is lengthy and used the \emph{method of acyclic models}. Look up elsewhere.
\end{proof}

\begin{theorem}[Topological Invariance of Homology]
Suppose $K$ and $L$ are simplicial complexes with $|K|$ and $|L|$ homeomorphic. Then,
for all $p$, the vector spaces $H_p(K; \mathbb{F}_2)$ and $H_p(K; \mathbb{F}_2)$ are isomorphic.
\end{theorem}

\begin{proof}
This proof is also lengthy and involves the previous theorem. Also look it up.
\end{proof}

\begin{corollary}
The Euler Poincar\'e characteristic is a topological invariant of a triangulable space.
\end{corollary}

\begin{theorem}
There are only five Platonic solids.
\end{theorem}

\begin{proof}
A polyhedron $P$ need not be a simplicial complex, since the faces can be polygons not necessarily triangles. However,
if we subdivide each constituent polygon into triangles, we get a simplicial complex. This means Eulers formula for $P$
computes the same result as the Euler-Poincar\'e characteristic $\chi(P)$. Thus since $P$ has a realization homeomorphic
to $S^2$, we know that $\chi(P) = 2$.

Suppose each face has $M$ edges and, at each vertex, $N$ faces meet. This gives us\dots
$$M n_2 / 2 = n_1$$
where $n_2$ is the number of faces and $n_1$ is the number of edges. Also we get\dots
$$N n_0 / 2 = n_1$$
where $n_0$ is the number of vertices. Putting these relations into Euler's formula we get\dots
\begin{align*}
    2 &= n_0 - n_1 + n_2\\
      &= (2n_1/N) - n_1 + (2n_1/M)\\
      &= n_1((2/N) + (2/M) - 1).
\end{align*}
It follows that\dots
$$\frac{n_1}{2} = \frac{MN}{2M + 2N - MN}.$$

If $N=1$ or $N=2$, there would be a boundary and so the polyhedron would fail to be a sphere. Since a Platonic solid
encloses space, $N > 2$. Also $M \geq 3$ since each face is a polygon. Finally, $n_1$ must be an integer which is at least
$M$.

These facts forces $M < 6$. To see this, suppose $M \geq 6$ and $N > 2$. Then $2 - N < 0$ and we have\dots
$$0 < 2M + 2N - MN = 2N + M(2 - M) \leq 2N + 6(2-N) = 12 - 4N.$$
This implies that $4N < 12$, or that $N<3$, which is impossible because $N$ is an integer and $N>2$.

Setting $M=3$ we get $n_1 = 6N/(6-N)$, which is an integer when $N = 3,4,5$. The case $N = 3$, $M = 3$ is realized by
the tetrehedron, $N = 4$ and $M = 3$ is realized by the octahedron, and for $N=5$, $M=3$ by the icosahedron.

For $M=4$ we have $n_1 = 8N/(8-2N) = 4N/(4-N)$, and so $N=3$ is the only case of interest which is realized by the cube.
Finally, for $M=5$ we have $n_1 = 10N/(10 - 3N)$ and so $N=3$ is the only possible case, which gives the dodecahedron.
\end{proof}

\begin{theorem}
If $K$ is a simplicial complex, then dim$_{\mathbb{F}_2} H_0(K; \mathbb{F}_2) = \#\pi_0(|K|)$, the number of path components of $|K|$.
\end{theorem}

\begin{proof}
This is lengthy and should be looked up.
\end{proof}

\begin{theorem}[Brouwer Fixed Point Theorem]
If $e^n = \{ x \in \mathbb{R}^n \, | \, ||x|| \leq 1 \}$ denotes the $n$-disk and $f :e^n \rightarrow e^n$ is a
continuous mapping, then there is a point $x_0 \in e^n$ with $f(x_0) = x_0$, that is, $e^n$ has the fixed point
property.
\end{theorem}

\begin{proof}
Suppose that $f : e^n \rightarrow e^n$ is a continuous mapping without fixed points. If $y \in e^n$, then
$y \neq f(y)$. Join $f(y)$ to $y$ and continue this ray until it meets $S^{n-1} = \textrm{bdy}e^n$ and denote
this point by $g(y)$. We can characterize $g(y)$ by $g(y) = (1-t)f(y) + ty$, where $t>0$ and $||g(y)|| = 1$. Because
we are in a nicely behaved inner porduct space, the argument for hte case of $n=2$ carries over exactly to prove that
$g: e^n \rightarrow S^{n-1}$ is continuous. Furthermore, by the definition of $g$, $g \circ i : S^{n-1} \rightarrow S^{n-1}$
is the identity when $i : S^{n-1} \rightarrow e^n$ is the inclusion of the boundary.

Apply homology to this composite $\textrm{id}_{S^{n-1}} = g \circ i$ to obtain $H(\textrm{id}_{S^{n-1}})$, an isomorphism,
written as $H(g) \circ H(i)$. However, $H_{n-1}(S^{n-1}; \mathbb{F}_2) \neq \{ 0 \}$ while $H_{n-1}(e^n; \mathbb{F}_2) = \{ 0 \}$,
because $e^n$ is homeomorphic to $\Delta^n$. Thus, $H(i) : H_{n-1}(e^n; \mathbb{F}_2) \rightarrow H_{n-1}(e^n; \mathbb{F}_2)$
is the zero homomorphism $[c] \mapsto 0$. An isomorphism\dots
$$H(\textrm{id}_{S^{n-1}}: H_{n-1}(S^{n-1}; \mathbb{F}_2) \rightarrow H_{n-1}(S^{n-1}; \mathbb{F}_2)$$
cannot be factored as $H(g) \circ ([c] \mapsto 0)$; and so a continuous mapping $f: e^n \rightarrow e^n$ without
fixed points cannot exist.
\end{proof}