\subsection{Applications to Simplicial Complexes}\label{simplicialhomology}

\subsubsection{Space of p-chains}\label{pchains}
Suppose $K$ is a simplicial complex. Then\dots
$$K_p = \{ S \in K | \textrm{dim} S = p \textrm{ and } S \textrm{ is nondegenerate} \}$$
forms a basis for the space of $p$-chains of $K$. The field we use for the rest of these notes is $\mathbb{F}_2$.
Thus the space is\dots
$$C_p(K; \mathbb{F}_2) = \mathbb{F}_2[K_p].$$

\noindent A simplicial mapping $\phi : K \rightarrow L$ induces a linear mapping $\phi_* : C_p(K; \mathbb{F}_2) \rightarrow C_p(L; \mathbb{F}_2)$
defined on a $p$-simplex as\dots
\[
\phi_*(\{v_0,\dots,v_p\}) = \begin{cases}
                                \phi(v_0),\dots,\phi(v_p) & \textrm{if } \{ \phi(v_0), \dots, \phi(v_p) \} \textrm{ is nondegenerate in } L,\\
                                \textbf{0} & \textrm{if } \{ \phi(v_0), \dots, \phi(v_p) \} \textrm{ is degenerate in } L.
                            \end{cases}.
\]

\subsubsection{Boundary Homomorphisms}\label{boundaryhomomorphism}
Given the simplex $S = \{ v_0, \dots, v_p \}$ the face opposing vertex $v_i$ is given by\dots
$$\partial_i(S) = \{ v_0, \dots, \hat{v_i}, \dots, v_p \} \subset S.$$
We can define a mapping $\partial : K_p \rightarrow C_{p-1}(K; \mathbb{F}_2)$ by summing all the $(p-1)$
faces of a $p$-simplex. The extension of this map to\dots
$$\partial : C_{p}(K; \mathbb{F}_2) \rightarrow C_{p-1}(K; \mathbb{F}_2)$$
given by\dots
$$\partial(S) = \sum_{i=0}^p \partial_i(S), \textrm{ for } S \in K_p.$$
is called the \emph{boundary homomorphism}.

\begin{proposition}
If $\phi : K \rightarrow L$ is a simplicial mapping, then\dots
$$\partial \circ \phi_* = \phi_* \circ \partial : C_{p}(K; \mathbb{F}_2) \rightarrow C_{p-1}(K; \mathbb{F}_2).$$
Furthermore, the composite $\partial \circ \partial : C_p(K; \mathbb{F}_2) \rightarrow C_{p-1}(K; \mathbb{F}_2)$ is
the zero mapping.
\end{proposition}

\subsubsection{Space of p-cycles}\label{pcycles}
The \emph{space of $p$-cycles} is the kernel of the boundary homomorphism\dots
$$Z_p(K) = \textrm{ker }\partial : C_p \rightarrow C_{p-1} = \{ c \in C_p(K; \mathbb{F}_2) | \partial(c) = \textbf{0} \}.$$

\subsubsection{Space of p-boundaries}\label{pboundaries}
The \emph{space of $p$-boundaries} is the image of the boundary homomorphism\dots
$$B_p(K) = \partial(C_{p+1}(K; \mathbb{F}_2)) = \{ b \in C_p(K; \mathbb{F}_2) | b = \partial(c), \textrm{ for some } c \in C_{p+1}(K; \mathbb{F}_2) \}.$$\newline

\noindent Since $\partial \circ \partial = 0$ we have $B_p(K) \subseteq Z_p(K)$ and can thus define the homology on a \hyperref[complexes]{chain
complex} of boundary homomorphisms.

\subsubsection{Chain Homotopies}\label{chainhomotopy}
Homology is a functor on chain complexes, that is, if $\phi : K \rightarrow L$ is a simplicial mapping, then\dots
$$H(\phi) : H_p(K; \mathbb{F}_2) \rightarrow H_p(L; \mathbb{F}_2) \textrm{ given by } H(\phi)(c + B_p(K)) = \phi_*(c) + B_p(L)$$
is a linear map and $H$ satisfies the functor properties.\newline

\noindent Given two simplicial mappings $\phi$ and $\psi : K \rightarrow L$, there is a \emph{chain homotopy} between them if
there is a linear mapping $h : C_p(K; \mathbb{F}_2) \rightarrow C_{p+1}(L; \mathbb{F}_2)$ for each $p$ which satisfies\dots
$$\partial \circ h + h \circ \partial = \phi_* + \psi_*.$$

\begin{theorem}
If there is a chain homomtopy between $\phi$ and $\psi$, then $H(\phi) = H(\psi)$.
\end{theorem}

\begin{corollary}
If $\phi$ and $\psi : K \rightarrow L$ are simplicial mappings, and $\phi$ is contiguous to $\psi$, then
$H(\phi) = H(\psi) : H_p(K; \mathbb{F}_2) \rightarrow H_p(L;\mathbb{F}_2)$ for all $p$.
\end{corollary}

\begin{corollary}
If $\phi$ and $\psi : K \rightarrow L$ are simplicial approximations of a continuous mapping $f : |K| \rightarrow |L|$,
then $H(\phi) = H(\psi) : H_p(K; \mathbb{F}_2) \rightarrow H_p(L; \mathbb{F}_2)$, for all $p$.
\end{corollary}