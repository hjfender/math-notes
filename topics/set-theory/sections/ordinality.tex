\subsection{Ordinal Numbers}\label{ordinalnumbers}

\subsubsection{Partial Orderings}\label{partialorderings}
A \emph{partial ordering} is a \hyperref[relation]{relation} $R$ such that\dots
\begin{enumerate}
  \item $R$ is transitive
  \item $R$ is irreflexive, that is for all $x$ we have $x \cancel{R} x$
\end{enumerate}

\begin{proposition}
Assume that $<$ is a partial ordering. Then for $x,y,$ and $z$:
\begin{enumerate}
  \item \emph{At most} one of the alternatives,
  		$$x < y, \; \; x = y, \; \; y < x,$$
  		can hold.
  \item $x \leq y \leq x \Rightarrow x = y.$
\end{enumerate}
\end{proposition}

\subsubsection{Linear Orderings}\label{partialorderings}
A \emph{linear ordering} is a partial ordering $R$ that satisfies \hyperref[trichotomy]{trichotomy}.

\subsubsection{Well Orderings}\label{wellorderings}
A \emph{well ordering} is a linear ordering $R$ on $A$ such that every nonempty subset of $A$ has a least element.

\begin{theorem}
Let $<$ be a linear ordering on $A$. Then if is a well ordering if and only if there does not exist any function $f: \omega \rightarrow A$ with $f(n^+) < f(n)$ for every $n \in \omega$.
\end{theorem}

\begin{theorem}[Transifinite Induction Principle]\label{transfiniteinduction}
Assume that $<$ is a well ordering on $A$. Assume that $B$ is a subset of $A$ with the special property that for every $t \in A$,
$$\textrm{seg } t \subseteq B \Rightarrow t \in B.$$
Then $B$ coincides with $A$.
\end{theorem}

\begin{proof}
If $B \subset A$, then $A \setminus B$ has a least element $m$. Bu the leastness, $y \in B$ for any $y < m$. But this is to say that $\textrm{seg } m \subseteq B$, so by assumption $m \in B$ after all.
\end{proof}

\begin{proposition}
Assume that $<$ is a linear ordering on $A$. Further assume that the only subset of $A$ such that $\forall \, t \in A, \textrm{ seg } t \subseteq B \Rightarrow t \in B$ is $A$ itself. Then $<$ is a well ordering on $A$.
\end{proposition}

\subsubsection{Transfinite Recursion}\label{transfiniterecursion}

\begin{theorem}[Transfinite Recursion Theorem Schema]
For any formula $\gamma (x,y)$ the following is a theorem:

Assume that $<$ is a well ordering on a set $A$. Assume that for any $f$ there is a unique $y$ such that $\gamma (f,y).$ Then there exists a unique 
function $F$ with domain $A$ such that\dots
$$\gamma (F \upharpoonright \textrm{ seg } t, \, F(t))$$
for all $t \in A$.
\end{theorem}

\noindent The following \hyperref[setaxioms]{axiom} is used to prove the transfinite recursion theorem schema.\newline

\noindent For any formula $\varphi (x,y)$ not containing the letter $B$, the following is an axiom:
$$\forall [(\forall x \in A) \forall y_1 \forall y_2 (\varphi (x,y_1) \land \varphi (x, y_2) \Rightarrow y_1 = y_2)$$
$$\Rightarrow \exists B \forall y (y \in B \Leftrightarrow (\exists x \in A) \varphi (x,y))].$$

\subsubsection{Epsilon Images}\label{epsilonimages}

Let $<$ be a well ordering on $A$ and let $\gamma (x,y)$ be the formulat $y = \textrm{ ran } x$. Then the transfinite recursion theorem gives an unique function $E$ with domain $A$ such that $\forall t \in A$:
\begin{align*}
E(t) &= \textrm{ ran }(E \upharpoonright \textrm{ seg } t)\\
	 &= E[\textrm{seg } t]\\
	 &= \{ E(x) | x < t \}.
\end{align*}

\noindent The $\epsilon$-image of $\langle A, < \rangle$ is the range of $E$.

\begin{proposition}
Let $<$ be a well ordering on $A$ and let $E$ be as above and $\alpha$ its epsilon image.
\begin{enumerate}
  \item $E(t) \not\in E(t)$ for any $t \in A.$
  \item $E$ maps $A$ bijectively to $\alpha$.
  \item For any $s$ and $t$ in $A$,
  				$$s < t \textrm{  if and only if  } E(s) \in E(t)$$
  \item $\alpha$ is a \hyperref[transitiveset]{transitive} set.
\end{enumerate}
\end{proposition}

\subsubsection{Ordinal Numbers}\label{ordinalnumberdefinition}

\begin{proposition}
Two well-ordered structures are isomorphic if and only if they have the same $\epsilon$-image.
That is, if $<_1$ and $<_2$ are well orderings on $A_1$ and $A_2$, respectively, then $\langle A_1, <_1 \rangle \cong \langle A_2, <_2 \rangle$
if and only if the $\epsilon$-image of $\langle A_1, <_1 \rangle$ is the same as the $\epsilon$-image of $\langle A_2, <_2 \rangle$.
\end{proposition}

The \emph{ordinal number} of $\langle A, < \rangle$ is its $\epsilon$-image. An \emph{ordinal number} is a set that is the ordinal number of some well-ordered structure.

\subsubsection{Cardinal Numbers}\label{cardinalnumberdefinition}

\begin{theorem}[Numeration Theorem]
Any set is equinumerous to some ordinal number.
\end{theorem}

\noindent For any set $A$, define the \hyperref[cardinalnumbers]{cardinal number} of $A$ (card $A$) to be the least ordinal equinumerous to $A$.
