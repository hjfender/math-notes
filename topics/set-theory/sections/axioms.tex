\subsection{Set Axioms}\label{setaxioms}

\subsubsection{Undefined notions}
\emph{Set:} $A,B,C,\dots$

\subsubsection{Axioms}\label{statementofsetaxioms}
\begin{enumerate}
  \item \emph{Extension:} $\forall A \forall B [\forall C ( C \in A \Leftrightarrow C \in B ) \Rightarrow A = B ]$
  \item \emph{Regularity:} $\forall A [\exists C (C \in A) \Rightarrow \exists B ( B \in A \land \lnot \exists D (D \in B \land D \in A))]$ \newline
  {\small (Every nonempty set contains a set that is disjoint from it. Also know as "Axiom of Foundation.")}
  \item \emph{Schema of Specification:} $\forall B \forall X_1 \forall X_2 \dots \forall X_n \exists A \forall C [C \in A \Leftrightarrow ( C \in B \land \phi)]$
  \item \emph{Pairing:} $\forall X_1 \forall X_2 \exists A (X_1 \in A \land X_2 \in A)$
  \item \emph{Union:} $\forall \mathcal{F}_A \exists U \forall A \forall X [(X \in A \land A \in \mathcal{F}_A) \Rightarrow X \in U]$
  \item \emph{Schema of Replacement:} $\forall A \forall X_1 \forall X_2 \dots \forall X_n[\forall B(B \in A \Rightarrow \exists! D \phi) \Rightarrow \exists B \forall C (C \in A \Rightarrow \exists D(D \in B \land \phi))]$
  \item \emph{Infinity:} $\exists \omega [\emptyset \in \omega \land \forall X(X \in \omega \Rightarrow X \cup X\} \in \omega)]$
  \item \emph{Power Set:} $\forall X \exists \mathcal{P}(X) \forall S [S \subseteq X \Rightarrow S \in \mathcal{P}(X)]$
  \item \emph{Empty Set:} $\exists A \forall X (X \not\in A)$
  \item \emph{Choice:} $\forall X [\emptyset \not\in X \Rightarrow \exists (f:X \rightarrow \bigcup X) \forall A \in X (f(A) \in A)]$
\end{enumerate}

\begin{proposition}
	The empty set axiom is implied by the other nine axioms.
\end{proposition}

\begin{proof}
	Just choose any formula that is always false such as $\phi(X) = X \in B \land X \not \in B$ and apply the axiom schema of specification. This
	will give the empty set. The axiom of extension proves uniqueness vacuously.
\end{proof}

\subsubsection{Universe}\label{universe}
A set $U$ is defined with the following properties\dots
\begin{enumerate}
  \item $x \in u \in U \Rightarrow x \in U$
  \item $u \in U \land v \in U \Rightarrow \{u,v\}, \langle u,v \rangle, u \times v \in U$
  \item $X \in U \Rightarrow \mathcal{P}(X) \in U \land \bigcup X \in U$
  \item $\omega \in U$ is the set of finite ordinals
  \item if $f:A \rightarrow B$ is a surjective function with $A \in U \land B \subset U$, then $B \in U$
\end{enumerate}

(See: \hyperref[setconstructions]{Set Constructions}.)\newline

\noindent In category theory, \emph{small sets} are members of $U$. \label{smallsets}