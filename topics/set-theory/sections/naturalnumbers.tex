\subsection{Natural Numbers}\label{naturalnumbers}

\subsubsection{Successor}\label{successor}
For a set $n$, its \emph{successor} $n^+$ is defined by\dots
$$n^+ = n \cup n\}$$

\subsubsection{Inductive}\label{inductive}
A set $N$ is \emph{inductive}\label{inductive} if and only if $\emptyset \in N$ and $(\forall n \in N) \, n^+ \in N.$\newline

\noindent The \hyperref[statementofsetaxioms]{Axiom of Infinity} may be restated in terms of "inductiveness," i.e.\dots $$\emph{There exists an inductive set } \omega_0.$$

\subsubsection{Natural Number}
A \emph{natural number} is a set that belongs to every inductive set, i.e. the intersection of them all.\newline

\noindent The following theorem is a consequence of the definition\dots

\begin{theorem}[Induction on $\omega_0$]
Any inductive subset of $\omega_0$ coincides with $\omega_0$.
\end{theorem}

\begin{proposition}
Every natural number except $0$ is the successor of some natural number.
\end{proposition}

\begin{proof}
Let $T = \{n \in \omega_0 | n=0 \lor (\exists p \in \omega_0) n = p^+\}$ and use induction.
\end{proof}

\subsubsection{Peano's Postulates}

\subsubsubsection{Peano System}\label{peanosystem}
An ordered triple $\langle N, S, e \rangle$ consiting of a set $N$, a function $S: N \rightarrow N$, and a member $e \in N$ such that the following three conditions are met:
\begin{enumerate}
  \item $e \not\in \textrm{ran} S$.
  \item $S$ is injective.
  \item Any subset $A \subseteq N$ that contains $e$ and is closed under $S$ equals $N$ itself.
\end{enumerate}

\begin{proposition}
Let $\sigma = \{\langle n, n^+ \rangle | n \in \omega_0\}$. Then $\langle \omega_0, \sigma, 0 \rangle$ is a Peano system.
\end{proposition}

\subsubsubsection{Transitive Set}\label{transitiveset}
A set $A$ is said to be a \emph{transitive set} if and only if $x \in a \in A \Rightarrow x \in A$.

\begin{proposition}
For a transitive set $a$,
$$\bigcup(a^+) = a.$$
\end{proposition}

\begin{proposition}
Every natural number is a transitive set and $\omega_0$ is a transitive set.
\end{proposition}

\begin{proof}
Use induction.
\end{proof}

\subsubsection{Recursion}\label{recursion}

\begin{theorem}[Recursion Theorem on $\omega_0$]
Let $A$ be a set, $a \in A$, and $F: A \rightarrow A.$ Then there exists an unique function $h: \omega_0 \rightarrow A$ such that\dots
$$h(0) = a,$$
and for every $n \in \omega_0$,
$$h(n^+) = F(h(n)).$$
\end{theorem}

\begin{proof}
The idea is to lef $h$ be the union of many approximating functions. For the purposes of this proof, call a function $v$ \emph{acceptable}
if and only if dom $v \subseteq \omega_0$, ran $v \subseteq A$, and the following conditions hold:
\begin{enumerate}
  \item If $0 \in$ dom $v$, then $v(0) = a$.
  \item If $n^+ \in$ dom $v$ (where $n \in \omega_0$), then also $n \in$ dom $v$ and $v(n^+) = F(v(n))$.
\end{enumerate}

Let $\mathcal{H}$ be the collection of all acceptable functions, and let $h = \bigcup \mathcal{H}.$ Thus\dots
\begin{align*}
(\star) \; \; \; \langle n, y \rangle \in h &\Leftrightarrow \langle n, y \rangle \textrm{ is a member of some acceptable } v\\
 						   &\Leftrightarrow v(n) = y \textrm{ for some acceptabe } v.
\end{align*}

We claim that this $h$ meets the demands of the theorem. This claim can be broken down into four parts. The four parts involve showing that (I) $h$ is a function,
(II) $h$ is acceptable, (III) dom $h$ is all of $\omega_0$, and (IV) $h$ is unique.\newline


I. We first claim that $h$ is a function. Let\dots
$$S = \{ n \in \omega_0 | \textrm{ for at most one } y, \langle n, y \rangle \in h \}.$$
We must check that $S$ is inductive. If $\langle 0, y_1 \rangle \in h$ and $\langle 0, y_2 \rangle \in h$, then by ($\star$) there exist acceptable $v_1$ and $v_2$
such that $v_1(0) = y_1$ and $v_2(0)=y_2$. But by (1) it follows that $y_1 = a = y_2$. Thus $0 \in S$.

Next suppose that $k \in S$. Consider $\langle k^+, y_1 \rangle \in h$ and $\langle k^+, y_2 \rangle \in h.$ As before there must exist acceptabel $v_1$ and $v_2$ such that
$v_1(k^+) = y_1$ and $v_2(k+) = y_2.$ By condition (2) it follows that\dots
$$y_1 = v_1(k^+) = F(v_1(k)) \; \; \; \textrm{ and } \; \; \; y_2 = v_2(k^+) = F(v_2(k)).$$
But since $k \in S$, we have $v_1(k) = v_2(k).$ Therefore\dots
$$y_1 = F(v_1(k)) = F(v_2(k)) = y_2.$$
So $k^+ \in S$, proving $S$ is inductive and conincides with $\omega_0$. Consequently $h$ is a function.\newline


II. Next we claime that $h$ itself is acceptable. We have just seen that $h$ is a function, and it is clear from ($\star$) that dom $h \subseteq \omega_0$ and ran $h \subseteq A.$

First examine (1). If $0 \in$ dom $h$, then there must be some acceptable $v$ with $v(0) = h(0).$ Since $v(0) = a$, we have $h(0) = a$.

Next examine (2). Assume $n^+ \in$ dom $h$. Again there must be some acceptable $v$ with $v(n^+) = h(n^+)$. Since $v$ is acceptable we have $n \in$ dom $v$ (and $v(n) = h(n)$) and
$$h(n^+) = v(n^+) = F(v(n)) = F(h(n)).$$
Thus $h$ satisfies (2) and so is acceptable.\newline


III. We now claim that dom $h = \omega_0$ (the function is nonempty). It suffices to show that dom $h$ is inductive. The function $\{ \langle 0,a \rangle \}$ is acceptable and hence
$0 \in$ dom $h$. Suppose the $k \in$ dome $h$. If $k^+ \not\in$ dom $h$, then let\dots
$$v = h \cup \{ \langle k^+, F(h(k)) \rangle \}.$$
Then $v$ is a function, dom $v \subseteq \omega_0$, and ran $v \subseteq A$. We will show that $v$ is acceptable.

Condition (1) holds since $v(0) = h(0) = a.$ For condition (2) there are two cases. If $n^+ \in$ dom $v$ where $n^+ \neq k^+$, then $n^+ \in$ dom $h$ and $v(n^+) = h(n^+) = F(h(n)) = F(v(n)).$
The other case occurs if $n^+ = k^+$. Since the successor operation is injective, $n=k$. By assumption $k \in$ dom $h$. Thus\dots
$$v(k^+) = F(h(k)) = F(v(k))$$
and (2) holds. Hence $v$ is acceptable. But then $v \subseteq h$, so that $k^+ \in$ dom $h$ after all. So dom $h$ is inductive and therefore coincides with $\omega_0$.\newline


IV. Finally we claim that $h$ is unique. For let $h_1$ and $h_2$ both satisfy the conclusion fo the theorem. Let\dots
$$S = \{n \in \omega_0 | h_1(n) = h_2(n) \}.$$

$S$ is inductive, showing $h_1 = h_2$. Thus $h$ is unique.
\end{proof}

\begin{example}
There is no function $h: \mathbb{Z} \rightarrow \mathbb{Z}$ such that for every $a \in \mathbb{Z}$,
$$h(a + 1) = h(a)^2 + 1.$$
\end{example}

\begin{proof}
Note $h(a) > h(a-1) > h(a-2) > \dots > 0.$  Recursion on $\omega_0$ reliex on there being a starting point $0$. $\mathbb{Z}$ has no analogous starting point.
\end{proof}

\begin{theorem}
Let $\langle N, S, e \rangle$ be a Peano system. Then $\langle \omega_0, \sigma, 0 \rangle$ is isomorphic to $\langle N, S, e \rangle$, i.e. there is a funciton $h$
mapping $\omega_0$ bijectively to $N$ in a way that preserves the successor operation
$$h(\sigma(n)) = S(h(n))$$
and the zero element
$$h(0) = e.$$
\end{theorem}


\subsubsection{Arithmetic}\label{arithmetic}

\subsubsubsection{Addition}\label{addition}
\emph{Addition} (+) is the binary operation on $\omega_0$ such that for any $m$ and $n \in \omega_0$,
$$m + n = A_m(n),$$
where $A_m: \omega_0 \rightarrow \omega_0$ is the unique function given by the \hyperref[recursion]{recursion theorem} for which\dots
\begin{itemize}
  \item $A_m(0) = m$
  \item $A_m(n^+) = A_m(n)^+ \; \forall n \in \omega_0.$
\end{itemize}

\begin{proposition}
For natural numbers $m$ and $n$,
\begin{itemize}
  \item $m + 0 = m,$
  \item $m + n^+ = (m+n)^+$
\end{itemize}
\end{proposition}

\subsubsubsection{Multiplication}\label{multiplication}
\emph{Multiplication} ($\cdot$) is the binary operation on $\omega_0$ such that for any $m$ and $n \in \omega_0$,
$$m \cdot n = M_m(n),$$
where $M_m: \omega_0 \rightarrow \omega_0$ is the unique function given by the \hyperref[recursion]{recursion theorem} for which\dots
\begin{itemize}
  \item $M_m(0) = 0$
  \item $M_m(n^+) = M_m(n) + m.$
\end{itemize}

\begin{proposition}
For natural numbers $m$ and $n$,
\begin{itemize}
  \item $m \cdot 0 = 0,$
  \item $m \cdot n^+ = m \cdot n + m$
\end{itemize}
\end{proposition}

\subsubsubsection{Exponentiation}\label{exponentiation}
\emph{Exponentiation} is the binary operation on $\omega_0$ such that for any $m$ and $n \in \omega_0$,
$$m^n = E_m(n),$$
where $E_m: \omega_0 \rightarrow \omega_0$ is the unique function given by the \hyperref[recursion]{recursion theorem} for which\dots
\begin{itemize}
  \item $E_m(0) = 1$
  \item $M_m(n^+) = E_m(n) \cdot m.$
\end{itemize}

\begin{proposition}
For natural numbers $m$ and $n$,
\begin{itemize}
  \item $m^0 = 1,$
  \item $m^{(n^+)} = m^n \cdot m.$
\end{itemize}
\end{proposition}

\subsubsection{Ordering on the natural numbers}\label{orderingonnaturalnumbers}
Define $m < n$ if and only if $m \in n.$

\begin{lemma}
For any natural numbers $m$ and $n$\dots
\begin{itemize}
  \item $m \in n \Leftrightarrow m^+ \in n^+.$
  \item $n \not\in n$
\end{itemize}
\end{lemma}

\begin{theorem}[Trichotomy Law for $\omega_0$]\label{trichotomy}
For any natural numbers $m$ and $n$, exactly one of the three conditions\dots
\begin{itemize}
  \item $m \in n$
  \item $m = n$
  \item $n \in m$
\end{itemize}
holds.
\end{theorem}

\begin{corollary}
For any natural numbers $m$ and $n$,
\begin{itemize}
  \item $m \in n \Leftrightarrow m \subset n$
  \item $(m \in n) \lor (m = n) \Leftrightarrow m \subseteq n$
\end{itemize}
\end{corollary}

\begin{proposition}
For any natural numbers $m,n$ and $p$,\dots
\begin{itemize}
  \item $m \in n \Leftrightarrow m + p \in n + p.$
  \item If, in addition, $p \neq 0$, then $m \in n \Leftrightarrow m \cdot p \in n \cdot p.$
\end{itemize}
\end{proposition}

\begin{corollary}
The following canncellation laws hold for $m,n,p \in \omega_0$\dots
\begin{itemize}
  \item $m + p \in n + p \Rightarrow m = n$
  \item If, in addition, $p \neq 0$, then $m \cdot p \in n \cdot p \Rightarrow m = n$
\end{itemize}
\end{corollary}

\begin{theorem}[Well Ordering of $\omega_0$]\label{wellorderingofnaturalnumbers}
Let $A$ be a nonempty set of $\omega_0.$ Then there is some $m \in A$ such that $(m \in n) \lor (m=n)$ for all $n \in A$.
\end{theorem}

\begin{proof}
Assume that $A$ is a subset of $\omega_0$ without a least element; we will 
show that $A = \emptyset$. We could attempt to do this by showing that the complement
$\omega_0 \setminus A$ is inductive. But in order to show that $k^+ \in \omega_0 - A$,
it is not enough to know merely that $k \in \omega_0 \setminus A$, we must know
that all numbers smaller than $k$ are in $\omega_0 \setminus A$ as well. Given this additional
information, we can argue that $k^+ \in \omega_0 \setminus A$ lest it be a least element of $A$.

To write down what is approximately this argument, let\dots
$$B = \{m \in \omega_0 | \textrm{ no number less than } m \textrm{ belongs to } A \}.$$
We claim that $B$ is inductive. $0 \in B$ vacuously. Suppose that $k \in B$. Then if 
$n$ is less that $k^+$, either $n$ is less than $k$ (in which case $n \not\in A$ since $k \in B$) or
$n = k$ (in which case $n \not\in A$ lest, by trichotomy, it be least in $A$). In either case, 
$n$ is outside of $A$. Hence $k^+ \in B$ and $B$ is inductive. It clearly follows that
$A = \emptyset$.
\end{proof}

\begin{corollary}
There is no function $f : \omega_0 \rightarrow \omega_0$ such that $f(n^+) \in f(n)$
for every natural number $n$.
\end{corollary}

\begin{theorem}[Strong Induction Principle for $\omega_0$]
Let $A$ be a subset of $\omega_0$, and assume the for every $n \in \omega_0$, if
every number less than $n$ is in $A$, then $n \in A.$ Then $A = \omega_0$.
\end{theorem}
