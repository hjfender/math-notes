\subsection{Relations}\label{relations}
$\mathcal{R} :\subseteq A \times B \textrm{ for some } A \times B$

\subsubsection{Equivalence Relations}\label{equivalencerelation}
Relations $\sim \subseteq A \times A$ such that $\forall a,b,c \in A$\dots
\begin{itemize}
  \item \emph{Reflexive:} $a \sim a$
  \item \emph{Symmetric:} $a \sim b \Rightarrow b \sim a$
  \item \emph{Transitive:} $a \sim b \land b \sim c \Rightarrow a \sim c$
\end{itemize}

\subsubsubsection{Equivalence Class}
$[a]_{\sim} := \{ b \in S | b \sim a \}$

\subsubsubsection{Set of Equivalence Classes}
$[A] = \{[a]_{\sim} | a \in A \}$

\subsubsubsection{Set Partition}
A set $P :\subseteq \mathcal{P}(X)$ such that\dots
\begin{itemize}
  \item $\bigcup P = X$
  \item $\forall S_1, S_2 \in P (S_1 \cap S_2 \neq \emptyset \Rightarrow S_1 = S_2)$
\end{itemize}

\begin{proposition}
Let $A$ is a set and $\sim$ an equivalence relation on $A$. Then $[A]$ is a partition of $A$.
\end{proposition}

\begin{proposition}
Let $A$ be a set and $P$ be a partition of $A$. Define a relation $x \sim y$ if and only if $x,y \in C \in P$. Then $\sim$ is an equivalence relation.
\end{proposition}

\subsubsubsection{Congruence Relation}\label{congruencerelation}
A \hyperref[congruence]{congruence} $\sim$ of a set $A$ with a binary operation $\mu : A \times A \rightarrow A$ is an equivalence relation such that\dots
$$\overline{\mu}([a],[b]) = [\mu (a,b)]$$
induces a well-defined binary operation on $[A]$.

\begin{proposition}
An equivalence relation $\sim$ on $A$ with $\mu : A \times A \rightarrow A$ is a congruence relation if for any $a,a',b,b' \in A$, whenever $[a] = [a']$ and
[b] = [b'], we have $[\mu(a,b)]=[\mu(a',b')]$.
\end{proposition}

\subsubsection{Functions}\label{function}
A relation $f: A \rightarrow B$ satisfying $\forall a \in A$ \mbox{$\exists! b \in B \textrm{ such that } afb \textrm{, denoted } f(a) = b$.}

\subsubsubsection{Injection}\label{injection}
A function $\ensuremath{f: A \hookrightarrow B}$ such that $\forall \, x,y \in A \textrm{ if } x \neq y, \textrm{ then } f(x) \neq f(y)$. (See: \hyperref[monomorphism]{monomorphism}. Injections have right inverses.)

\subsubsubsection{Surjection}\label{surjection}
A function $\ensuremath{f: A \twoheadrightarrow B}$ such that $\forall \, b \in B \; \exists \, a \in A \textrm{ such that } f(a) = b$. (See: \hyperref[epimorphism]{epimorphism}, \hyperref[secondstirlingnumbers]{Stirling numbers of the second kind}.
Surjections have left inverses, called \emph{sections}.)

\subsubsubsection{Bijection}\label{bijection}
A function $f: A \xrightarrow{\sim} B$ which is an injection and a surjection. (See: \hyperref[isomorphism]{isomorphism})

\subsubsubsection{Restriction}\label{restriction}
For $C \subseteq A$ and $f:A \rightarrow B$, $\ensuremath{f\mathord{\upharpoonright}_C:C \rightarrow B}$ where \mbox{$\forall c \in C \, \ensuremath{f\mathord{\upharpoonright}_C(c)} := f(c)$}

\subsubsubsection{Image}\label{image}
$f(A) := \{f(a) | a \in A \}$

\begin{proposition}
For a function $f : A \rightarrow B$ and a family $\{X_i\}_{i \in I}$ where $\forall i \in I$ $X_i \subseteq A$\dots
\begin{itemize}
  \item $f(\bigcup_i X_i) = \bigcup_i f(X_i)$
  \item $\textrm{In general, } f(\bigcap_i X_i) \neq \bigcap_i f(X_i)$
  \item $\textrm{In general, } f(X)^c \neq f(X^c)$
\end{itemize}
\end{proposition}

\subsubsubsection{Preimage}\label{preimage}
$f^{-1}(A) := \{a \in A | f(a) \in B \}$

\begin{proposition}
Given a function $f: X \rightarrow Y$, $f$ is surjective if and only if $\forall A \subseteq Y$, where $A \neq \emptyset$, $f^{-1}(A) \neq \emptyset$.
\end{proposition}

\begin{proposition}
Given a function $f: X \rightarrow Y$, $f$ is injective if and only if $\forall A \subseteq$ ran $f$, where $A$ is a singleton, $f^{-1}(A)$ is a singleton.
\end{proposition}

\begin{proposition}
Given a function $f : X \rightarrow Y$\dots
\begin{itemize}
  \item If $B \subseteq Y$, then $f(f^{-1}(B)) \subseteq B.$
  \item If $f$ is surjective, then $f(f^{-1}(B)) = B.$
  \item If $A \subseteq X$, then $A \subseteq f^{-1}(f(A)).$
  \item If $f$ is injective, then $A = f(f^{-1}(A)).$
  \item If $\{B_i\}$ is a family of subset of $Y$, then $f^{-1}(\bigcup_i B_i)=\bigcup_i f^{-1}(B_i)$ and $f^{-1}(\bigcap_i B_i)=\bigcap_i f^{-1}(B_i).$
\end{itemize}
\end{proposition}

\subsubsubsection{Function Composition}\label{functioncomposition}
$f:X \rightarrow Y \textrm{ and } g:Y \rightarrow Z \Rightarrow g \circ f:X \rightarrow Z \textrm{ where } \forall \, x \in X, \, g \circ f(x) := g(f(x))$

