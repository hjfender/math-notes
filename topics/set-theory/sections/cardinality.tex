\subsection{Cardinality}\label{cardinality}

\subsubsection{Equinumerosity}
Two sets $A$ and $B$ are \emph{equinumerous}, denoted $A \approx B$, if and only if there is a bijection $f : A \rightarrow B.$ \label{equinumerous}

\begin{proposition}
Equinumerosity is an equivalence relation. (See: \hyperref[isomorphism]{isomorphism})
\end{proposition}

\begin{theorem}[Diagonalization]\label{diagonalization}
The set $\omega$ is not equinumerous to the set $\mathbb{R}$ of real numbers.
\end{theorem}

\begin{proof}\renewcommand{\qedsymbol}{\Lightning}
Suppose for the sake of contradition that there is a bijection $f: \omega \rightarrow \mathbb{R}$. Thus we can imagine a list of successive values\dots
$$f(0) = 236.001\dots$$
$$f(1) = -7.777\dots$$
$$f(2) = 3.1415\dots$$
$$\vdots$$
Then consider the real number $0.a_1a_2a_3\dots$ where:
\[
	a_n = \begin{cases}
				7 & \textrm{ if the nth decimal of } f(n) \neq 7\\
				6 & \textrm{otherwise.}
		  \end{cases}
\]
This number cannot be in the range of $f$, so it is not a bijection.
\end{proof}

\begin{theorem}[Diagonalization]
No set is equinumerous to its power set.
\end{theorem}

\begin{proof}
Let $g: A \rightarrow \mathcal{P}(A)$. Consider\dots
$$B = \{ x \in A | x \not \in g(x) \}.$$
Then $B \subseteq A$, but for each $x \in A,$
$$x \in B \Leftrightarrow x \not\in g(x).$$
Hence $B \not\in$ ran $g$ and $g$ is not a bijection.
\end{proof}

\subsubsection{Finite/Infinite}
A set is \emph{finite}\label{infinite} if and only if it is equinumerous to some natural number. Otherwise it is \emph{infinite}\label{infinite}.

\begin{theorem}[Pigeonhole Principle]\label{pigeonholeprinciple}
No natural number is equinumerous to a proper subset of itself.
\end{theorem}

\begin{proof}
Suppose $f : N \rightarrow N$ is a bijection from a finite set to itself. We will show that ran $f$ is all of the set $n$. This suffices to prove the theorem.

We use the induction on $n$. Define:
$$T = \{ n \in \omega | \textrm{ every injection from } n \textrm{ into } n \textrm{ has range } n \}$$

We have that $0 \in T$; the only function from the set $0$ into the set $0$ is the empty function, which has range $0$. Now suppose that $k \in T$ and that $f$ is an injection from $k^+$ into $k+$. Note that 
the \hyperref[restriction]{restriction} $f \mathord{\upharpoonright}_k$ maps $k$ injectively into $k^+$. There are two cases\dots

Case I: The set $k$ is closed under $f$. Then $f \mathord{\upharpoonright}_k$ maps the set $k$ into the set $k$. Then because $k \in T$ we may conclude that ran $(f \mathord{\upharpoonright}_k) = k$. Since
$f$ is injective, the only possible value for $f(k)$ is the number $k$. Hence ran $f$ is $k \cup \{ k \},$ which is the set $k^+$.

Case II: Otherwise $f(p) = k$ for some number $p$ less than $k$. In this case we interchange two values of teh functino. Define $\hat{f}$ by\dots
$$\hat{f}(p) = f(k),$$
$$\hat{f}(k) = f(p) = k,$$
$$\hat{f}(x) = f(x) \textrm{ for other } x \in k^+.$$
The $\hat{f}$ maps the set $k^+$ injectively into the set $k^+$, and the set $k$ is closed under $\hat{f}$. So we can apply Case I.

Thus ran $f = k^+$.
\end{proof}

\begin{corollary}
No finite set is equinumerous to a proper subset of itself.
\end{corollary}

\begin{corollary}
Any set equinumerous to a proper subset of itself is infinite.
\end{corollary}

\begin{corollary}
The set $\omega$ is infinite.
\end{corollary}

\begin{corollary}
Any finite set is equinumerous to a unique natural number.
\end{corollary}

\begin{lemma}
If $C$ is a proper subset of a natural number $n$, the $C \approx m$ for some $m$ less than $n$.
\end{lemma}

\begin{corollary}
Any subset of a finite set if finite.
\end{corollary}

\subsubsection{Cardinal Numbers}\label{cardinalnumbers}

For any set $A$, the \hyperref[ordinalnumbers]{cardinal number} of $A$, denoted card $A$, is a set\dots
\begin{enumerate}
  \item For any sets $A,B$\dots
  	$$\textrm{card } A = \textrm{card }B \Leftrightarrow A \approx B.$$
  \item For a finite set $A$, card $A$ is the natural number $n$ for which $A \approx n$.
\end{enumerate}

(See: \hyperref[cardinalnumberdefinition]{cardinal number definition using ordinals})

\subsubsubsection{Cardinal Arithmetic}
Let $\kappa$ and $\lambda$ be any cardinal numbers.
\begin{itemize}
  \item $\kappa + \lambda =$ card$(K \cup L)$, where $K$ and $L$ are any disjoint sets of cardinality $\kappa$ and $\lambda$, respectively.
  \item $\kappa \cdot \lambda =$ card$(K \times L)$, where $K$ and $L$ are any sets of cardinality $\kappa$ and $\lambda$, respectively.
  \item $\kappa^{\lambda} =$ card$^LK$, where $K$ and $L$ are any sets of cardinality $\kappa$ and $\lambda$, respectively.
\end{itemize}

\begin{proposition}
Assume that $K_1 \approx K_2$ and $L_1 \approx L_2$.
\begin{enumerate}
  \item If $K_1 \cap L_1 = K_2 \cap L_2 = \emptyset$, then $K_1 \cup L_1 \approx K_2 \cup L_2$.
  \item $K_1 \times L_1 \approx K_2 \times L_2.$
  \item $^{L_1}K_1 \approx ^{L_2}K_2.$
\end{enumerate}
\end{proposition}

\begin{proposition}
For any cardinal numbers $\kappa,\lambda,$ and $\mu$\dots
\begin{itemize}
  \item $\kappa + \lambda = \lambda + \kappa$ and $\kappa \cdot \lambda = \lambda \cdot \kappa$.
  \item $\kappa + (\lambda + \mu) = (\kappa + \lambda) + \mu$ and $\kappa \cdot (\lambda \cdot \mu) = (\kappa \cdot \lambda) \cdot \mu$.
  \item $\kappa \cdot (\lambda + \mu) = \kappa \cdot \lambda + \kappa \cdot \mu$.
  \item $\kappa^{\lambda + \mu} = \kappa^{\lambda} \cdot \kappa^{\mu}$.
  \item $(\kappa \cdot \lambda)^{\mu} = \kappa^{\mu} \cdot \lambda^{\mu}$.
  \item $(\kappa^{\lambda})^{\mu} = \kappa^{\lambda \cdot \mu}$
\end{itemize}
\end{proposition}

\begin{proposition}
Let $m$ and $n$ be finite cardinals. Then\dots
\begin{itemize}
	\item $m + n = m +_{\omega} n$
	\item $m \cdot n = m \cdot_{\omega} n$
	\item $m^n = m^n$
\end{itemize}
(See: \hyperref[arithmetic]{natural number arithmetic}.)
\end{proposition}

\begin{corollary}
If $A$ and $B$ are finite, then $A \cup B$, $A \times B$, and $^BA$ are also finite.
\end{corollary}

\subsubsubsection{Ordering Cardinal Numbers}\label{cardinalordering}
A set $A$ is \emph{dominated}\label{dominated} by a set $B$ (written $A \preceq B$) if and only if there is an injective function from $A$ into $B$.

\begin{theorem}[Schr\"oder-Bernstein Theorem]\label{schroderbernsteinthm}
If $A \preceq B$ and $B \preceq A$, then $A \approx B$.
\end{theorem}

\begin{proof}
The proof is accomplished with mirrors. Given injections $f : A \rightarrow B$ and $g: B \rightarrow A$. Define $C_n$ by recursion, using the formulas
$$C_0 = A \setminus \textrm{ran } g \; \; \textrm{ and } \; \; C_{n^+} = g[f[C_n]].$$
Thus $C_0$ is the troublesome part that keeps $g$ from being a bijection. We bounce it back and forth, obtaining $C_1, C_2, \dots$. This function showing that $A \approx B$ is the function $h: A \rightarrow B$ defined by\dots
\[
	h(x) = \begin{cases}
				f(x) & \textrm{ if } x\in C_n \textrm{ for some } n,\\
				g^{-1}(x) & \textrm{otherwise.}
		   \end{cases}
\]
Note that in the second case $(x \in A$ but $x \not\in C_n$ for any $n)$ it follows that $x \not \in C_0$ and hence $x \in$ ran $g$. So $g^{-1}(x)$ makes sense in this case.
We verify that $h$ is indeed a bijection. Define $D_n = f[C_n]$, so that $C_{n^+} = g[D_n]$. Consider distinct $x,y \in A$. Since both $f$ abd $g^{-1}$ are injective, the only possible problem arises when, say, $x \in C_m$
and $y \in \bigcup_{n \in \omega} C_n$. In this case,
$$h(x) = f(x) \in D_m,$$
whereas,
$$h(y) = g^{-1}(y) \not \in D_m,$$
lest $y \in C_{m^+}$. So $h(x) \neq h(x')$, showing $h$ is injective.

Finally, we show $h$ is surjective. Certainly each $D_n \subseteq$ ran $h$, because $D_n = h[C_n]$. Consider then a point $y$ in $B \setminus \bigcup_{n \in \omega} D_n$. Where is $g(y)$? Certainly $g(y) \not\in C_0$. Also $g(y) \not\in C_{n^+}$,
because $C_{n^+} = g[D_n]$, $y \not\in  D_n$, and $g$ is injective. So $g(y) \not\in C_n$ for any $n$. Therefore $h(g(y))=g^{-1}(g(y)) = y.$ This shows that $y \in$ ran $h$, thereby proving part (a).
\end{proof}

\begin{theorem}[Restated Schr\"oder-Bernstein Theorem]
For cardinal numbers $\kappa$ and $\lambda$, if $\kappa \leq \lambda$ and $\lambda \leq \kappa$, then $\kappa = \lambda$.
\end{theorem}

\begin{proposition}
Let $\kappa, \lambda$ and $\mu$ be cardinal numbers.
\begin{itemize}
  \item $\kappa \leq \lambda \Rightarrow \kappa + \mu \leq \lambda + \mu$
  \item $\kappa \leq \lambda \Rightarrow \kappa \cdot \mu \leq \lambda \cdot \mu$
  \item $\kappa \leq \lambda \Rightarrow \kappa^{\mu} \leq \lambda^{\mu}$
  \item $\kappa \leq \lambda \Rightarrow \mu^{\kappa} \leq \mu^{\lambda}$;	if not both $\kappa$ and $\mu$ equal zero.
\end{itemize}
\end{proposition}

\subsubsubsection{Infinite Cardinal Arithmetic}

\begin{lemma}
For any infinite cardinal $\kappa$, we have $\kappa \cdot \kappa = \kappa.$
\end{lemma}

\begin{theorem}[Absorption Law of Cardinal Arithmetic]
Let $\kappa$ and $\lambda$ be cardinal numbers, the larger of which is infinite and the smaller of which is nonzero. Then\dots
$$\kappa + \lambda = \kappa \cdot \lambda = \textrm{max}(\kappa,\lambda).$$
\end{theorem}