\subsection{Homomorphisms of Abelian Groups}\label{abeliangrouphomomorphism}

\begin{proposition}
\label{homabgroups}
For any two abelian groups $G,H$, Hom$_{Ab}(G,H)$ is an abelian group under addition inherited from $H$.
\end{proposition}

\begin{proof}
Define the operation $\varphi + \psi$ for $\varphi, \psi \in$ Hom$_{Ab}(G,H)$, where\dots
$$(\varphi + \psi)(g) = \varphi(g) +_H \psi(g).$$
Observe that $\varphi + \psi$ is a homomorphism\dots
\begin{align*}
	(\varphi + \psi)(a +_G b) &= \varphi(a +_G b) + \psi(a +_G b) = (\varphi(a) +_H \varphi(b)) +_H (\psi(a) +_H \psi(b))\\
							  &\overset{!}{=} (\varphi(a) +_H \psi(a)) +_H (\varphi(b) +_H \psi(b)) = (\varphi + \psi)(a) +_H (\varphi + \psi)(b)
\end{align*}

\noindent From here it is easy to show that Hom$_{Ab}(G,H)$ is an abelian group.
\end{proof}

\noindent Note: By the same logic, if $A$ is a set and $H$ an abelian group, then $H^A$ is an abelian group.\newline
 
\noindent In fact, by adding the additional operation $\circ$ (treated as multiplication), we transform End$_{Ab}(G) \; :=$ Hom$_{Ab}(G,G)$
into a \hyperref[ringdefinition]{ring}.

\begin{proposition}
End$_{Ab}(\mathbb{Z}) \cong \mathbb{Z}$ as rings.
\end{proposition}

\begin{proof}
Consider the function\dots
$$\varphi : \textrm{End}_{Ab}(\mathbb{Z}) \rightarrow \mathbb{Z}$$
defined by\dots
$$\varphi(\alpha) = \alpha(1)$$
for all group homomorphisms $\alpha: \mathbb{Z} \rightarrow \mathbb{Z}$. Then $\varphi$ is a group
homomorphism: the addition in End$_{Ab}(\mathbb{Z})$ is defined so that $\forall n \in \mathbb{Z}$\dots
$$(\alpha + \beta)(n) = \alpha(n) + \beta(n);$$
in particular\dots
$$\varphi(\alpha + \beta) = (\alpha + \beta)(1) = \alpha(1) + \beta(1) = \varphi(\alpha) + \varphi(\beta).$$
Further, $\varphi$ is a ring homomophism. Indeed, for $\alpha, \beta \in \textrm{End}_{Ab}(\mathbb{Z})$ denote
$\alpha(1)$ by $a$; then\dots
$$\alpha(n) = n\alpha(1) = na = an$$
for all $n \in \mathbb{Z}$; in particular,
$$\alpha(\beta(1)) = a\beta(1) = \alpha(1)\beta(1).$$
Therefore,
$$\varphi(\alpha \circ \beta) = (\alpha \circ \beta)(1) = \alpha(\beta(1)) = \alpha(1)\beta(1) = \varphi(\alpha)\varphi(\beta)$$
as needed. Also, $\varphi(\textrm{id}_{\mathbb{Z}}) = id_{\mathbb{Z}}(1) = 1.$

Finally, $\varphi$ has an inverse: for $a \in \mathbb{Z}$, the $\psi(a)$ be the homomorphism $\alpha : \mathbb{Z} \rightarrow \mathbb{Z}$
defined by\dots
$$(\forall n \in \mathbb{Z}): \; \alpha_a(n) = an.$$

This inverse is a ring homomorphism\dots
\begin{itemize}
  \item $\psi(a+b) = \alpha_{a+b} = \alpha_{a} + \alpha_{b} = \psi(a) + \psi(b)$;
  \item $\psi(a \cdot b) = \alpha_{a \cdot b} = \alpha_{a} \circ \alpha_{b} = \psi(a) \circ \psi(b)$;
  \item $\psi(1) = \alpha_1 = \textrm{id}_{\mathbb{Z}}$.
\end{itemize}
\end{proof}

\begin{proposition}
Let $R$ be a ring. Then the function $r \mapsto \lambda_r$ is an injective ring homomorphism\dots
$$\lambda : R \rightarrow \textrm{End}_{Ab}(R).$$
\end{proposition}

\begin{proof}
For any $r \in R$ and for all $a,b \in R$, distributivity gives\dots
$$\lambda_r(a+b) = r(a+b) = ra + rb = \lambda_r(a) + \lambda_r(b):$$
this shows that $\lambda_r$ is indeed an endomorphism of the \emph{group} $\langle R,+ \rangle$, that is,
$\lambda_r \in \textrm{End}_{Ab}(R).$

The function $\lambda : R \rightarrow \textrm{End}_{Ab}(R)$ defined by the assignment $r \mapsto \lambda_r$
is clearly injective, since $r \neq s$, then\dots
$$\lambda_r(1) = r \neq s = \lambda_s(1),$$
so that $\lambda_r \neq \lambda_s$.

Now we show that $\lambda$ is a homomorphism. Additive preservation follows from \ref{homabgroups} and distributivity.
Associativity can be used to show that multiplication is preserved. The identity is clearly preserved.
\end{proof}