\subsection{Classification of Finite Abelian Groups}\label{classificationoffiniteabeliangroups}

\begin{lemma}
Let $G$ be an abelian group, and let $H$, $K$ be subgroups such that $|H|$, $|K|$ are relatively prime. Then $H + K \cong H \oplus K$. 
\end{lemma}

\begin{proof}
By Lagrange's theorem, $H \cap K = \{ 0 \}$. Since subgroups of abelian groups are automatically normal, the statement follows from \ref{whensubgroupsmultiply}.
\end{proof}

\begin{corollary}
\label{abeliansylowgroups}
Every finite abelian group is the direct sum of its nontrivial Sylow subgroups.
\end{corollary}

\begin{lemma}
\label{elementoforderpinabeliangroup}
Let $p$ be a prime integer and $r \geq 1$. Let $G$ be a noncyclic abelian group of order $p^{r+1}$, and let $g \in G$ be an element of order $p^r$. Then there exists an element $h \in G$,
$h \not \in \langle g \rangle$, such that $|h| = p$.
\end{lemma}

\begin{proof}
Denote $\langle g \rangle$ by $K$, and let $h'$ be any element of $G$, $h' \not \in K$. The subgroup $K$ is normal in $G$ since $G$ is abelian; the quotient group $G/K$ has order $p$.
Since $h' \not \in K$, the coset $h' + K$ has order $p$ in $G/K$; that is, $ph' \in K$. Let $k = ph'.$

Note that $|k|$ divides $p^r$; hence it is a power of $p$. Also $|k| \neq p^r$, otherwise $|h'| = p^{r+1}$ and $G$ would be cyclic, contrary to the hypothesis.

Therefore $|k| = p^s$ for some $s < r$; $k$ generates a subgroup $\langle k \rangle$ of the cyclic group $K$, or order $p^s$. By \ref{subgroupsofcyclicgroups}, $\langle k \rangle = \langle p^{r-s}r \rangle$.
Since $s < r$, $\langle k \rangle \subseteq \langle pg \rangle$; thus, $k = mpg$ for some $m \in \mathbb{Z}$.

Then let $h = h' - mg$: $h \neq 0$ (since $h' \not \in K$), and\dots
$$ph = ph' - p(mg) - k -k =0,$$
showing that $|h| = p$, as stated.
\end{proof}

\begin{lemma}
\label{sequenceabeliangroups}
Let $G$ be an abelian $p$-group, let $g \in G$ be an element of maximal order. Then the exact sequence\dots
$$0 \rightarrow \langle g \rangle \rightarrow G \rightarrow G / \langle g \rangle \rightarrow 0$$
splits.
\end{lemma}

\begin{proof}
Argue by induction on the order of $G$; the case $|G| = p^0 = 1$ requires no proof. Thus we will assume that $G$ is nontrivial and that the statement is true for every $p$-group smaller that $G$.

Let $g \in G$ be an element of maximal order, say $p^r$, and denote by $K$ the subgroup $\langle g \rangle$ generated by $g$; this subgroup is normal, as $G$ is abelian. If $G = K$, then the statement
holds trivially. If not, $G/K$ is a nontrivial $p$-group, and hence it contains an element of order $p$ by Cauchy's theorem. This element generates a subgroup of order $p$ in $G/K$, corresponding to a subgroup $G'$
of $G$ of order $p^{r+1}$, containing $K$. This subgroup is not cyclic (otherwise the order of $g$ is not maximal).

That is, we are in the situation of \ref{preimagesubgroup}: hence we can conclude that there is an element $h \in G'$ (and hence $h \in G$) with $h \not \in K$ and $|h| = p$. Let $H$ = $\langle h \rangle \subseteq G$
be the subgroup generated by $h$, and note that $K \cap H = \{ 0 \}$.

Now work modulo $H$. The quotient group $G/H$ has smaller size that $G$, and $g + H$ generates a cyclic subgroup $K' = (K + H)/H \cong K/(K \cap H) \cong K$ of maxiaml order in $G/H$. By the induction hypothesis, there
is a subgroup $L'$ of $G/H$ such that $K' + L' = G/H$ and $K' \cap L' = \{ 0_{G/H} \}$. This subgroup $L'$ corresponds to a subgroup $L$ of $G$ containing $H$.

Now: (i) $K + L = G$ and (ii) $K \cap L = \{ 0 \}$. Indeed, we have the following:
(i) For any $a \in G$, there exist $mg + H \in K'$, $l + H \in L'$ such that $a + H = mg + l + H$ (since $K' + L' = G/H$). This implies $a - mg \in L$, and hence $a \in K + L$ as needed.
(ii) If $a \in K \cap L$, then $a + H \in K' \cap L' = \{ 0_{G/H} \}$, and hence $a \in H$. In particular, $a \in K \cap H = \{ 0 \}$, forcing $a = 0$, as needed.

(i) and (ii) imply the lemma, as observed in the comments following the statement.
\end{proof}

\begin{corollary}
Let $G$ be a finite abelian group. Then $G$ is a direct sum of cyclic groups, which may be assumed to be cyclic $p$-groups.
\end{corollary}

\begin{proof}
As noted in \ref{abeliansylowgroups}, $G$ is a direct sum of $p$-groups (as a consequence of the Sylow theorems). I claim that every abelian $p$-group $P$ is a direct sum of cyclic $p$-groups.

To establish this, argue by induction on $[P]$. There is nothing to prove if $P$ is trivial. If $P$ is not trivial, let $g$ be an element of $P$ of maximal order. By the previous lemma
$$P =\langle g \rangle \oplus P'$$
for some subgroup $P'$ of $P$; by the induction hypothesis $P'$ is a direct sum of cyclic $p$-groups, concluding the proof.
\end{proof}

\noindent Restated more precisely (and more famously) we have\dots

\begin{theorem}[Classification of Finite Abelian Groups]
\label{finiteabelianclassificationtheorem}
Let $G$ be a finite nontrivial abelian group. Then\dots
\begin{itemize}
  \item there exist prime integers $p_1, \dots, p_r$ and positive integers $n_{ij}$ such that $|G| = \prod_{i,j} p_i^{n_{ij}}$ and\dots
  $$G \cong \bigoplus_{i,j} \frac{\mathbb{Z}}{p_i^{n_{ij}}\mathbb{Z}}$$
  \item there exist positive integers $1 < d_1 | \cdots | d_s$ such that $|G| = d_1 \dots d_s$ and
  $$G \cong \frac{\mathbb{Z}}{d_1\mathbb{Z}}\oplus \cdots \oplus \frac{\mathbb{Z}}{d_s\mathbb{Z}}.$$
  Further, these decompositions are uniquely determined by $G$.
\end{itemize}
\end{theorem}