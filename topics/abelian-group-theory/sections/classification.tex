\subsection{Classification of Finite Abelian Groups}\label{classificationoffiniteabeliangroups}

\begin{lemma}
Let $G$ be an abelian group, and let $H$, $K$ be subgropus such that $|H|$, $|K|$ are relatively prime. Then $H + K \cong H \oplus K$. 
\end{lemma}

\begin{proof}
By Lagrange's theorem, $H \cap K = \{ 0 \}$. Since subgroups of abelian groups are automatically normal, the statement follows from \ref{whensubgroupsmultiply}.
\end{proof}

\begin{corollary}
Every finite abelian group is the direct sum of its nontrivial Sylow subgroups.
\end{corollary}

\begin{lemma}
\label{elementoforderpinabeliangroup}
Let $p$ be a prime integer and $r \geq 1$. Let $G$ be a noncyclic abelian group of order $p^{r+1}$, and let $g \in G$ be an element of order $p^r$. Then ther exists an element $h \in G$,
$h \not \in \langle g \rangle$, such that $|h| = p$.
\end{lemma}

\begin{proof}
Denote $\langle g \rangle$ by $K$, and let $h'$ be any element of $G$, $h' \not \in K$. The subgroup $K$ is normal in $G$ since $G$ is abelian; the quotient group $G/K$ has order $p$.
Since $h' \not \in K$, the coset $h' + K$ has order $p$ in $G/K$; that is, $ph' \in K$. Let $k = ph'.$

Note that $|k|$ divides $p^r$; hence it is a power of $p$. Also $|k| \neq p^r$, otherwise $|h'| = p^{r+1}$ and $G$ would be cyclic, contrary to the hypothesis.

Therefore $|k| = p^s$ for some $s < r$; $k$ generates a subgroup $\langle k \rangle$ of the cyclic group $K$, or order $p^s$. By \ref{subgroupsofcyclicgroups}, $\langle k \rangle = \langle p^{r-s}r \rangle$.
Since $s < r$, $\langle k \rangle \subseteq \langle pg \rangle$; thus, $k = mpg$ for some $m \in \mathbb{Z}$.

Then let $h = h' - mg$: $h \neq 0$ (since $h' \not \in K$), and\dots
$$ph = ph' - p(mg) - k -k =0,$$
showing that $|h| = p$, as stated.
\end{proof}

\begin{lemma}
\label{sequenceabeliangroups}
Let $G$ be an abelian $p$-group, let $g \in G$ be an element of maximal order. Then teh exact sequence\dots
$$0 \rightarrow \langle g \rangle \rightarrow G \rightarrow G / \langle g \rangle \rightarrow 0$$
splits.
\end{lemma}

\begin{proof}
Argue by induction on the order of $G$; the case $|G| = p^0 = 1$ requires no proof. Thus we will assume that $G$ is nontrivial and that the statement is true for every $p$-group smaller that $G$.

Let $g \in G$ be an element of maximal order, say $p^r$, and denote by $K$ the subgroupr $\langle g \rangle$ generated by $g$;
\end{proof}