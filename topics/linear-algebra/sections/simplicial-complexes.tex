\subsection{Simplicial Complexes}

\subsubsection{General Position}\label{generalposition}
A set of vectors $S = \{ v_0, \dots, v_n \}$ in $\mathbb{R}^N$ for $N$ large is in \emph{general position}
if the set $\{ v_0 - v_n, v_1 - v_n, \dots, v_{n-1} - v_n \}$ is linearly independent.

\subsubsection{Simplex}\label{simplex}
A set of vectors $\{ v_0, \dots, v_n \}$ in general position is called an $n$-\emph{simplex}. It determines the following subset of $\mathbb{R}^N$\dots
$$\Delta^n[S] = \{t_0 v_0 + t_1 v_1 + \cdots + t_n v_n \in \mathbb{R}^N | t_i \geq 0, \, t_0 + \cdots + t_n = 1 \}$$
called the convex hull of $\{ v_0, \dots, v_n \}$.

\subsubsection{Barycentric Coordinantes}\label{barycentricxy}
The \emph{barycentric coordinates} of $p \in \Delta^n$, an $n$-simplex, is the list of coefficients $(t_0, t_1, \dots, t_n)$
such that\dots
$$p = t_0 v_0 + \cdots + t_n v_n.$$

\begin{proposition}
Let $\Delta^n$ denote the subspace of $\mathbb{R}^{n+1}$ given by $\Delta^n = \{ (t_0, \dots, t_n) \in \mathbb{R}^{n+1} | t_0 + \cdots + t_n = 1, t_i \geq 0 \}.$
If $S = \{ v_0, \dots, v_n \}$ is a set of vectors in general position in $\mathbb{R}^N$, then $\Delta^n[S]$ is homeomorphic to $\Delta^n$.
\end{proposition}

\begin{proposition}
The points $p \in \Delta^n[S]$ with barycentric coordinates that satisfy $t_i > 0$ for all $i$ form an open subset of $\Delta^n[S]$ (as a subspace of $\mathbb{R}^N$);
$p$ is in the boundary of $\Delta^n[S]$ if and only if $t_i = 0$ for some $i$.
\end{proposition}

\subsubsection{Geometric Simplicial Complex}\label{geometricsimplicialcomplex}
A \emph{geometric simplicial complex} is a finite collection $K$ of simplices in $\mathbb{R}^N$ satisfying\dots
\begin{enumerate}
  \item if $S = \{ v_0, \dots, v_n \}$ in in $K$ and $T \prec S$, then $T$ is also in $K$;
  \item for $S$ and $T$ in $K$, if $\Delta^n[S]\cap \Delta^m[T] \neq \emptyset$, then $\Delta^n[S] \cap \Delta^m[T] = \Delta^k[U]$ for some $U$ in $K$ (intersect along a common face)
\end{enumerate}
The dimension of one of these complexes is the largest $n$ for which there is an $n$-simplex in $K$.

\subsubsection{Realization of a Geometric Simplicial Complex}\label{realization}
$$|K| = \bigcup_{S \in K} \Delta^n[S] \subseteq \mathbb{R}^N.$$
Called the \label{realization} of $K$, or the underlying space of $K$, the geometric carrier of $K$, or the polyhedron determined by $K$.

\subsubsection{Abstract Simplicial Complex}\label{abstractsimplicialcomplex}
A finite collection of sets\dots
$$L = \{ S_{\alpha} | S_{\alpha} = \{ v_{\alpha 0}, \dots, v_{\alpha n_{\alpha}}, 1 \leq \alpha \leq N \} \}$$
is an \emph{abstract simplicial complex} if whenever $T = \{ v_{j0}, \dots, v_{jk} \}$ is a subset of $S$ and $S$ is in $L$, then $T$ is also in $L$.

\subsubsection{Triangulable}\label{triangulable}
A topological space $X$ is said to be \emph{triangulable} if there is an abstract simplicial complex $K$ and a homeomorphism $f : X \rightarrow |K|$.