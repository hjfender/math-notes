\subsection{Finite Subgroups of Multiplicative Groups of Fields}\label{subgroupsofmultgrpinfld}

\begin{lemma}
Let $G$ be a finite abelian group, and assume that for every integer $n > 0$ the number of elements $g \in G$ such that $ng = 0$ is an most $n$. Then $G$ is cyclic.
\end{lemma}

\begin{proof}
By \ref{finiteabelianclassificationtheorem}\dots
$$G \cong \frac{\mathbb{Z}}{d_1\mathbb{Z}}\oplus \cdots \oplus \frac{\mathbb{Z}}{d_s\mathbb{Z}}$$
for some positive integers $1 < d_1 | \cdots | d_s$. But if $s > 1$, Then $|G| > d_s$ and $d_sg = 0$ for all $g \in G$ (so that the order of $g$ divides $d_s$), contradicting the hypotheses. Therefore $s = 1$; that is, $G$ is cyclic.
\end{proof}

\begin{proposition}
Let $F$ be a field, and let $G$ be a finite subgroup of the multiplicative group $(F^*, \cdot)$. Then $G$ is cyclic.
\end{proposition}

\begin{proof}
A polynomial $f(x) \in F[x]$ is divisible by $(x-a)$ if and only if $f(a) = 0$; since a nonzero polynomial of degree $n$ over a field can have as most
$n$ linear factors, this shows that if $f(x) \in F[x]$ has degree $n$, then $f(a) = 0$ for at most $n$ distinct elements $a \in F$.
Thus, for every $n$ there are at most $n$ elements $a \in F$ such that $a^n - 1 = 0$, that is, at most $n$ elements $a \in G$ such that $a^n = 1$.
The preceding lemma implies then that $G$ is cyclic.
\end{proof}