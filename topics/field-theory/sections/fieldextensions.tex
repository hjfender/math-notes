\subsection{Field Extensions}\label{fieldextensions}

\begin{proposition}
Let $k$ be a field, and let $f(t) \in k[t]$ be a nonzero irreducible polynomial. Then\dots
$$F := \frac{k[t]}{(f(t))}$$
is a field, endowed with a natural homomorphism $i : k \hookrightarrow F$ (obtained as the composition $k \rightarrow k[x] \rightarrow F$)
realizing it as an extension of $k$. Further,
\begin{itemize}
  \item $f(x) \in k[x] \subseteq F[x]$ has a root in $F$, namely the coset of $t$;
  \item if $k \subseteq K$ is any extension in which $f$ has a root, then there exists a homomorphism $j : F \rightarrow K$ such that the diagram
  \begin{figure}[H]
  \centering
  \begin{tikzcd}
k \arrow[rr, hook] \arrow[rd, hook, "i"] & & K\\
 & F \arrow[ur, hook, "j"] &
\end{tikzcd}
  \end{figure}
  commutes.
\end{itemize}
\end{proposition}

\noindent A field extension $k \subseteq F$ is \emph{finite}, of \emph{degree} $n$, if $F$ has
(finite) dimension dim $F = n$ as a vector space over $k$. The extension is \emph{infinite} otherwise.
The degree is denoted $[F:k]$.

\subsubsection{Simple Extensions}\label{simpleextensions}

A field extension $k \subseteq F$ is \emph{simple} is there exists an element $\alpha \in F$ such that
$F = k(\alpha)$, the smallest subfield of $F$ containing both $k$ and $\alpha$.

\begin{proposition}
Let $k \subseteq k(\alpha)$ be a simple extension. Consider the evaluation map $\epsilon : k[t] \rightarrow k(\alpha)$,
defined by $f(t) \rightarrow f(\alpha)$. Then we have the following:
\begin{itemize}
  \item $\epsilon$ is injective if and only if $k \subseteq k(\alpha)$ is an infinite extension. In this case,
  $k(\alpha)$ is isomorphic to the field of rational functions $k(t)$
  \item $\epsilon$ is not injective if and only if $k \subseteq k(\alpha)$ is finite. In this case, there exists
  a unique monic irreducible nonconstant polynomial $p(t) \in k[t]$ of degree $n = [k(\alpha) : k]$ such that\dots
  $$k(\alpha) \cong \frac{k[t]}{(p(t))}.$$
  Via this isomorphism, $\alpha$ corresponds to the coset of $t$. The polynomial $p(t)$ is the monic polynomial
  of smallest degree in $k[t]$ such that $p(\alpha) = 0$ in $k(\alpha)$.
\end{itemize}
\end{proposition}

\begin{proof}
Apply the first isomorphism theorem and examine the kernel in each case. The bit about the field of fractions
is to account for the fact that $k[t]$ is an integral domain and not (quite) a field.
\end{proof}

\begin{proposition}
\label{extensionofextension}
Let $k_1 \subseteq F_1 = k_1(\alpha_1)$, $k_2 \subseteq F_2 = k_2(\alpha_2)$ be two finite simple
extensions. Let $p_1(t) \in k_1[t]$, resp., $p_2(t) \in k_2[t]$, be the minimal polynomials of $\alpha_1$,
resp., $\alpha_2$. Let $i : k_1 \rightarrow k_2$ be an isomorphism, such that\dots
$$i(p_1(t)) = p_2(t).$$
Then there exists a unique isomorphism $j : F_1 \rightarrow F_2$ agreeing with $i$ on $k_1$ and such that
$j(\alpha_1) = \alpha_2$.
\end{proposition}

\subsubsection{Group of Automorphisms of an Extension}\label{grpautomorphismext}

Let $k \subseteq F$ be a field extension. The \emph{group of automorphisms} of the extension,
denoted Aut$_k(F)$, is the group of field automorphisms $j : F \rightarrow F$ such that $j \upharpoonright k = \textrm{id}_k$.

\begin{corollary}
Let $k \subseteq F = k(\alpha)$ be a simple finite extension, and let $p(x)$ be the minimal polynomial of $\alpha$
over $k$. Then $|\textrm{Aut}_k(F)|$ equals the number of distinct roots of $p(x)$ in $F$, in particular,
$$|\textrm{Aut}_k(F)| \leq [F : k],$$
wiht equality if and only if $p(x)$ factors over $F$ as a product of distinct linear polynomials.
\end{corollary}

\subsubsection{Algebraic Extensions}\label{algebraicextension}
Let $k \subseteq F$ be a field extension, and let $\alpha \in F$. Then $\alpha$ is \emph{algebraic} over $k$,
\emph{of degree} $n$ if $n = [k(\alpha) : k]$ is finite; $\alpha$ is \emph{transcendental} over $k$ otherwise.\newline

\noindent The extension $k \subseteq F$ is \emph{algebraic} if every $\alpha \in F$ is algebraic over $k$.

\begin{lemma}
Let $k \subseteq F$ be a finite extension. Then every $\alpha \in F$ is algebraic over $k$, of degree $\leq [F:k]$.
\end{lemma}

\begin{proposition}
Let $k \subseteq E \subseteq F$ be field extensions. Then $k \subseteq F$ is finite if and only if both $k\subseteq E$ and $E \subseteq F$
are finite. In this case,
$$[F:k] = [F:E][E:k].$$
\end{proposition}

\begin{corollary}
Let $k \subseteq F$ be a finite extension, and let $E$ be an intermediate field (that is, $k \subseteq E \subseteq F$).
Then both $[E : k]$ and $[F : E]$ divide $[F : k]$.
\end{corollary}

\subsubsection{Finitely Generated Extension}\label{finitelygenext}
A field extension $k \subseteq F$ is \emph{finitely generated} if there exist $\alpha_1, \dots, \alpha_n \in F$ such that\dots
$$F = k(\alpha_1)(\alpha_2)\dots(\alpha_n) = k(\alpha_1,\alpha_2,\dots,\alpha_n).$$

\begin{proposition}
Let $k \subseteq F = k(\alpha_1,\alpha_2,\dots,\alpha_n)$ be a finitely generated field extension. Then the
following are equivalent:
\begin{enumerate}
\item $k \subseteq F$ is a finite extension
\item $k \subseteq F$ is an algebraic extension
\item Each $\alpha_i$ is algebraic over $k$.
\end{enumerate}
If these conditions are satisfied, then $[F : k] \leq$ the product of the degrees of $\alpha_i$ over $k$.
\end{proposition}

\begin{corollary}
Let $k \subseteq F$ be a field extension. Let\dots
$$E = \{ \alpha \in F | \alpha \textrm{ is algebraic over } k \}.$$
Then $E$ is a field.
\end{corollary}

\begin{corollary}
Let $k \subseteq E \subseteq F$ be field extensions. Then $k \subseteq F$ is algebraic if
and only if both $k \subseteq F$ and $E \subseteq F$ are algebraic.
\end{corollary}

\subsubsection{Algebraic Closure}\label{algebraicclosure}
\begin{lemma}
For a field $K$, the following are equivalent:
\begin{itemize}
  \item $K$ is algebraically closed.
  \item $K$ has no nontrivial extensions.
  \item If $K \subseteq L$ is any extension and $\alpha \in L$ is algebraic over $K$, then $\alpha \in K$.
\end{itemize}
\end{lemma}

\noindent An \emph{algebraic closure} of a field $k$ is an algebraic extension $k \subseteq \overline{k}$ such that
$\overline{k}$ is algebraically closed.

\begin{theorem}
Every field $k$ admits an algebraic closure $k \subseteq \overline{k}$;
this extension is unique up to isomorphism.
\end{theorem}

\begin{proof}
The algebraic closure is constructed in the remaining theorems and observations
of this section.
\end{proof}

\noindent First we tackle existence\dots

\begin{lemma}
Let $k$ be a field. Then there exists an extension $k \subseteq K$ such that
every nonconstant polynomial $f(x) \in k[x]$ has at least one root in $K$.
\end{lemma}

\begin{proof}
(Emil Artin) Consider a set $\mathcal{I} = \{ t_f \}$ in bijection with the set of nonconstant monic polynomials $f(x) \in k[x]$, and let $k[\mathcal{I}]$
be the corresponding polynomial ring in all the indeterminates $t_f$. Let $I \subseteq k[\mathcal{I}]$ be the ideal generated by all polynomials $f(t_f)$.

Then $I$ is a proper ideal. Indeed, otherwise we could write\dots
$$1 = \sum^n_{i=1} a_i \cdot f_i(t_i),\tag{*}$$
where $a_i \in k[\mathcal{I}]$. I claim that this cannot be done: indeed, we can construct an extension $k \subseteq F$ where the polynomials $f_1(x), \dots, f_n(x)$
have roots $\alpha_1, \dots, \alpha_n$, respectively; view (*) as an identity in $F[\mathcal{I}]$, and plug in $t_{f_i} = \alpha_i$, obtaining\dots
$$1 = \sum^n_{i=1} a_i \cdot f_i(\alpha_i) = \sum^n_{i=1} a_i \cdot 0 = 0,$$
which is nonsense.

Since $I$ is proper, it is contained in a maximal ideal $\textbf{m}$ (Zorn's Lemma). Thus, we obtain a field extension\dots
$$k \subseteq K := \frac{k[\mathcal{I}]}{\textbf{m}};$$
by construction every nonconstant monic (and hence every nonconstant) polynomial $f(x)$ has a root in $K$,
namely the coset of $t_f$.
\end{proof}

Define the field $L$ as the union of the chain of field extensions\dots
$$k \subseteq K_1 \subseteq K_2 \subseteq K_3 \subseteq \dots$$
obtained by applying the previous lemma so that every root of every polynomial in $k[x]$ exists in some $K_m$.

\begin{proposition}
The field $L$ is algebraically closed.
\end{proposition}

\begin{proof}
If $f(x) \in L[x]$ is a nonconstant polynomial, then $f(x) \in K_i[x]$ for some $i$, hence $f(x)$ has a root in $K_{i+1} \subseteq L$.
That is, every nonconstant polynomial in $L[x]$ has a root in $L$, as needed.
\end{proof}

\begin{lemma}
Let $K \subseteq L$ be a field extension, with $L$ algebraically closed. Let\dots
$$\overline{k} := \{ \alpha \in L | \alpha \textrm{ is algebraic over } k \}.$$
Then $\overline{k}$ is an algebraic closure of $k$.
\end{lemma}

\noindent Last we tackle uniqueness\dots

\begin{lemma}
Let $k \subseteq L$ be a field extension, with $L$ algebraically closed. Let $K \subseteq F$ be any algebraic extension.
Then there exists a morphism of extensions $i : F \rightarrow L$.
\end{lemma}

\begin{proof}
This argument also relies on Zorn's lemma. Consider the set $Z$ of homomorphisms\dots
$$i_K : K \rightarrow L$$
where $K$ is an intermediate field, $k \subseteq K \subseteq F$, and $i_K$ restricts to the identity on $k$;
$Z$ is nonempty, since the extension $i_k : k \subseteq L$ defines an element of $Z$. We give a poset structure to $Z$
by defining\dots
$$i_K \succeq i_{K'}$$
if $K \subseteq K' \subseteq F$ and $i_{K'}$ restricts to $i_K$ on $K$. To verify that every chain $C$ in $Z$ has an upper bound in $Z$,
let $K_C$ be the union of the sources of all $i_K \in C$; if $\alpha \in K_C$, define $i_{K_C}(\alpha)$ to be $i_K(\alpha)$, where $i_K$ is any
element of $C$ such that $\alpha \in K$. This prescription is clearly independent of the chosen $K$ and defines a homomorphism $K_C \rightarrow L$
restricting to the identity on $k$. This homomorphism is an upper bound for $C$.

By Zorn's lemma, $Z$ admits a mzximal element $i_G$, corresponding to an intermediate field $k \subseteq G \subseteq F$. Let $H = i_G(G)$ be the image
of $G$ in $L$.

I claim that $G = F$: this will prove the statement, because it will imply that there is a homomorphism $i_F : F \rightarrow L$ extending the identity on $k$.

Arguing by contradiction, assume that there exists an $\alpha \in F \setminus G$, and consider the extension $G \subseteq G(\alpha)$. Since $\alpha \in F$ is
algebraic over $k$, it is algebraic over $G$; thus, it is a root of an irreducible polynomial $g(x) \in G[x]$. Consider the induced homomorphism\dots
$$i_G : G[x] \rightarrow H[x],$$
and let $h(x) = i_G(g(x))$. Then $h(x)$ is an irreducible polynomial over $H$, and it has a root $\beta$ in $L$. We are in the situation of \ref{extensionofextension},
thus $i_G$ lifts to an isomorphism\dots
$$i_{G(\alpha)} : G(\alpha) \rightarrow H(\beta) \subseteq L$$
sending $\alpha$ to $\beta$. THis contradicts the maximality of $i_G$; hence $G = F$, concluding the argument.
\end{proof}