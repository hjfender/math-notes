\subsection{Cyclic Groups}\label{cyclicgroups}

\subsubsection{Modular Arithmetic}\label{modular arithmetic}
Let $n \in \mathbb{Z}^+$. Consider the \hyperref[equivalencerelation]{equivalence relation} on $\mathbb{Z}$ defined by\dots
$$a \equiv b \textrm{ mod } n \Leftrightarrow n | (b-a) \Leftrightarrow b-a \in n\mathbb{Z}.$$
It is called \emph{congruence modulo} $n$.

\subsubsection{Definition}
Let $\mathbb{Z} / n \mathbb{Z} = \{[z]_{\textrm{mod } n} | z \in \mathbb{Z} \}.$

\begin{lemma}
Addition $([a]_n + [b]_n := [a+b]_n)$ is well defined on $\mathbb{Z} / n \mathbb{Z}$.
\end{lemma}

\noindent Thus $C_n := \langle \mathbb{Z} / n \mathbb{Z}, + \rangle$ is a \emph{finite cyclic group}. We take $\langle \mathbb{Z},+ \rangle$ to be the \emph{infinite cyclic group}.

\begin{proposition}
The order of $[m]_n$ in $\mathbb{Z} / n \mathbb{Z}$ is $1$ if $n | m$, and more generally\dots
$$|[m]_n| = \frac{n}{\textrm{gcd}(m,n)}.$$
\end{proposition}

\begin{proof}
If $n | m$, then $[m]_n = [0]_n$. If $n \cancel{|} m$, $[m]_n = m[1]_n$ and apply \ref{orderofmultipleofelement}.
\end{proof}

\begin{corollary}
The class $[m]_n$ generates $\mathbb{Z} / n \mathbb{Z}$ if and only if gcd$(m,n) = 1$.
\end{corollary}

\noindent The \emph{cyclic groups} are an isomorphism class. Explicitly\dots
\begin{center}
A group $G$ is \emph{cyclic} if it is isomorphic to $\mathbb{Z}$ or $C_n$

for some positive interger $n$.
\end{center}

\begin{proposition}
If $|G| = p$ is a prime integer, then necessarily $G \cong \mathbb{Z} / p \mathbb{Z}$.
\end{proposition}

\begin{proof}
Use Lagrange's theorem (\ref{elementorderdividesgrouporder}).
\end{proof}

\subsubsection{Presentation}
We say that a group is \emph{cyclic} when it is generated by exactly one of its elements.

\noindent Finite: $\langle x | x^n \rangle$\newline

\noindent Infinite: $\langle x \rangle$

\subsubsection{Subgroups}

\begin{proposition}
Let $G \subseteq \mathbb{Z}$ be a subgroup. Then $G = d\mathbb{Z}$ for some $d \geq 0$.
\end{proposition}

\begin{proof}
If $G = \{ 0 \}$, then $G = 0\mathbb{Z}$. If not, note that if $a \in G$ and $a < 0$, then $-a \in G$ and $-a > 0$.
We can then let $d$ be the \emph{smallest positive integer} in $G$ and $G = d\mathbb{Z}$.

The inclusion $d\mathbb{Z} \subseteq G$ is clear. To verify $G \subseteq d\mathbb{Z}$, let $m \in G$, and apply 'division with remainder'
to write\dots

$$m = dq + r,$$

with $0 \leq r < d$. Since $m \in G$ and $d\mathbb{Z} \subseteq G$ and since $G$ is a subgroup, we see that\dots

$$r = m - dq \in G.$$

But $d$ is the smallest \emph{positive} integer in $G$, and $r \in G$ is smaller that $d$; so $r$ cannot be positive. This shows $r = 0$, that is,
$m = qd \in d\mathbb{Z}$; $G \subseteq d\mathbb{Z}$ follows.
\end{proof}

\begin{proposition}
\label{infinitecyclicsubgroup}
Let $n > 0$ be an integer and let $G \subseteq \mathbb{Z}/n\mathbb{Z}$. Then $G$ is the cyclic subgroup of $\mathbb{Z}/n\mathbb{Z}$ generate by $[d]_n$,
for some divisor $d$ of $n$.
\end{proposition}

\begin{proof}
Let $\pi_n : \mathbb{Z} \rightarrow \mathbb{Z}/n\mathbb{Z}$ be the quotient map, and consider $G' := \pi_n^{-1}(G).$
By \ref{preimagesubgroup}, $G'$ is a subgroup of $\mathbb{Z}/n\mathbb{Z}$; by \ref{infinitecyclicsubgroup}, $G'$ is a
\emph{cyclic} subgroup of $\mathbb{Z}$, generated by a nonnegative integer $d$. It follows that\dots
$$G = \pi_n(G') = \pi_n(\langle d \rangle) = \langle [d]_n \rangle$$;
thus $G$ is indeed a cyclic subgroup of $\mathbb{Z}/n\mathbb{Z}$, generated by a class $[d]_n$. Further, since
$n \in G'$ (because $\pi_n(n) = [n]_n = [0]_n \in G$) and $G' = d\mathbb{Z}$, we see that $d$ divides $n$, as
claimed.
\end{proof}
