\subsection{Alternating Group}\label{alternatinggroup}

Let\dots
$$\Delta_n = \prod_{i \leq i < j \leq n}(x_i - x_j) \in \mathbb{Z}[x_1, \dots, x_n].$$

\subsubsection{Sign of a permutation}\label{signpermutation}
The \emph{sign} of a permutation $\sigma \in S_n$, denoted $(-1)^{\sigma}$, is determined
by the action of $\sigma$ on $\Delta_n$:
$$\Delta_n \sigma = (-1)^{\sigma} \Delta_n.$$
We say that a permutation is \emph{even} if if its sign is $+1$ and \emph{odd} if its sign is $-1$.

\subsubsection{Transposition}\label{transposition}
A \emph{transposition} is a cycle of length $2$.

\begin{lemma}
Transpositions generate $S_n$.
\end{lemma}

\begin{proof}
Indeed, by \ref{writingpermutations} it suffices to show that every \emph{cycle} is a product of transpositions, and indeed\dots
$$(a_1 \dots a_r) = (a_1a_2)(a_1a_3)\cdots(a_1a_r),$$
as may be checked by applying both sides to every element of $\{ 1, \dots, n \}$.
\end{proof}

\begin{lemma}
Let $\sigma = \tau_1 \cdots \tau_r$ be a product of transpositions. Then $\sigma$ is even, resp. odd, according to whether $r$ is even, resp., odd.
\end{lemma}

\begin{proof}
This follows immediately from the facts that $\varepsilon$ is a homomorphism and the sign of a transposition is $-1$: indeed, $(ij)$ acts on $\Delta_n$ by
permuting its factors and changing the sign of an odd number of factors (for $i < j$, the factor $(x_i - x_j)$ and the pairs of factors $(x_i - x_k)$, $(x_k - x_j)$
for all $i < k < j$).
\end{proof}

\subsubsection{Definition}
The \emph{alternating group} on $\{1, \dots, n \}$, denoted $A_n$, consists of all even permutations $\sigma \in S_n$. \newline

\noindent The alternating group is a \emph{normal} subgroup of $S_n$, and\dots
$$[S_n : A_n] = 2$$
for $n \geq 2$.

\subsubsection{Conjugacy}\label{conjugacyalternatinggroup}

\begin{lemma}
Let $n \geq 2$, and let $\sigma \in A_n$. Then $[\sigma]_{A_n} = [\sigma]_{S_n}$ or size of $[\sigma]_{A_n}$ is half the size of $[\sigma]_{S_n}$, according to whether
the centralizer $Z_{S_n}(\sigma)$ is not or is contained in $A_n$.
\end{lemma}

\begin{proof}
Not that\dots
$$Z_{A_n}(\sigma) = A_n \cap Z_{S_n}(\sigma):$$
this follows immediately from the definition of centralizer. Now recall that the centralizer of $\sigma$ is its stabilizer under conjugation, and therefore the size of the conjugacy
class of $\sigma$ equals the index of its centralizer.

If $Z_{S_n}(\sigma) \subseteq A_n$, then $Z_{A_n}(\sigma) = Z_{S_n}(\sigma)$, so that\dots
$$[S_n : Z_{S_n}(\sigma)] = [S_n : Z_{A_n}(\sigma)] = [S_n :A_n][A_n : Z_{A_n}(\sigma)] = 2 \cdot [A_n : Z_{A_n}(\sigma)];$$
therfore, $[\sigma]_{A_n}$ is half the size of $[\sigma]_{S_n}$ in this case.

If $Z_{S_n} \not \subseteq A_n$, then note that $A_nZ_{S_n}(\sigma) = S_n$: indeed, $A_nZ_{S_n}(\sigma)$ is a subgroup of $S_n$, and it properly contains $A_n$, so
it must equal $S_n$ as $A_n$ has index $2$ in $S_n$. By index considerations\dots
$$[A_n : Z_{A_n}(\sigma)] = [A_n : A_n \cap Z_{S_n}(\sigma)] = [A_nZ_{S_n}(\sigma) : Z_{S_n}(\sigma)] = [S_n : Z_{S_n}(\sigma)],$$
so the classes have the same size. Since $[\sigma]_{A_n} \subseteq [\sigma]_{S_n}$ in any case, it follows that $[\sigma]_{A_n} = [\sigma]_{S_n}$, completing the proof.
\end{proof}

\begin{proposition}
\label{conjugacyclassofalternatinggroup}
Let $\sigma \in A_n$, $n \geq 2$. Then the conjugacy class $\sigma$ in $S_n$ splits into two conjugacy classes in $A_n$ precisely if the type of $\sigma$
consists of distinct odd numbers.
\end{proposition}

\begin{proof}
By the previous lemma, we have to verify that $Z_{S_n}(\sigma)$ is contained in $A_n$ precisely when the stated condition is satisfied; that is, we have to show that\dots
$$\sigma = \tau \sigma \tau^{-1} \Rightarrow \tau \textrm{ is even}$$
precisely when the type of $\sigma$ consists of distinct odd numbers.

Write $\sigma$ in cycle notation (including cycles of length $1$):
$$\sigma = (a_1 \dots a_{\lambda})(b_1 \dots b_{\mu})\cdots(c_1\dots c_{\nu}),$$
and recall that\dots
$$\tau \sigma \tau^{-1} = (a_1\tau^{-1}\dots a_{\lambda}\tau^{-1})(b_1\tau^{-1}\dots b_{\mu}\tau^{-1})\cdots(c_1\tau^{-1}\dots c_{\nu}\tau^{-1}).$$

Assume that $\lambda, \mu, \dots , \nu$ are odd and distinct. If $\tau \sigma \tau^{-1} = \sigma$, then conjugation by $\tau$ must preserve each cycle in $\sigma$, as all cycle lengths are distinct:
$$\tau (a_1 \dots a_{\lambda})\tau^{-1} = (a_1 \dots a_{\lambda}), \textrm{ etc.}$$
that is,
$$(a_1 \tau^{-1} \dots a_{\lambda} \tau^{-1}) = (a_1 \dots a_{\lambda}), \textrm{ etc.}$$

This means that $\tau$ acts as a cyclic permutation on (e.g.) $a_1, \dots, a_{\lambda}$ and therefore in
the same way as a power of $(a_1 \dots a_{\lambda})$. It follows that\dots
$$\tau = (a_1 \dots a_{\lambda})^r(b_1 \dots b_{\mu})^s \dots (c_1 \dots c_{\nu})^t$$
for suitable $r,s, \dots, t$. Since all cycles have odd lengths, each cycle is an even permutation; and $\tau$ must then be even as it is a product of evne permutations. This proves that $Z_{S_n}(\sigma) \subseteq A_n$
if the stated condition holds.

Conversely, assume that the stated condition does not hold: that is, either some of the cycles in the cycle decomposition have even length or all have odd length but two of the cycles have the same length.

In the first case, let $\tau$ be an even-length cycle in the cycle decomposition of $\sigma$. Note that $\tau \sigma \tau^{-1} = \sigma$: indeed, $\tau$ commutes with itself and with all cycles in $\sigma$ other than $\tau$.
Since $\tau$ has even length, then it is odd as permutation: this shows that $Z_{S_n}(\sigma) \not subseteq A_n$, as needed.

In the second case, without loss of generality assume $\lambda = \mu$, and consider the odd permutation\dots
$$\tau = (a_1b_1)(a_2b_2)\cdots(a_{\lambda}b_{\lambda}):$$
conjugating by $\tau$ simply interchanges the first two cycles in $\sigma$; hence $\tau \sigma \tau^{-1} = \sigma$. As $\tau$ is odd, this again shows that $Z_{S_n}(\sigma) \not \subseteq A_n$, and we are done.
\end{proof}

\subsubsection{Simplicity}\label{simplictyalternatinggroup}

\begin{corollary}
The alternating group $A_5$ is a simple noncommutative group of order $60$.
\end{corollary}

\begin{proof}
A normal subgroup of $A_5$ is necessarily the union of conjugacy classes, contains the identity, and has order equal to a divisor of $60$. The divisors of $60$ other than $1$ and $60$ are\dots
$$2,3,4,5,6,10,12,15,20,30;$$
counting the elements other than the identity would give one of\dots
$$1,2,3,4,5,9,11,14,19,29$$
as a sum of numbers $\neq 1$ from the class formula for $A_5$. But the simply does not happen.

\end{proof}

\begin{lemma}
\label{alternatinggroupthreecycles1}
The alternating group $A_n$ is generated by $3$-cycles.
\end{lemma}

\begin{proof}
Since every even permutation is a product of an even number $2$-cycles, it suffices to show that every product of two $2$-cycles may be written as product of $3$-cycles. Therefore, consider a product\dots
$$(ab)(cd)$$
with $a \neq b$, $c \neq d$. If $(ab) = (cd)$, then this product is the identity, and their is nothing to prove. If $\{ a,b \}, \{ c,d \}$ have exactly one element in common, then we may assume $c = a$ and observe\dots
$$(ab)(ab) = (abd).$$
It $\{ a,b \}, \{ c,b \}$ are disjoint, then\dots
$$(ab)(cd) = (abc)(adc),$$
and we are done.
\end{proof}

\begin{proposition}
\label{alternatinggroupthreecycles2}
Let $n \geq 5$. If a normal subgroup of $A_n$ contains a $3$-cycle, then it contains all $3$-cycles.
\end{proposition}

\begin{proof}
Normal subgroups are unions of conjugacy classes, so we just need to verify that $3$-cycles form a conjugacy class in $A_n$, for $N \geq 5$. But they do in $S_n$, and the type of a
$3$-cycle is $[3,1,1,\dots]$ for $n \geq 5$; hence the conjugacy class does not split in $A_n$, by \ref{conjugacyclassofalternatinggroup}.
\end{proof}

\begin{theorem}
The alternating group $A_n$ is simple for $n \geq 5$.
\end{theorem}

\begin{proof}
We have already checked this for $n = 5$ and $n = 6$. For $n > 6$, let $N$ be a nontrivial normal subgroup of $A_n$; we will show that necessarily $N = A_n$, by proving that $N$ contains $3$-cycles.

Let $\tau \in N$, $\tau \neq (1)$, and let $\sigma \in A_n$ be a $3$-cycle. Since the center of $A_n$ is trivial and $3$-cycles generate $A_n$, we may assume that $\tau$ and $\sigma$ do not commute,
that is, the commutator\dots
$$[\tau, \sigma] = \tau(\sigma \tau^{-1} \sigma^{-1}) = (\tau \sigma^{-1})\sigma^{-1}$$
is not the identity. This element is in $N$ and is a product of two $3$-cycles.

Therefore, replaceing $\tau$ by $[ \tau, \sigma ]$ if necessary, we may assume that $\tau \in N$ is a nonindentity permutation acting on $\leq 6$ elements: that is, on a subset of a set
$T \subseteq \{ 1, \dots, n \}$ with $|T| = 6$. Now we may view $A_6$ as a subgroup of $A_n$, by letting it act on $T$. The subgroup $N \cap A_6$ of $A_g$ is then normal (because $N$ is normal) and
nontrivial (because $\tau \in N \cap A_6$ and $\tau \neq (1)$). Since $A_6$ is simple, this implies $N \cap A_6 = A_6$. In particular, $N$ contains $3$-cycles.

By \ref{alternatinggroupthreecycles2}, this implies that $N$ contains \emph{all} $3$-cycles. By \label{alternatinggroupthreecycles1}, it follows that $N = A_n$, as needed.
\end{proof}

\subsubsection{Solvability}\label{solvabilityalternatinggroup}

\begin{corollary}
For $n \geq 5$, the group $S_n$ is not solvable.
\end{corollary}

\begin{proof}
Since $A_n$ is simple, the sequence\dots
$$S_n \supset A_n \supset \{ (1) \}$$
is a composition series for $S_n$. It follows that the composition factors of $S_n$ are $\mathbb{Z} / 2 \mathbb{Z}$
and $A_n$. By \ref{solvablecharacterization}, $S_n$ is not solvable.
\end{proof}